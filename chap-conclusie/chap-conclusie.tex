\chapter{Conclusie en verder onderzoek}\label{sec:conclusie}
We hebben getoond hoe we een klassediagram kunnen vertalen naar FO($\cdot$). Die vertaling kunnen we gebruiken om de consistentie van het diagram te controleren. We hebben ook een meer algemene logische theorie opgesteld die men kan gebruiken om te controleren op een aantal mogelijke kwaliteitsgebreken.

We hebben ook een procedure beschreven om een verzameling sequentiediagrammen die elkaar onderling kunnen oproepen te vertalen naar een logische theorie. Speciale aandacht ging hier naar het in \'e\'en stap kunnen bepalen naar welke instructie precies moet gesprongen worden indien de uitvoering een gecombineerd fragment tegenkomt. Die vertaling en de vertaling van de klassediagram kan men tezamen gebruiken om de uitvoering van het gemodelleerde systeem of een deel ervan te simuleren. Het is ook mogelijk om modelexpansie te gebruiken om te verifi\"eren of een diagram voldoet aan vooropgelegde vereisten. In termen van rekentijd en \textit{grounding}-grootte kunnen simulatie en verificatie echter duur zijn om uit te voeren als de sequentiediagrammen samen een zekere omvang bereiken.

Wat betreft verder onderzoek is het duidelijk dat een vertaler voor de integratie van logica met UML-sequentiediagrammen voorgesteld in sectie \ref{sec:newlang} significant kortere theorie\"en---omdat instructies krachtiger zijn---en kleinere vocabularia---omdat er minder variabelen nodig zijn---kan opleveren. Dit zou vanzelfsprekend een grote invloed hebben op rekentijd en \textit{grounding}-grootte. Er zouden zo meer opties opengaan voor verscheidene verificaties. Als men bijvoorbeeld een stel diagrammen heeft dat het spel Reversi modelleert, zou men kunnen controleren of die diagrammen toelaten dat een speler een ongeldige zet doet.

Een verdere uitdaging bestaat erin om gegeven een structuur te controleren of die structuur consistent is met een volledige of gedeeltelijke uitvoering van een sequentiediagram. Zo kan men bijvoorbeeld controleren of een structuur die overeenkomt met de zetten die een bepaalde speler doet geldig zijn volgens een bepaald stel diagrammen dat een spel modelleert. Liefst zou zulk een structuur zoveel mogelijk onafhankelijk zijn van interne boekhouding van de diagrammen. De procedure moet kunnen afleiden welke logische symbolen overeenkomen met logische symbolen van de vertaling van de diagrammen en welke symbolen keuzes van de speler aanduiden. Progressie\"inferentie moet kunnen gebruikt worden om die keuzes in rekening te brengen en zo tot een antwoord te komen of de structuur overeenkomt met een model voor de theorie die als resultaat wordt gegeven door het diagramvertaalproces.