\chapter{Conclusie en verder onderzoek}\label{sec:conclusie}
We hebben getoond hoe we een klassediagram kunnen vertalen naar FO($\cdot$). Die vertaling kunnen we gebruiken om de consistentie van het diagram te controleren. We hebben ook een alternatieve voorstellingswijze ontworpen die men kan gebruiken om te controleren op een aantal mogelijke kwaliteitsgebreken.

We hebben ook een procedure beschreven om een verzameling sequentiediagrammen die elkaar onderling kunnen oproepen te vertalen naar een logische theorie. Speciale aandacht ging hier naar het in \'e\'en stap kunnen bepalen naar welke instructie precies moet gesprongen worden indien de uitvoering een gecombineerd fragment tegenkomt. Die vertaling en de vertaling van de klassediagram kan men tezamen gebruiken om de uitvoering van het gemodelleerde systeem of een deel ervan te simuleren met progressie\"inferentie. Het is ook mogelijk om modelexpansie te gebruiken om te verifi\"eren of een diagram voldoet aan vooropgelegde vereisten en ook of een gewenste eigenschap van het systeem als geheel geldt doorheen de uitvoering van het systeem. In termen van rekentijd, \textit{grounding}-grootte en virtueel geheugengebruik kunnen simulatie en verificatie echter duur zijn om uit te voeren als de sequentiediagrammen samen een zekere omvang bereiken.

We hebben een aanzet gegeven om declaratieve constructies aan te bieden voor de ontwerptaal voor sequentiediagrammen. Het basisprincipe achter de nieuwe instructies is dat we toelaten om de toestand van alle objecten in een verzameling tegelijkertijd op te vragen of aan te passen. Modelexpansie en progressie\"inferentie zijn voor de resulterende theorie\"en significant performanter in termen van rekentijd en geheugengebruik dan voor theorie\"en die resulteren uit de voorstellingsmethode die we voordien hadden ontworpen. Dit komt omdat instructies nu krachtiger zijn en vocabularia kleiner, wat leidt tot compactere theorie\"en en kleinere zoekruimtes.

Wat betreft verder onderzoek mist er nog een mechanisme om meerdere oproepen van een sequentiediagram tegelijkertijd toe te laten. Dit zou leiden tot meerdere acties op het systeem tegelijkertijd per tijdstap, en vermoedelijk tot nog betere performantie in termen van rekentijd en geheugengebruik. Er zouden immers minder tijdstappen nodig zijn om een taak uit te voeren waar een methode wordt opgeroepen op een verzameling van objecten. We stellen hiervoor twee alternatieven voor: E\'en waarbij parallele oproepen altijd zonder meer zijn toegelaten, en \'e\'en waar een methode expliciet wordt gemarkeerd als parallel uitvoerbaar en geen andere parallele oproepen toelaat terwijl er nog een oproep actief is. Het eerste alternatief is waarschijnlijk minder performant door de benodigde boekhouding, maar geeft meer flexibiliteit voor een ontwerp. Het tweede alternatief is waarschijnlijk performanter omdat er minder boekhouding nodig is, maar beperkt de mogelijkheden voor een ontwerp.

Een verdere uitdaging bestaat erin om gegeven een structuur te controleren of die structuur consistent is met een volledige of gedeeltelijke uitvoering van een sequentiediagram. Zo kan men bijvoorbeeld controleren of een structuur die overeenkomt met de zetten die een bepaalde speler doet geldig zijn volgens een bepaald stel diagrammen dat een spel modelleert. Liefst zou zulk een structuur zoveel mogelijk onafhankelijk zijn van interne boekhouding van de diagrammen. De procedure moet kunnen afleiden welke logische symbolen overeenkomen met logische symbolen van de vertaling van de diagrammen en welke symbolen keuzes van de speler aanduiden. Progressie\"inferentie moet kunnen gebruikt worden om die keuzes in rekening te brengen en zo tot een antwoord te komen of de structuur overeenkomt met een model voor de theorie die als resultaat wordt gegeven door het diagramvertaalproces.