\chapter{Relevante literatuur}
In de literatuur kan men verscheidene methoden om UML-diagrammen te vertalen naar logica onderscheiden. Het gaat voornamelijk over klassediagrammen, terwijl sommige werken ook sequentiediagrammen en toestandsdiagrammen beschouwen. De logica waarnaar vertaald wordt is vaak eerste-orde-predikatenlogica, maar een aantal onderzoekers verkiezen talen die behoren tot de klasse van de description logics.

In deze sectie bespreken we een aantal van deze werken.

\section{Redeneren op UML-klassediagrammen gebruikmakend van description logics}
In \textit{Reasoning on class diagrams} \cite{BerardiDaniela2005RoUc} toont men eerst een methode om een klassediagram automatisch te vertalen naar eerste-orde logica.