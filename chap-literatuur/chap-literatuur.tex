\chapter{Relevante literatuur}
In de literatuur kan men verscheidene methoden om UML-diagrammen te vertalen naar logica onderscheiden. Het gaat voornamelijk over klassediagrammen, terwijl sommige werken ook sequentiediagrammen en toestandsdiagrammen beschouwen. De logica waarnaar vertaald wordt is vaak eerste-orde-predikatenlogica, maar een aantal onderzoekers verkiezen talen die behoren tot de klasse van de description logics.

In deze sectie bespreken we een aantal van deze werken.

\section{Redeneren op UML-klassediagrammen gebruikmakend van description logics}
In \cite{BerardiDaniela2005RoUc} toont men eerst een methode om een klassediagram automatisch te vertalen naar eerste-orde logica.

Voor elke klasse definieert men een unair predicaat dat lidmaatschap van die klasse uitdrukt. Vervolgens definieert men voor alle attributen van een klasse een predikaat dat een object van die klasse in verband brengt met een waarde voor dat attribuut. Elke methode krijgt ook een predikaat dat voor combinaties van object waar de methode voor wordt opgeroepen en een stel van parameterwaarden een resultaat definieert. In de uitvoertheorie voegt men dan voorwaarden toe waaraan beantwoord moet worden. Een object heeft zoveel verschillende waardes voor een bepaald attribuut zoals opgelegd door het diagram. Voor attribuutwaarden, parameterwaarden en resultaatwaarden wordt afgedwongen dat ze van het juiste type zijn. Een bepaalde combinatie van oproepobject en parameterwaarden moet ook een uniek resultaat hebben bij een oproep.

Associaties krijgen een predikaat dat objecten van de betrokken klasses in verband brengt. De uitvoertheorie dwingt af dat de objecten van het juiste type zijn en dat er gehoorzaamd wordt aan de gespecificeerde multipliciteiten.

Om overerving te modelleren, stelt de uitvoertheorie dat als een object een lid is van een subklasse, dat het dan ook een lid is van de superklasse. Voor een subklasse $C_1$ van superklasse $C$ wordt dit uitgedrukt als $\forall{x}(C_1(x) \Rightarrow C(x)$. Merk op dat deze aanpak niet uitsluit dat een object een lid kan zijn van twee klasses die niet in verband staan met elkaar in een overervingshi\"erarchie, zelfs als de ontwerper dit niet wenst.

De paper stelt een aantal redeneringstaken voor die mogelijk zouden moeten zijn voor deze modelleringsmethode. Allereerst is er controle op consistentie van het gehele diagram, wat wil zeggen of er minstens \'e\'en model bestaat van het diagram waarvan de interpretatie voor minstens \'e\'en klasse een object bevat. Verder is er klasseconsistentie, namelijk of voor een klasse een model bestaat waarvoor de interpretatie van die klasse niet leeg is; klassesubsumptie, wat betekent dat het diagram impliceert dat de ene klasse een subklasse is van een andere klasse; klasse\"equivalentie, wat betekent het diagram oplegt dat voor twee klasses de interpretatie dezelfde is in alle modellen en dat \'e\'en van die klasses dus uit het diagram kan worden gehaald; en het detecteren van eigenschappen van klasses en associaties die maken dat bepaalde gespecificeerde multipliciteiten en typering in het diagram eigenlijk strenger zijn dan expliciet neergeschreven in het diagram. De auteurs nodigen verder uit om nog andere redeneringstaken te bedenken.

De rest van de paper handelt over description logics \todo{Nederlandstalige term?}. Men geeft een beschrijving van de talen $\mathcal{DLR}_{ifd}$, $\mathcal{ALCQI}$ en $\mathcal{ALC}$, beschrijft hoe men een klassediagram kan vertalen naar $\mathcal{DLR}_{ifd}$ en geeft men een gevalstudie van software die een specificatie van een klassediagram vertaalt naar een beschrijving in $\mathcal{ALCQI}$ en kan antwoorden op queries zoals of een object van een bepaalde klasse exact eenmaal deelneemt aan een bepaalde associatie zonder dat het diagram dit expliciet oplegt. Men concludeert dat dit soort van redeneringstaak behoort tot de klasse van EXPTIME-harde problemen.

\section{Het modelleren van klassediagrammen en OCL-constraints in relationele logica}
In \cite{KuhlmannMirco2012FUaO} geeft men een inleiding tot relationele logica en de semantiek ervan. Men beschrijft hoe een instantie van een model van relationele logica wordt opgebouwd door relaties die atomen of tupels van atomen bevatten en welke operaties op relaties beschikbaar zijn. Relationele logica biedt relationele bewerkingen zoals joins, cartesisch product en transitieve sluiting aan. Verder kan men ook gebruikmaken van verzamelingcomprehensie, bewerkingen op verzamelingen zoals unie en deelverzameling, booleaanse operatoren zoals conjunctie en implicatie, universele en existenti\"ele kwantoren en bewerkingen op gehele getallen. In de paper vertaalt men klassediagrammen en OCL-constraints \todo{Nederlandstalige term?} op zulk een manier dat er rechtstreeks gebruik wordt gemaakt van de structuur aangeboden door relationele logica. Op die manier bekomt men modellen die op een effici\"ente manier verwerkt kunnen worden ten koste van significante beperkingen op diagrammen en OCL-constraints die kunnen dienen als invoer.