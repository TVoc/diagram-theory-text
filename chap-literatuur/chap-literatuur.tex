\chapter{Relevante literatuur}\label{sec:literatuur}
In de literatuur kan men verscheidene methoden om UML-diagrammen te vertalen naar een vari\"eteit aan logica's onderscheiden. Het gaat voornamelijk over klassediagrammen, terwijl sommige werken ook activiteitsdiagrammen en toestandsdiagrammen beschouwen. De logica waarnaar vertaald wordt is vaak eerste-orde-predicatenlogica, maar een aantal onderzoekers verkiezen talen die behoren tot de klasse van de description logics of relationele logica.

In deze sectie bespreken we een aantal van deze werken.

\section{Redeneren op UML-klassediagrammen gebruikmakend van description logics}
In \cite{BerardiDaniela2005RoUc} toont men eerst een methode om een klassediagram automatisch te vertalen naar eerste-orde logica.

Voor elke klasse definieert men een unair predicaat dat lidmaatschap van die klasse uitdrukt. Vervolgens definieert men voor alle attributen van een klasse een predicaat dat een object van die klasse in verband brengt met een waarde voor dat attribuut. Elke methode krijgt ook een predicaat dat voor combinaties van object waar de methode voor wordt opgeroepen en een stel van parameterwaarden een resultaat definieert. In de uitvoertheorie voegt men dan voorwaarden toe waaraan beantwoord moet worden. Een object heeft zoveel verschillende waardes voor een bepaald attribuut als opgelegd door het diagram. Voor attribuutwaarden, parameterwaarden en resultaatwaarden wordt afgedwongen dat ze van het juiste type zijn. Een bepaalde combinatie van oproepobject en parameterwaarden moet ook een uniek resultaat hebben bij een oproep.

Associaties krijgen een predicaat dat objecten van de betrokken klasses in verband brengt. De uitvoertheorie dwingt af dat de objecten van het juiste type zijn en dat er gehoorzaamd wordt aan de gespecificeerde multipliciteiten.

Om overerving te modelleren, stelt de uitvoertheorie dat als een object een lid is van een subklasse, dat het dan ook een lid is van de superklasse. Voor een subklasse $C_1$ van superklasse $C$ wordt dit uitgedrukt als $\forall{x}(C_1(x) \Rightarrow C(x))$. Merk op dat deze aanpak niet uitsluit dat een object een lid kan zijn van twee klasses die niet in verband staan met elkaar in een overervingshi\"erarchie, zelfs als de ontwerper dit niet wenst.

De paper stelt een aantal redeneertaken voor die mogelijk zouden moeten zijn voor deze modelleringsmethode. Allereerst is er controle op consistentie van het gehele diagram, wat wil zeggen of er minstens \'e\'en model bestaat van het diagram waarvan de interpretatie voor minstens \'e\'en klasse een object bevat. Verder is er klasseconsistentie, namelijk of voor een klasse een model bestaat waarvoor de interpretatie van die klasse niet leeg is; klassesubsumptie, wat betekent dat het diagram impliceert dat de ene klasse een subklasse is van een andere klasse; klasse\"equivalentie, wat betekent dat het diagram oplegt dat voor twee klasses de interpretatie gelijk is in alle modellen en dat \'e\'en van die klasses dus verwijderd kan worden uit het diagram; en het detecteren van eigenschappen van klasses en associaties die ervoor zorgen dat bepaalde gespecificeerde multipliciteiten en typering in het diagram eigenlijk strenger zijn dan expliciet neergeschreven in het diagram.

De rest van de paper handelt over \textit{description logics}. Men geeft een beschrijving van de talen $\mathcal{DLR}_{ifd}$, $\mathcal{ALCQI}$ en $\mathcal{ALC}$, beschrijft hoe men een klassediagram kan vertalen naar $\mathcal{DLR}_{ifd}$ en geeft men een gevalstudie van software die een specificatie van een klassediagram vertaalt naar een beschrijving in $\mathcal{ALCQI}$ en kan antwoorden op queries zoals of een object van een bepaalde klasse exact eenmaal deelneemt aan een bepaalde associatie zonder dat het diagram dit expliciet oplegt. Men concludeert dat dit soort van redeneertaak behoort tot de klasse van EXPTIME-harde problemen.

De methode om klassediagrammen te vertalen naar eerste-orde-predicatenlogica voorgesteld in deze paper dient als de basis voor onze eigen methode gepresenteerd in hoofdstuk \ref{sec:consistentie}. We passen deze methode aan om gebruik te maken van concepten aangeboden door FO($\cdot$)\cite{DeCatBroes2014PLaa}.

\section{Het modelleren van klassediagrammen en OCL-constraints in relationele logica}
In \cite{KuhlmannMirco2012FUaO} geeft men een inleiding tot relationele logica en de semantiek ervan. Men beschrijft hoe een instantie van een model van relationele logica wordt opgebouwd door relaties die atomen of tupels van atomen bevatten en welke operaties op relaties beschikbaar zijn. Relationele logica biedt relationele bewerkingen zoals \textit{joins}, cartesisch product en transitieve sluiting aan. Verder kan men ook gebruikmaken van verzamelingscomprehensie, bewerkingen op verzamelingen zoals unie en deelverzameling, booleaanse operatoren zoals conjunctie en implicatie, universele en existenti\"ele kwantoren en bewerkingen op gehele getallen. In de paper vertaalt men klassediagrammen en beperkingen uitgedrukt in OCL-voorwaarden\cite{WarmerJosB1999Ocl:} op zulk een manier dat er rechtstreeks gebruik wordt gemaakt van de structuur aangeboden door relationele logica. Op die manier bekomt men modellen die op een effici\"ente manier verwerkt kunnen worden ten koste van significante beperkingen op diagrammen en OCL-voorwaarden die kunnen dienen als invoer.

Men schetst hoe primitive types voorgesteld worden door unaire relaties van atomen. In relationele logica worden objecten voorgesteld door constanten, en deze constanten worden gebruikt als atomen in binaire relaties die objecten verbinden met waarden voor klasseattributen. Eveneens brengt men objecten in verband met elkaar in relaties die associaties voorstellen. Er worden voorwaarden gelegd op relaties die attributen en associaties voorstellen zodanig dat de gebruikte objecten van het juiste type zijn en dat de multipliciteiten gespecificeerd in het diagram afgedwongen worden. Aangezien men relaties op zulk een eenvoudige wijze gebruikt, is het eenvoudig om een stel van relaties en type- en multipliciteitsvoorwaarden te vertalen naar een klassediagram.

Men beschrijft hoe uitdrukkingen in OCL vertaald kunnen worden naar relationele logica. Bewerkingen op booleaanse waarden, gehele getallen, het allesomvattende type OclAny\cite{WarmerJosB1999Ocl:}, bewerkingen op verzamelingen met \textit{collect} in het bijzonder en navigatie van associaties worden ondersteund.

De auteurs gebruiken Kodkod\cite{10.1007/978-3-540-71209-1_49} om geldige instanties van een relationeel model te berekenen. Kodkod vertaalt relationele modellen naar SAT-formules en vertaalt oplossingen naar instanties van dat relationeel model.

\section{Klassediagrammen en activiteitsdiagrammen modelleren met \textit{Concurrent Transaction Frame Logic}}

In \cite{RamalhoFranklin2004CTFL} vertaalt men klassediagrammen en activiteitsdiagrammen\cite{RumbaughJames2005Tuml} naar uitvoerbare programma's in Concurrent Transaction Frame Logic\cite{kifer1995deductive,kifer1996concurrency} (CTFL). CTFL is een integratie van twee uitbreidingen van eerste-orde Horn-logica: Frame Logic (FL), wat objecten en overerving kan modelleren; en Concurrent Transaction Logic (CTL), wat gedrag betreffende simultane logische databank updates, transacties, procescommunicatie en tijdsgebonden randvoorwaarden betreffende uitvoering modelleert.

FL biedt een syntaxis aan waarmee men een klasse kan defini\"eren. Men specificeert een klassenaam, een klasse die dient als superklasse, een lijst van attributen met hun type en een lijst van methodes met parameters en hun type en een resultaattype. Objectcreatie wordt ook aangeboden, waarbij men de ge\"instantieerde klasse moet opgeven en ook attribuutwaarden en combinaties van argumenten en resultaat voor methodes. Deze twee soorten termen noemt men \textit{F-Molecules}. Binnen \textit{F-Molecules} worden logische variabelen ondersteund in alle mogelijke velden van klassedefinitie of objectinstantiatie. De auteurs tonen hoe klasses, associaties en overerving eenvoudig kunnen worden gemodelleerd met deze \textit{F-Molecules}.

CTL breidt Horn-logica uit met vijf nieuwe connectieven: Seri\"ele conjunctie, seri\"ele disjunctie, concurrente conjunctie, concurrente disjunctie en atomische modaliteit. Deze laatste voorkomt gedeeltelijke uitvoering van een formule. Als de uitvoering wordt onderbroken of faalt, moet de toestand van alle betrokken variabelen en objecten teruggebracht worden zoals die was v\'o\'or de uitvoering van de formule begon. CTL definieert ook enkele bewerkingen die atomische veranderingen en synchronisatie voor hun rekening nemen. De auteurs tonen hoe de verscheidene elementen van een activiteitsdiagram, zijnde \textit{action states}, \textit{activity states}, \textit{fork} en \textit{join pairs}, \textit{synch states} en vertakkingen binnen een activiteitsdiagram gemodelleerd kunnen worden met behulp van CTFL-termen en deze vijf connectieven. Vervolgens wordt getoond hoe de genoemde klasses, attributen en methodes in een activiteitsdiagram in verband worden gebracht met hun definities in het klassediagram.

Het resultaat van deze vertaling is een uitvoerbaar programma in CTFL waarmee men de consistentie van het ontwerp in het klassediagram en de activiteitsdiagrammen kan controleren.

\parbreak

In het voorgaande merken we dat de literatuur zich voornamelijk toelegt op het vertalen van UML-diagrammen naar talen die een deelverzameling zijn van eerste-orde-predicatenlogica. Op deze manier wordt er ingeboet aan expressiviteit ten voordele van rekentijd en ruimtegebruik. We denken echter aan het modelleren van overerving: Het bepalen van tot welke klasses een object behoort ten gevolge van de gespecificeerde klassehi\"erarchie\"en is een voorbeeld van het berekenen van een transitieve sluiting, wat niet uit te drukken is in eerste-orde-predicatenlogica en dus ook niet in deze deelverzamelingen. We merken ook op dat het ontbreekt aan een simulatiemethode voor sequentiediagrammen. In deze masterproef richten we ons op sequentiediagrammen in plaats van activiteitsdiagrammen omdat voorgaande bovenop het modelleren van interacties tussen objecten ook toelaten om variabelen te defini\"eren.