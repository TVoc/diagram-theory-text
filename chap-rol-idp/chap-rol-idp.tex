\chapter{Het kennisbanksysteem IDP}\label{sec:rol-idp}
\section{Korte inleiding in IDP}
IDP is een kennisbanksysteem\cite{DeCatBroes2014PLaa}. Deze kennisbanken zijn opgesteld in FO($\cdot$), een uitbreiding van predicatenlogica. De basisblokken van een specificatie in IDP zijn als volgt:

\begin{itemize}
	\item \textbf{Vocabularium}: Hier specificeert de ontwerper de logische types die bestaan in het beschouwd domein en de predicaten en functies over die logische types.
	\item \textbf{Theorie}: Hier schrijft de ontwerper zinnen in FO($\cdot$) die bepalen welke structuren over het beschouwde vocabularium modellen zijn. Indien de ontwerper een inconsistente theorie ontwerpt, zullen er geen modellen zijn.
	\item \textbf{Structuur}: De ontwerper vult hier de logische types gedefinieerd in het vocabularium in. \textit{Constructed types} hebben geen invulling nodig aanegzien ze al volledig worden gespecificeerd in het vocabularium. De ontwerper kan ook voor \'e\'en of meerdere predicaten aangeven welke tupels wel of geen lid zijn. Hij kan ook voor \'e\'en of meerdere functies specificeren welke elementen uit het domein afgebeeld worden op welk element uit het codomein.
\end{itemize}

De ontwerper kan meerdere vocabularia, theorie\"en en structuren neerschrijven. Elke theorie en structuur kan wel maar de symbolen van \'e\'en vocabularium gebruiken.

IDP gebruikt zijn eigen symbolen voor universele kwantoren, existenti\"ele kwantoren en logische connectieven. \cite{DeCatBroes2014PLaa} bevat een uitgebreidere inleiding tot IDP.

\section{Gebruik van modeluitbreiding}
Gegeven een structuur over een bepaald vocabularium kan de gebruiker de opdracht geven aan IDP om een uitbreiding te vinden van deze structuur die ervoor zorgt de structuur een model is van een gegeven theorie. Dit is een vorm van inferentie die men \textbf{modeluitbreiding} noemt. Het kan echter het geval zijn dat IDP antwoordt dat zulk een uitbreiding niet bestaat of dat de uitvoering nooit eindigt.


\subsection{Controleren van consistentie}
In bijlage \ref{app:consistentie} staat de logische theorie die werd gegenereerd volgens de regels uitgelijnd in hoofdstuk \ref{sec:consistentie}. Als men dit geeft als invoer aan IDP, is het besluit dat er een model bestaat voor de theorie en dat het diagram inderdaad consistent is.

\subsection{Detecteren van kwaliteitsgebreken}
In bijlage \ref{app:kwaliteitsgebrek} staat de logische theorie die werd gegenereerd uit een combinatie van de diagrammen uit figuren \ref{fig:diagram-voorbeeld} en \ref{fig:hierarchie} volgens de regels uitgelijnd in hoofdstuk \ref{sec:kwaliteitsgebrek}. IDP vindt alle many-to-many associaties, besluit dat \textit{Loose} een losstaande klasse is en dat de associatie \textit{B}---\textit{D} overbodig is door de samenloop van de klassehi\"erarchie en de opgelegde multipliciteiten.