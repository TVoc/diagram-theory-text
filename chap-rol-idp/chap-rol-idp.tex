\chapter{Het kennisbanksysteem IDP}\label{sec:rol-idp}
\subsection{Inleiding tot IDP}
IDP is een kennisbanksysteem\cite{DeCatBroes2014PLaa}. Deze kennisbanken zijn opgesteld in FO($\cdot$), een uitbreiding van predicatenlogica.

IDP bouwt verder op eerder werk in het onderzoeksdomein van het logisch programmeren. Het ondersteunt constructies gebaseerd op logisch programmeren, in het bijzonder inductieve definities, maar breekt ook met enkele fundamentele idee\"en uit dat paradigma. Logische theorie\"en opgesteld in FO($\cdot$) zijn geen uitvoerbare programma's---ze zijn beschrijvingen van mogelijke toestanden in het beschouwde probleemdomein. De fundamentele oplossingsstrategie voor problemen in een toepassingsdomein waar IDP zich op richt is het specificeren van informatie over het domein in logica. De gebruiker past dan een vorm van inferentie toe om een antwoord te krijgen op concrete vragen geformuleerd in termen van het toepassingsdomein.

Bij het opstellen van logische theorie\"en ondersteunt IDP alle concepten uit de klassieke eerste-orde-predicatenlogica: Constanten, variabelen, predicaten, functies en existenti\"ele en universele kwantoren. IDP ondersteunt ook concepten die eerste-orde-predicantelogica uitbreiden. Concreet gaat het over de volgende vier concepten:

\begin{itemize}
	\item \textbf{Logische types}: Alle variabelen gebruikt in een logische zin hebben een type. Dit betekent ook dat alle argumenten van een predicaat en functie en het resultaat van een functie een type hebben.
	\item \textbf{Aggregaten}: Dit zijn de functies \textit{cardinaliteit}, \textit{som}, \textit{product}, \textit{minimum} en \textit{maximum}. De ontwerper specificeert \'e\'en variabele of een tupel van variabelen. Hij bindt die variabele of tupel dan in een logische zin. Als de ontwerper \'e\'en variabele specificeert, worden alle waarden die voldoen aan de logische zin als argument doorgegeven aan de aggregaatfunctie. Als de ontwerper een tupel specificeert, dan wordt voor elke tupel die voldoet aan de zin het eerste element doorgegeven aan de aggregaatfunctie.
	\item \textbf{Inductieve definities}: Dit is een mechanisme om de interpretatie van een predicaat of functie te defini\"eren analoog aan hoe een concept kan gedefinieerd worden in een recursieve definitie in de wiskunde of in verscheidene domeinen van de wetenschap.
	\item \textbf{Parti\"ele functies}: IDP laat toe om een functie te specificeren als \textit{partieel}. Dit betekent dat niet alle elementen van het domein noodzakelijk worden afgebeeld op een element uit het codomein.
\end{itemize}

De inferentievormen die IDP ondersteunt zijn: Modelexpansie gegeven een deels ingevulde structuur; bepalen of een structuur een model is voor een theorie en/of bepalen of een structuur kan uitgebreid worden tot een model voor de theorie; het vinden van optimale modellen gegeven een aggregate term; propagatie, wat gegeven een structuur en een theorie een preciezere structuur geeft die alle oplossingen behoudt; \textit{query}-inferentie, wat alle objecten die beantwoorden aan een bepaalde \textit{query} ophaalt uit een gegeven structuur; deductie; en, gegeven een deels ingevulde structuur en theorie, het berekenen van symmetrische structuren over de theorie.

De basisblokken van een specificatie in IDP zijn als volgt:

\begin{itemize}
	\item \textbf{Vocabularium}: Hier specificeert de ontwerper de logische types die bestaan in het beschouwd domein en de predicaten en functies over die logische types.
	\item \textbf{Theorie}: Hier schrijft de ontwerper zinnen in FO($\cdot$) die bepalen welke structuren over het beschouwde vocabularium modellen zijn. Indien de ontwerper een inconsistente theorie ontwerpt, zullen er geen modellen zijn.
	\item \textbf{Structuur}: De ontwerper vult hier de logische types gedefinieerd in het vocabularium in. \textit{Constructed types} hebben geen invulling nodig aangezien ze al volledig worden gespecificeerd in het vocabularium. De ontwerper kan ook voor \'e\'en of meerdere predicaten aangeven welke tupels wel of geen lid zijn. Hij kan ook voor \'e\'en of meerdere functies specificeren welke elementen uit het domein afgebeeld worden op welk element uit het codomein.
\end{itemize}

De ontwerper kan meerdere vocabularia, theorie\"en en structuren neerschrijven. Elke theorie en structuur kan wel maar de symbolen van \'e\'en vocabularium gebruiken.

Broes De Cat et al.\cite{DeCatBroes2014PLaa} geven een uitgebreidere inleiding tot IDP.