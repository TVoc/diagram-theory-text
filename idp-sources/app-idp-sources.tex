\chapter{IDP-bestand resulterend uit de procedure beschreven in hoofdstuk \ref{sec:consistentie}}\label{app:consistentie}

\lstinputlisting[title=Codebestand 0.4.3]{chap-rol-idp/generatedtheory.idp}\label{code:consistentie}

\chapter{IDP-bestand resulterend uit de procedure beschreven in hoofdstuk \ref{sec:kwaliteitsgebrek}}\label{app:kwaliteitsgebrek}

\lstinputlisting[title=Theorie voor het opsporen van kwaliteitsgebreken]{chap-rol-idp/defs.idp}\label{code:kwaliteitsgebrek}

\chapter{IDP-bestand voor sequentiediagram van het spelvoorbeeld}\label{app:seq-diagram-game}

\lstinputlisting[title=Modellering van het gedrag van het sequentiediagram in figuur \ref{fig:seq-diagram-game}]{idp-sources/seq-diagram-game.idp}\label{code:seq-diagram-game}

\chapter{IDP-bestand voor sequentiediagram voor het voorbeeld over recursie}\label{app:seq-recursion}

\lstinputlisting[title=Modellering van het gedrag van de sequentiediagrammen in figuur \ref{fig:seq-recursion}]{idp-sources/recursion.idp}\label{code:seq-recursion}

\chapter{IDP-bestand voor sequentiediagram voor het voorbeeld over extra instructies}\label{app:new-nim}

\lstinputlisting[title=Modellering van Nim gebruikmakend van nieuwe soorten instructies]{idp-sources/new-lang.idp}\label{code:new-nim}

\chapter{IDP-bestand voor het ontwerp van Nim}

\lstinputlisting[title=Modellering van Nim voor hoofdstuk \ref{sec:evaluatie}]{chap-evaluatie/nimmodel.idp}\label{code:nim-eval}

\chapter{Wetenschappelijk artikel}

\includepdf[pages=-]{IEEEtran/article.pdf}

\chapter{Poster}

\includepdf[pages=1,fitpaper]{postera4.pdf}