\RequirePackage{filecontents}
\begin{filecontents*}{\jobname-url.mst}
	% Input style specifiers
	keyword "\\urlentry"
	% Output style specifiers
	preamble "\\begin{theurls}"
	postamble "\n\\end{theurls}\n"
	group_skip ""
	headings_flag 0  
	item_0 "\n\\urlitem{"
		delim_0 "}{"
		delim_t "}"
	line_max 500
\end{filecontents*}

\documentclass[master=cws,masteroption=gs,dutch]{kulemt}
\setup{title={Vertaling van UML-klassediagrammen en sequentiediagrammen naar FO($\cdot$)-theorie\"en voor simulatie en verificatie},
	author={Thomas Vochten},
	promotor={Prof. dr. Marc Denecker},
	assessor={Dr. Bart Bogaerts \\ Dr. Dominique Devriese},
	assistant={Ir. Matthias van der Hallen}}

\setup{coverpageonly}
\setup{font=lm}

\setup{filingcard,
	translatedtitle={Translating UML Class Diagrams and Sequence Diagrams to FO($\cdot$) to Facilitate Simulation and Verification},
	articletitle={Translating UML Class Diagrams and Sequence Diagrams to FO($\cdot$) to Facilitate Simulation and Verification},
	udc=681.3,
	shortabstract={ UML is een grafische taal die een aantal soorten diagrammen aanbiedt om een mogelijk ontwerp voor een gewenst softwaresysteem op te stellen. Als de ontwerper echter een groot aantal diagrammen opstelt en de diagrammen een zekere omvang bereiken, is het makkelijk om het overzicht te verliezen. Hierdoor worden er sneller fouten tegen de gewenste specificaties voor het softwaresysteem gemaakt en worden die fouten minder makkelijk opgemerkt. We bieden een gereedschap aan om automatisch klassediagrammen en sequentiediagrammen te verificeren. We beschrijven een methode om een klassediagram voor te stellen in FO($\cdot$). Met deze voorstellingen kunnen we consistentie van een klassediagram aantonen en bepaalde soorten inconsistentie bewijzen. We kunnen ook de aanwezigheid van een aantal soorten kwaliteitsgebreken die een diagram minder verstaanbaar maken aantonen. Deze voorstellingsmethode voor klassediagrammen breiden we uit zodat we bijhorende sequentiediagrammen kunnen voorstellen in de lineaire tijdscalculus. Op deze manier kunnen we sequentiediagrammen simuleren, de uitvoer van een sequentiediagram verificeren en nagaan of gewenste eigenschappen van het systeem als geheel gelden doorheen de uitvoering ervan. Dit gaat echter gepaard met een lange rekentijd en hoog geheugengebruik. De laatste bijdrage is daarom een aanzet om logica te integreren in de ontwerptaal beschikbaar voor sequentiediagrammen. We tonen dat de uitgebreide ontwerptaal leidt tot een significant compactere modellering van een softwaresysteem. Dit leidt tot een significant kortere rekentijd en kleiner geheugengebruik bij inferentie op de bekomen voorstelling van het systeem. }}

%\usepackage[pdfusetitle,colorlinks,plainpages=false]{hyperref}
\usepackage[hidelinks]{hyperref}
\usepackage{float}
\usepackage{todonotes}
\usepackage{amsmath}
\usepackage{amssymb}
\usepackage{listings}
\usepackage{svg}
\svgsetup{clean=true}
\usepackage{relsize}
\usepackage{pdflscape}
\usepackage{graphicx}
\usepackage{subcaption}
\usepackage[ruled,vlined,linesnumbered]{algorithm2e}
\usepackage{xifthen}
\usepackage{xparse}
\usepackage{pdfpages}
\usepackage{rotating}
\usepackage{afterpage}
\usepackage{amsthm}
\usepackage{pdfescape}
\usepackage{xstring}

% ----------- LIST OF URLS -------------

\makeatletter
\newwrite\file@url
\openout\file@url=\jobname-url.idx\relax

\newcommand*{\write@url}[1]{%
	\begingroup
	\EdefEscapeHex\@tmp{#1}%
	\protected@write\file@url{}{%
		\protect\urlentry{\@tmp}{\thepage}%
	}%
	\endgroup
}
\let\saved@hyper@linkurl\hyper@linkurl
\renewcommand*{\hyper@linkurl}[2]{%
	\write@url{#2}%
	\saved@hyper@linkurl{#1}{#2}%
}
\newcommand*{\listurlname}{Lijst van gebruikte online locaties}
\newcommand*{\printurls}{%
	\InputIfFileExists{\jobname-url.ind}{}{}%
}
\newenvironment{theurls}{%
	\section*{\listurlname}%
	\@mkboth{\listurlname}{\listurlname}%
	\let\write@url\@gobble  
	\ttfamily
	\raggedright
	\setlength{\parfillskip}{0pt}%
}{%
	\par
}
\newcommand*{\urlitem}[2]{%
	\hangindent=1em
	\hangafter=1   
	\begingroup    
	\EdefUnescapeHex\@tmp{#1}%
	\expandafter\url\expandafter{\@tmp}%
	\endgroup
	\urlindex@pfill
	\IfSubStr{#2}{,}{pp}{%
		\IfSubStr{#2}{-}{pp}{p}%
	}.\@\space\ignorespaces
	#2%
	\par
}
\newcommand*{\urlindex@pfill}{% from \pfill of package `doc'
	\unskip~\urlindex@dotfill
	\penalty500\strut\nobreak
	\urlindex@dotfil~\ignorespaces
}
\newcommand*{\urlindex@dotfill}{% from \dotfill of package `doc'
	\leaders\hbox to.6em{\hss .\hss}\hskip\z@ plus  1fill\relax
}
\newcommand*{\urlindex@dotfil}{% from \dotfil of package `doc'
	\leaders\hbox to.6em{\hss .\hss}\hfil
}
\makeatother

% --------------------------------------

\lstset{
	basicstyle=\scriptsize\ttfamily,
	commentstyle=\ttfamily\color{gray},
	numbers=left,
	numberstyle=\ttfamily\color{gray}\footnotesize,
	stepnumber=1,
	numbersep=5pt,
	backgroundcolor=\color{white},
	showspaces=false,
	showstringspaces=false,
	showtabs=false,
	frame=single,
	tabsize=2,
	captionpos=b,
	breaklines=true,
	breakatwhitespace=false,
	title=\lstname,
	escapeinside={},
	keywordstyle={},
	morekeywords={}
}

\IfFileExists{lipsum.sty}%
{\usepackage{lipsum}\setlipsumdefault{11-13}}%
{\newcommand{\lipsum}[1][11-13]{\par And some text: lipsum ##1.\par}}

\newcommand{\parbreak}{\vspace{5mm}}

\DeclareDocumentCommand{\transitionentry}{o m m}{
	\IfNoValueTF{#1}{
		#3 \xleftarrow[#2]{}}{
		#3 \xleftarrow[#2]{} #1}}

%combined fragments
\newcommand{\fragname}[1]{\textit{#1}}
\newcommand{\fragmessage}[1]{\textit{#1}}
\newcommand{\fragcond}[1]{#1?}
\newcommand{\fragcondt}[1]{``#1?''}

%list of online resources
\newcommand{\onlineitem}[2]{#1: \\ \url{#2}}

%\DeclareDocumentCommand{\transitionentry}{o m m}{
%	\IfNoValueTF{#1}{
%		\{#3 \rightarrow ``#2''\}}{
%		#3 \rightarrow \{#1 \rightarrow ``#2''\}}}

\SetKwRepeat{Do}{do}{while}

%\captionsetup{compatibility=false}

\newtheorem{definition}{Definitie}

\begin{document}
	\begin{preface}
		Ik gebruik deze gelegenheid met plezier om die personen die ik oprechte dankbaarheid ben verschuldigd te eren.
		
		Allereerst dank ik mijn gezin voor hun werkelijk bodemloos geduld en hun steun waar ik altijd op kon rekenen, zij het stil of in de vorm van aanmoedigingen en advies. In het bijzonder dank in mijn ouders die altijd perfect wisten hoe te reageren tijdens momenten waar het niet zo gemakkelijk ging.
		
		Ik dank ook Tante Els voor de interesse die ze toonde in mijn vooruitgang en voor de verscheidene keren dat ze duidelijk maakte dat ze aan me dacht. Weten dat je te allen tijde ondersteund wordt door je familie is een groot comfort in je leven.
		
		Ik dank mijn vrienden omdat ze altijd klaarstonden voor een gesprek als ik dat nodig had en steeds een luisterend oor boden als het er even uit moest dat het minder goed ging. Jullie zijn stuk voor stuk goede gasten.
		
		\textit{Last but not least} dank ik mijn begeleider, Matthias van der Hallen. Hij vervulde die rol altijd met bekwaamheid en empathie, zelfs al duurde het wat langer om dit punt te bereiken.
		
		Deze masterproef is opgedragen aan u allen.
	\end{preface}
	
	\tableofcontents*
	\listoffigures
	\listoftables
	
	\begin{abstract}
		UML is een grafische taal die een aantal soorten diagrammen aanbiedt om een mogelijk ontwerp voor een gewenst softwaresysteem op te stellen. Als de ontwerper echter een groot aantal diagrammen opstelt en de diagrammen een zekere omvang bereiken, is het makkelijk om het overzicht te verliezen. Hierdoor worden er sneller fouten tegen de gewenste specificaties voor het softwaresysteem gemaakt en worden die fouten minder makkelijk opgemerkt. We bieden een gereedschap aan om automatisch klassediagrammen en sequentiediagrammen te verificeren. We beschrijven een methode om een klassediagram voor te stellen in FO($\cdot$). Met deze voorstellingen kunnen we consistentie van een klassediagram aantonen en bepaalde soorten inconsistentie bewijzen. We kunnen ook de aanwezigheid van een aantal soorten kwaliteitsgebreken die een diagram minder verstaanbaar maken aantonen. Deze voorstellingsmethode voor klassediagrammen breiden we uit zodat we bijhorende sequentiediagrammen kunnen voorstellen in de lineaire tijdscalculus. Op deze manier kunnen we sequentiediagrammen simuleren, de uitvoer van een sequentiediagram verificeren en nagaan of gewenste eigenschappen van het systeem als geheel gelden doorheen de uitvoering ervan. Dit gaat echter gepaard met een lange rekentijd en hoog geheugengebruik. De laatste bijdrage is daarom een aanzet om logica te integreren in de ontwerptaal beschikbaar voor sequentiediagrammen. We tonen dat de uitgebreide ontwerptaal leidt tot een significant compactere modellering van een softwaresysteem. Dit leidt tot een significant kortere rekentijd en kleiner geheugengebruik bij inferentie op de bekomen voorstelling van het systeem.
	\end{abstract}
	
	\mainmatter
	
	\chapter{Inleiding}
Binnen software engineering is UML een veelgebruikt gereedschap om het domein waarin de software die ontworpen wordt alsook de structuur van de software zelf grafisch weer te geven. Het is voor de ontwerper interessant om uit een tekstuele beschrijving van wat de voorgestelde software moet kunnen de relevante concepten en procedures te halen, die neer te zetten in een diagram en door middel van de verscheidene symbolen aangeboden door UML uit te drukken hoe die concepten en procedures met elkaar interageren. Op deze manier kan een team snel duidelijkheid scheppen in welke doelen ze precies moeten bereiken.

Het is echter makkelijk om het overzicht te verliezen als de gebruikte diagrammen omvangrijk worden. Dit kan een probleem zijn omdat fouten die worden gemaakt in de ontwerpfase en pas laat in het productieproces ontdekt worden kostbaar zijn om recht te zetten. Het komt ook voor dat een ontwerper per vergissing overbodige informatie toevoegt aan een diagram en dat daardoor het diagram minder duidelijk wordt.

In deze masterproef worden in het bijzonder UML-klassediagrammen beschouwd. Een klassediagram beschrijft welke concepten (in deze tekst verder \textit{klasses} genoemd) er bestaan binnen de software. Elk van die klasses kan attributen en operaties hebben. Verder geeft een klassediagram ook weer welke klasses in relatie staan tot elkaar. Deze relaties leggen vast aan welke beperkingen alle mogelijke toestanden van de beschreven software moeten voldoen om beschouwd te worden als correct.

Beschouw volgend klassediagram:

\begin{figure}[H]
	\label{fig:cd}
	\centering
	\includegraphics{intro/cd.png}
	\caption{Een voorbeeld van een klassediagram}
\end{figure}

Dit klassediagram drukt uit dat er twee klasses bestaan: \textit{C} en \textit{D}. \textit{C} heeft \'e\'en attribuut, \textit{name}, dat van type \textit{string} is. Het heeft ook \'e\'en operatie \textit{capitalize} dat \textit{input}, van type \textit{string}, als parameter heeft. \textit{capitalize} geeft een resultaat terug dat ook van type \textit{string} is. Voorts drukt de lijn tussen \textit{C} en \textit{D} uit dat er een relatie bestaat tussen de twee klasses. Beschouw klasse C. Als we vanuit die klasse de lijn volgen, zien we dat er aan het ander uiteinde staat dat elke \textit{C}-object in relatie moet staan tot exact twee \textit{D}-objecten. Zo ook zien we dat, als we vertrekken vanuit \textit{D}, elk \textit{D}-object in relatie moet staan tot exact \'e\'en \textit{C}-object.

Met het voorgaande in het achterhoofd beschouwen we in deze masterproef twee categorie\"en van gebreken in een klassediagram:

\begin{itemize}
	\item \textbf{Inconsistenties:} Het klassediagram is zo opgebouwd dat geen enkele mogelijke toestand van de software kan beantwoorden aan de voorwaarden die worden opgelegd. Dit betekent dat het stuk van de software dat wordt beschreven in het diagram onmogelijk kan werken.
	\item \textbf{Kwaliteitsgebreken:} Deze gebreken hebben een negatieve impact op de kwaliteit van het softwareontwerp. Zo kunnen ze bijvoorbeeld onduidelijkheden in het ontwerp introduceren of het onderhoud van de software \'e\'enmaal ingezet in productie bemoeilijken.
\end{itemize}

Deze masterproef heeft tot doel om automatisch op een gestructureerde manier uit te drukken welke informatie een UML-klassediagram juist bevat. Die informatie willen we op zijn beurt terug gebruiken om inconsistenties en kwaliteitsgebreken te detecteren. Concreter willen we predikatenlogica gebruiken om de informatie neer te schrijven en om aan detectie van gebreken te doen. De volgende hoofdstukken beschrijven hoe we dit exact willen bereiken. \todo{relevante structuur van document beschrijven}
	\chapter{Relevante literatuur}
In de literatuur kan men verscheidene methoden om UML-diagrammen te vertalen naar logica onderscheiden. Het gaat voornamelijk over klassediagrammen, terwijl sommige werken ook sequentiediagrammen en toestandsdiagrammen beschouwen. De logica waarnaar vertaald wordt is vaak eerste-orde-predikatenlogica, maar een aantal onderzoekers verkiezen talen die behoren tot de klasse van de description logics.

In deze sectie bespreken we een aantal van deze werken.

\section{Redeneren op UML-klassediagrammen gebruikmakend van description logics}
In \textit{Reasoning on class diagrams} \cite{BerardiDaniela2005RoUc} toont men eerst een methode om een klassediagram automatisch te vertalen naar eerste-orde logica.
	\chapter{Een semantiek voor UML-klassediagrammen en controle op consistentie}\label{sec:consistentie}
In dit hoofdstuk geven we een beschrijving van de verscheidene componenten in een klassediagram die we beschouwen in deze masterproef. Die beschrijving gebruiken we om een modelsemantiek te defini\"eren voor klassediagrammen alsook consistentie en inconsistentie van een klassediagram. We beschrijven ook een methode om een voorstellingswijze voor klassediagrammen in FO($\cdot$) te bekomen die overeenkomt met die semantiek.

\section{Componenten van een klassediagram}\label{sec:cd-components}

We gebruiken figuren \ref{fig:voorbeeld1} en \ref{fig:voorbeeld2} ter illustratie van de componenten in een klassediagram, die we beschrijven in de volgende subsecties.

\begin{figure}[h]
	\centering
	\includegraphics[width=0.6\textwidth]{chap-consistentie/voorbeeld1.png}
	\caption{Klasses, associaties en generalisaties}
	\label{fig:voorbeeld1}
\end{figure}

\begin{figure}[h]
	\centering
	\includegraphics[width=0.6\textwidth]{chap-consistentie/voorbeeld2.png}
	\caption{Meervoudige associaties}
	\label{fig:voorbeeld2}
\end{figure}

\subsection{Klasse}

Klassediagrammen beschrijven de mogelijke structuur van de toestand van een object in het gemodelleerde probleemdomein en welke bewerkingen er beschikbaar zijn op een object. \textbf{Klasses} bepalen hoe die structuur eruit ziet voor een object en benoemt de bewerkingen. We zeggen dat een object een \textbf{instantie} is van een klasse als de toestand van en bewerkingen beschikbaar op dat object overeenkomen met die klasse. Dit concept defini\"eren we preciezer later in deze sectie.

Een klasse bestaat uit de volgende elementen:

\begin{itemize}
	\item Elke klasse heeft een unieke \textbf{naam} binnen eenzelfde klassediagram.
	\item \textbf{Attribuut}: De toestand van een klasse bestaat uit de combinatie van zijn attributen. Elk attribuut heeft een naam en een type. De type van een attribuut kan zowel een primitief type zijn zoals \textit{int} of \textit{string} of een andere klasse uit het diagram, al is dit tweede niet gebruikelijk. Klasse \textit{A} in figuur \ref{fig:voorbeeld1} heeft \textit{name} als attribuut met als type \textit{string}. Een attribuut kan ook een multipliciteit hebben, genoteerd als $n..m$ waarvoor $n,m \in \mathbb{N}$ en $n \leq m$. Dit betekent dat een attribuut minimaal $n$ verschillende waarden heeft en maximaal $m$ verschillende waarden. In de plaats van een bovengrens kan er ook $*$ staan, wat betekent dat een attribuut een arbitrair aantal aan verschillende waarden kan hebben. Indien de ondergrens 0 is, kan het zijn dat er geen waarde bestaat voor die attribuut.
	\item \textbf{Operatie}: De operaties van een klasse duiden de beschikbare bewerkingen op een klasse aan. De definitie van een operatie bestaat uit een naam, mogelijks een lijst van argumenten samen met hun types, en het type van het resultaat. Het resultaattype is ofwel een primitief type, een andere klasse uit het diagram, of \textit{void}, wat eigenlijk betekent dat de operatie geen resultaat heeft. Klasse \textit{C} heeft een operatie met als naam \textit{getB}, \'e\'en argument \textit{index} met type \textit{int}, en met als resultaattype de klasse \textit{B}.
\end{itemize}

\subsection{Associatie}

Een associatie relateert een object \'e\'en of meerdere andere objecten van ofwel dezelfde klasse ofwel een andere klasse. Een associatie is ofwel binair ofwel meervoudig. We bespreken deze twee apart.

\subsubsection{Binaire associatie}

Een binaire associatie relateert een object van een klasse aan objecten van ofwel dezelfde klasse ofwel een andere klasse. In figuur \ref{fig:voorbeeld1} relateert de associatie \textit{B}---\textit{C} objecten van klasse \textit{B} aan objecten van klasse \textit{C}, terwijl de associatie \textit{A}---\textit{A} objecten van klasse A relateert aan andere objecten van klasse \textit{A}. Er wordt niet uitgesloten dat een object van klasse \textit{A} gerelateerd wordt aan zichzelf.

Elk uiteinde van een associatie is onderhevig aan een multipliciteit, genoteerd als $n..m$ met $n,m \in \mathbb{N}$ en $n \leq m$ of $n..*$ met $n \mathbb{N}$. Beschouw de associatie \textit{B}---\textit{C} in figuur \ref{fig:voorbeeld1}. Een object van klasse \textit{B} moet gerelateerd zijn aan minimaal \'e\'en object van klasse \textit{C}. Er is geen bovengrens op de hoeveelheid objecten van klasse \textit{C} waaraan een object van klasse \textit{B} gerelateerd is. Een object van klasse \textit{C} moet gerelateerd zijn aan minimaal \'e\'en object van klasse \textit{B} en maximaal drie objecten van klasse \textit{B}.

De multipliciteiten op de associatie \textit{A}---\textit{A} impliceren dat elk object van klasse \textit{A} moet gerelateerd zijn aan exact \'e\'en object van klasse \textit{A}, zij het zichzelf of een ander object.

Het is toegelaten om tussen twee klasses meerdere associatierelaties te defini\"eren. Deze worden beschouwd als aparte associaties en kunnen hun eigen multipliciteiten defini\"eren. Voor eenvoud van implementatie beschouwen we echter in deze masterproef enkel klassediagrammen die tussen twee klasses hoogstens \'e\'en associatierelatie defini\"eren.

In deze masterproef beschouwen we alleen bidirectionele associaties. Dit betekent dat beide uiteindes van een associatie \textbf{navigeerbaar} zijn. We stellen dat een associatie \textit{X}---\textit{Y} impliciet voor klasse \textit{X} de operatie \textit{getY()} definieert dat het gerelateerde \textit{Y}-object als resultaat heeft indien de multipliciteit aan de \textit{Y}-kant ofwel $0..1$ is of $1..1$. Voor $0..1$ kan het natuurlijk het geval zijn dat zulk een object niet bestaat. Indien de multipliciteit aan de \textit{Y}-kant in de plaats van de vorm $0..*$, $0..n$, $n..*$ of $n..m$ is, waarbij $n, m > 0$ en $n \leq m$, definieert men \textit{getAllY()} dat de verzameling van alle gerelateerde \textit{Y}-objecten als resultaat heeft. Als de multipliciteit van de vorm $0..*$ of $0..n$  is, kan deze verzameling uiteraard leeg zijn.

Voor reflexieve associaties kan men rolnamen gebruiken om de uiteindes te desambigueren, maar we beschouwen rolnamen hier niet verder. In de plaats kan men stellen dat de associatie \textit{A}---\textit{A} in figuur \ref{fig:voorbeeld1} impliciet de operaties \textit{getA1()} en \textit{getA2()} definieert.

\subsubsection{Meervoudige associatie}

Er kunnen associaties bestaan tussen meer dan twee klasses tegelijk. Het diagram in figuur \ref{fig:voorbeeld2} bevat een voorbeeld van een ternaire associatie die objecten van de klasses \textit{D}, \textit{E} en \textit{F} aan elkaar relateert. De specificatie voor UML-klassediagrammen geeft geen eenduidige semantiek voor meervoudige associaties. We illustreren de semantiek die in deze masterproef wordt gebruikt aan de hand van figuur \ref{fig:voorbeeld2}:

\begin{itemize}
	\item Voor elk tupel $(d,e)$ waar $d \in D, e \in E$ geldt dat minimaal $l$ objecten van klasse \textit{F} gerelateerd zijn aan het tupel en maximaal $k$ objecten van klasse \textit{F}.
	\item Voor elk tupel $(e,f)$ waar $e \in E, f \in F$ geldt dat minimaal $n$ objecten van klasse \textit{D} gerelateerd zijn aan het tupel en maximaal $m$ objecten van klasse \textit{D}.
	\item Voor elk tupel $(f,d)$ waar $f \in F, d \in D$ geldt dat minimaal $i$ objecten van klasse \textit{E} gerelateerd zijn aan het tupel en maximaal $j$ objecten van klasse \textit{D}.
\end{itemize}

Indien de ondergrens voor de multipliciteit van een uiteinde $0$ is, is het mogelijk dat een tupel geen gerelateerd object heeft. Indien de bovengrens $*$ is, is er een arbitrair aantal aan gerelateerde objecten voor een tupel.

De semantiek voor ternaire associaties opgesteld aan de hand van figuur \ref{fig:voorbeeld2} is eenvoudig te veralgemenen voor meervoudige associaties van arbitraire ariteit.

We stellen dat de associatie \textit{D}---\textit{E}---\textit{F} voor de klasse \textit{D} impliciet de operaties \textit{getAllE(F)} en \textit{getAllF(E)} definieert. Met \textit{getAllE(F)} kan men gegeven een \textit{F}-object de verzameling van alle gerelateerde \textit{E}-objecten opvragen voor het resulterende \textit{D}---\textit{F} tupel. Als de ondergrens aan de \textit{E}-kant 0 is, kan het zijn dat die verzameling leeg is. Als de bovengrens aan de \textit{E}-kant 1 is, wordt de operatie in de plaats \textit{getE(F)}, dat het gerelateerde object teruggeeft voor het resulterende tupel. Indien de multipliciteit $0..1$ is, is het mogelijk dat zulk een object niet bestaat voor het tupel. Als de bovengrens aan de \textit{E}-kant * is, kan de resulterende verzameling een arbitraire grootte hebben. \textit{getAllF(E)} defini\"eren we analoog. Er bestaan gelijkaardige operaties voor de klasses \textit{E} en \textit{F}.

In deze sectie treden we meer in detail over hoe we klassediagrammen voorstellen in FO($\cdot$). Daarvoor willen we een specifieke vorm van logische theorie automatisch laten genereren. In deze theorie\"en staan \textbf{objecten} centraal. Deze objecten zijn instanties van een klasse die voorkomt in het beschouwde diagram, hebben exact de attributen en operaties van die klasse en maken deel uit van exact die relaties die het diagram voorschrijft voor die klasse.

\subsection{Generalizatie}

Klassediagrammen laten toe om directe overervingsrelaties te defini\"eren tussen klasses. Het diagram in figuur \ref{fig:voorbeeld1} stelt dat \textit{A} een directe superklasse is van \textit{B}. Klasses kunnen de directe superklasse zijn van meerdere klasses en klasses kunnen de directe subklasse zijn van meerdere klasses. Dit betekent dat klassediagrammen meervoudige overerving toestaan. Klassehi\"erarchie\"en worden gedefinieerd door de transitieve sluiting over de directe overervingsrelaties die aanwezig zijn in een diagram.

We kunnen nu volledig het concept van \textbf{instantie} defini\"eren. Een object kan een directe instantie zijn van \'e\'en of meerdere klasses, al is dat tweede niet gebruikelijk. Stel dat klasse \textit{Y} een subklasse is van klasse \textit{X} volgens de klassehi\"erarchie\"en gedefinieerd door het diagram. Als een object een directe instantie is van klasse \textit{Y}, is dat object ook een instantie van klasse \textit{X}. Merk op dat als een object een directe instantie is van meer dan \'e\'en klasse, dat het object mogelijks deelneemt aan meer dan \'e\'en klassehi\"erarchie, ook al wordt meervoudige overerving niet toegepast in het diagram.

Subklasses erven alle attributen, operaties en associaties van al hun superklasses. Objecten moeten dus deelnemen aan alle associaties gedefinieerd voor de klasses waar het een directe instantie van is alsook alle associaties gedefinieerd voor alle superklasses waar het object een instantie van is.

De specificatie van klassediagrammen stelt dat in een klassehi\"erarchie een attribuut met een bepaalde signatuur, bestaande uit zijn naam, type en multipliciteit, maar \'e\'enmaal mag gedefinieerd worden. Verder mag een operatie met een bepaalde signatuur, zijnde haar naam, lijst van parameters met hun bijhorende type waarbij een subklasse het type van een parameter mag vervangen door een subklasse ervan, en resultaattype, dat een subklasse ook mag vervangen door een subklasse ervan, meermaals gedefinieerd worden in deze klassehi\"erarchie. Doorgaans doet men dit om aan te geven dat een subklasse de implementatie van een bepaalde methode anders invult. Men duidt de herdefinitie van een operatie in een subklasse aan met de term \textit{overriding}. Als een klasse echter deel uitmaakt van meerdere klassehi\"erarchie\"en, mag die klasse maar een definitie van een operatie van een bepaalde signatuur overerven van ten hoogste \'e\'en klassehi\"erarchie. Als een klasse in dat geval de operatie zelf herdefinieert, is het diagram wel geldig.

Voor eenvoud van implementatie houden we in deze masterproef echter geen rekening met herdefinitie van een attribuut, \textit{overriding} of het overerven van een operatie uit meerdere klassehi\"erarchie\"en. Dit betekent dat wij overge\"erfde attributen en attributen met dezelfde signatuur gedefinieerd in de subklasse zelf behandelen als aparte attributen. We behandelen overge\"erfde operaties en operaties met dezelfde signatuur gedefinieerd in een subklasse op een analoge manier. De conflicten genoemd in de vorige paragraaf leiden er dus niet toe dat een klassediagram ongeldig is.

\section{Semantiek voor klassediagrammen}

In deze sectie beschrijven we een modelsemantiek voor klassediagrammen gebaseerd op de componenten genoemd in sectie \ref{sec:cd-components} en de eigenschappen daarvan.

Met de \textbf{instantiatie} van een klasse bedoelen we de verzameling van alle instanties van een klasse, zij het directe instanties of als gevolg van de klassehi\"erarchie\"en in het diagram.

Allereerst is er een triviaal model voor alle klassediagrammen. Dit bestaat uit lege instantiaties voor alle klasses.

Attributen, operaties, associaties en overervingsrelaties hebben allemaal gevolgen voor objecten.

Alle objecten moeten voor alle attributen gedefinieerd in de klasses waar het een directe instantie van is en voor alle overge\"erfde attributen zoveel verschillende waarden hebben als voorgeschreven door de multipliciteit van het attribuut. Als de ondergrens 0 is, kan het zijn dat er geen waarde aanwezig is.

Alle objecten moeten voor alle operaties gedefinieerd in de klasses waar het een directe instantie van is en voor alle overge\"erfde operaties voldoen aan de volgende voorwaarde: Voor alle mogelijke combinaties van argumenten, allemaal van het juiste type, moet er een resultaat van het juiste type zijn. Deze voorwaarde ten gevolge van operaties laten we vallen wanneer we de vertaling van sequentiediagrammen beschouwen in hoofdstuk \ref{sec:gedrag}. Sequentiediagrammen specificeren immers het gedrag van operaties.

Alle objecten moeten voor alle associaties gedefinieerd voor de klasses waar het een directe instantie van is en voor alle overge\"erfde associaties een invulling hebben die voldoet aan de multipliciteit gespecificeerd voor het andere uiteinde.

Stel dat er een associatie \textit{X}---\textit{Y} bestaat en dat object \textit{x} een instantie is van \textit{X}. Zij $n..m$ met $n \leq m$ de multipliciteit aan uiteinde \textit{Y}. Object \textit{x} moet dan in de context van associatie \textit{X}---\textit{Y} gerelateerd zijn aan minstens $n$ objecten van klasse \textit{Y} en maximaal $m$ objecten van klasse \textit{Y}. De minimumvoorwaarde valt weg als $n = 0$ en de maximumvoorwaarde valt weg als $m = *$.

We geven een voorbeeldmodel voor het diagram in figuur \ref{fig:voorbeeld1}:

\begin{itemize}
	\item Klasse \textbf{\textit{A}}: $\{a; b1; b2; b3\}$
	\item Klasse \textbf{\textit{B}}: $\{b1; b2; b3\}$
	\item Klasse \textbf{\textit{C}}: $\{c\}$
	\item Attribuut \textbf{\textit{name}} in \textbf{\textit{A}}: $\{(a,``a''); (b1,``b1''); (b2,``b2''); (b3,``b3'')\}$
	\item Attribuut \textbf{\textit{id}} in \textbf{\textit{B}}: $\{(b1,1);(b2,2);(b3,3)\}$
	\item Attribuut \textbf{\textit{id}} in \textbf{\textit{C}}: $\{(c,1)\}$
	\item Operatie \textbf{\textit{calculateNumber} in \textbf{\textit{A}}: $\{(a,n)\rightarrow{}n\}$} voor alle instanties \textit{a} van \textit{A} en voor alle $n \in \mathbb{N}$
	\item Operatie \textbf{\textit{getNumber}} in \textbf{\textit{B}}: $\{b1\rightarrow{}1; b2\rightarrow{}2; b1\rightarrow{}3\}$
	\item Operatie \textbf{\textit{getText}} in \textbf{\textit{C}}: $\{c\rightarrow{}``foo''\}$
	\item Operatie \textbf{\textit{calcB}} in \textbf{\textit{C}}: $\{(c,n)\rightarrow{}b1\}$ voor alle $n \in \mathbb{N}$
	\item Associatie \textbf{\textit{A}---\textit{A}}: $\{a\leftrightarrow{}a; b1\leftrightarrow{}b1; b2\leftrightarrow{}b2; b3\leftrightarrow{}b3$
	\item Associatie \textbf{\textit{B}---\textit{C}}: $\{b1\rightarrow{}\{c\}; b2\rightarrow{}\{c\}; b3\rightarrow{}\{c\}; c\rightarrow{}\{b1;b2;b3\}\}$
\end{itemize}

\subsection{Consistentie van klassediagrammen}

We nemen de definities voor consistentie over van Daniela Berardi et al.\cite{BerardiDaniela2005RoUc}.

Een klassediagram is consistent als en slechts als er een eindig model bestaat waarvoor minstens \'e\'en klasse minstens \'e\'en instantie bevat. Deze definitie sluit dus het triviale model waarvoor geen enkele klasse een instantie bevat uit.

Een klasse in een klassediagram is consistent als en slechts als er een eindig model bestaat waarvoor de klasse minstens \'e\'en instantie bevat.

Als gevolg van de eerder gedefinieerde semantiek voor de beschikbare componenten voor klassediagrammen in deze masterproef kan een klassediagram of een klasse inconsistent zijn ten gevolge van de aanwezige associaties en overervingsrelaties. We illustreren hoe dit kan gebeuren aan de hand van figuren \ref{fig:incon-assoc-simple}, \ref{fig:incon-assoc-multi} en \ref{fig:incon-gen-assoc}.

\begin{figure}
	%\vspace*{-0.5cm}
	\hspace{-3cm}
	\centering
	\begin{subfigure}[b]{0.3\textwidth}
		\includegraphics[width=0.6\textwidth]{chap-consistentie/voorbeeld3.png}
		\caption{}
		\label{fig:incon-assoc-simple}
	\end{subfigure}%
	\begin{subfigure}[b]{0.3\textwidth}
		\includegraphics[width=2\textwidth]{chap-consistentie/voorbeeld4.png}
		\caption{}
		\label{fig:incon-assoc-multi}
	\end{subfigure}
	\caption{Voorbeelden van oneindige modellen ten gevolge van associaties}
	\label{fig:incon-assoc}
\end{figure}

\begin{figure}
	\centering
	\includegraphics[width=0.175\textwidth]{chap-consistentie/voorbeeld5.png}
	\caption{Voorbeeld van oneindige modellen ten gevolge van een overervingsrelatie en associatie}
	\label{fig:incon-gen-assoc}
\end{figure}

Beschouw het diagram in figuur \ref{fig:incon-assoc-simple}. Stel dat de klasse \textit{U} twee instanties \textit{u1} en \textit{u2} bevat. We proberen de associatie in te vullen als volgt: $\{u1\leftrightarrow{}u1; u1\leftrightarrow{}u2\}$. Om \textit{u2} twee gerelateerde objecten aan de rechterkant te geven, zouden we allereerst $u2\leftrightarrow{}u1$ moeten toevoegen, maar dit zou ingaan tegen de multipliciteit van 1 aan de linkerkant voor \textit{u1}. Voor een gelijkaardige reden kan $u2\leftrightarrow{}u2$ ook niet toegevoegd worden. Zo wordt er afgedwongen dat er twee extra instanties moeten toegevoegd worden: \textit{u3} en \textit{u4}. Zo kunnen we $u2\leftrightarrow{}u3$ en $u2\leftrightarrow{}u4$ toevoegen. Wanneer we \textit{u3} en \textit{u4} allebei twee gerelateerde objecten aan de rechterkant willen geven, doet zich echter een gelijkaardig probleem als voor \textit{u2} voor waardoor de toevoeging van \textit{u3} en \textit{u4} nodig was in de eerste plaats. Dit zorgt ervoor dat een geldig niet-leeg model voor dit diagram oneindig groot zou moeten zijn en dat de klasse \textit{U} inconsistent is.

Beschouw het diagram in figuur \ref{fig:incon-assoc-multi}. De associaties \textit{J}---\textit{K} en \textit{K}---\textit{L} hebben tot gevolg dat \textit{J}, \textit{K} en \textit{L} exact even veel instanties moeten bevatten. De associatie \textit{J}---\textit{L} stelt echter dat alle instanties van klasse \textit{L} in verband moeten staan met twee instanties van klasse \textit{J}. Dit zorgt ook voor oneindig grote niet-lege modellen en dus inconsistentie van het diagram.

Beschouw het diagram in figuur \ref{fig:incon-gen-assoc}. Stel dat klasse \textit{V} als directe instantie \textit{v} heeft en klasse \textit{W} als directe instanties \textit{w1} en \textit{w2}. We stellen $\{v\rightarrow{}w1; v\rightarrow{}w2; w1\rightarrow{}v; w2\rightarrow{}v\}$. \textit{w1} en \textit{w2} zijn echter ook instanties van klasse \textit{V}, waardoor er vier nieuwe objecten nodig zijn om de associatie voor beide objecten te vervullen. Het probleem zet zich voor de vier objecten verder. Dit zorgt weer voor oneindig grote niet-lege modellen en inconsistentie van het diagram.

\section{Voorstellingswijze van klassediagrammen in FO($\cdot$) volgens de voorgestelde semantiek}

Het merendeel van het werk in deze sectie bestaat erin om de methode om klassediagrammen te vertalen naar eerste-orde-predicatenlogica ge\"introduceerd in Daniela Berardi et al.\cite{BerardiDaniela2005RoUc} aan te passen om gebruik te maken van logische types zoals gedefinieerd in FO($\cdot$).

Aan de hand van volgend voorbeeld zullen we illustreren welke regels we gebruiken om een theorie op te bouwen:

\begin{figure}[h]
	\includegraphics[width=0.95\textwidth]{chap-consistentie/diagram-voorbeeld.png}
	\caption{Leidend voorbeeld van een klassediagram}
	\label{fig:diagram-voorbeeld}
\end{figure}

Meer bepaald willen we uitdrukken welke \textbf{klasses} er bestaan in het diagram waarvan een object een instantie kan zijn, welke \textbf{attributen} en \textbf{operaties} elke klasse bevat, welke \textbf{associaties} er bestaan tussen de verscheidene klasses en welke \textbf{klassehi\"erarchie\"en} er bestaan.

\subsection{Logische types voor objecten}
We moeten een manier hebben om objecten te benoemen in een theorie en te specificeren tot welke klasse die objecten behoren. Voor beide van deze noden zijn logische types geschikt. We voegen voor elke klasse een logisch type toe aan het vocabularium. Er is een logisch type \textit{Character}, een logisch type \textit{Position}, etc.

\subsection{Voorstellen van attributen}
Voor elk attribuut voegen we een binair predicaat toe waarvan de naam beantwoordt aan het patroon: \textit{Klassenaamattribuutnaam}. Voor klasse \textit{Character} en attribuut \textit{name} resulteert dit dus in het predicaat \textit{Charactername/2}. Het eerste argument van dit predicaat is een \textit{Character}. Het type van het tweede argument hangt af van wat er in het diagram staat: Als het een primitief type is zoals \textit{string} of \textit{int}, zal dat ook het type zijn van het tweede predicaat; in het andere geval is het type van het tweede argument het overeenkomstig logisch type voor die klasse. Op deze manier wordt afgedwongen dat elk argument van het predicaat van het juiste type is.
De signatuur van \textit{Charactername/2} is daarom \textit{Charactername(Character, string)}.
Voor elk attribuut wordt ook een regel omtrent multipliciteit afgeleid. Zij \textit{lowerBound} de ondergrens en \textit{upperBound} de bovengrens. Dan is de meest algemene vorm van deze regel als volgt:
	
\begin{align*}
	\forall{o1}[KlasseType](lowerBound \geq \#\{o2 [AttribuutType] : \\ Klassenaamattribuutnaam(o1,o2)\} \geq upperBound).
\end{align*}
	
waarbij \textit{lowerBound} wordt weggelaten als deze $0$ is en \textit{upperBound} wordt weggelaten als deze $*$ is. Indien beide van deze voorwaarden gelden, wordt er geen regel afgeleid betreffende de multipliciteit van het attribuut. Als $lowerBound = upperBound$, wordt deze regel in de plaats:
	
\begin{align*}
	\forall{o1}[KlasseType] \exists_{=upperBound}{o2}[AttribuutType](Klassenaamattribuutnaam(o1,o2)).
\end{align*}
	
Voor \textit{Charactername/2} wordt daarom afgeleid:
	
\begin{align*}
	\forall{o1}[Character]\exists!{x}[string](Charactername(o,x)).
\end{align*}

\subsection{Voorstellen van operaties}
Voor elke operatie voegen we een functie toe dat beantwoordt aan volgend patroon: \sloppy
	 \textit{Klassenaamoperatienaam/(m+1) : ResultaatType}, waarbij $m$ het aantal argumenten dat als invoer wordt meegegeven aan de operatie en \textit{ResultaatType} het type van het resultaat, zij het een primitief type of een klasse. De signatuur ziet eruit als \textit{$Klassenaamoperatienaam(o,p_1,\ldots,p_m)$}, waarbij \textit{o} het object van het logisch type overeenkomstig de klasse waarvan de operatie deel is en \textit{$p_1$} \ldots \textit{$p_m$} de argumenten. Indien er geen argumenten zijn, ziet de signatuur eruit als \textit{Klassenaamoperatienaam(o)}. Voor \textit{determineDamageWeaponFrom(Weapon)} van \textit{Character} wordt dit dus \textit{CharacterdetermineDamageFrom(Character,Weapon) : int}.

Voor elke combinatie van object waarop de operatie wordt opgeroepen en mogelijke invoerparameters moet het het geval zijn dat er \'e\'en enkel resultaat is. Dit is triviaal het geval aangezien een functie elk element uit het domein afbeeldt op exact \'e\'en element uit het codomein.

\subsection{Voorstellen van associaties}
Voor elke associatie voegen we een predicaat toe dat beantwoordt aan volgend patroon: \textit{$ClassOneand\ldots{}andClassM/m$}, waarbij \textit{m} de ariteit van de associatie. Voor de associatie tussen \textit{Inventory} en \textit{Item} wordt dit dus \textit{InventoryandItem(Inventory,Item)}.

De multipliciteit voor elke rol moet worden uitgedrukt. Voor alle $o_l$ waarvoor $1 \leq l \leq m$ wordt een regel toegevoegd van de volgende vorm:\\

Zij $lowerBound_l$ de ondergrens en $upperBound_l$ de bovengrens:
\begin{align*}
	&\forall{c_1}[Klasse_1]\ldots\forall{c_m}[Klasse_m](lowerBound_l \leq
	\\
	&\#\{o_l[Klasse_l] : ClassOneand\ldots{}andClassM(c_1,\ldots,o_l,\ldots,c_m)\} \leq upperBound_l).
\end{align*}
	
waarbij de \textit{c} met index \textit{l} overgeslagen wordt. Indien de ondergrens gelijk is aan $0$ of de bovengrens gelijk is aan $*$ worden deze weggelaten. Als beide voorwaarden gelden, wordt voor deze \textit{l} geen regel afgeleid. Indien $lowerBound_l = upperBound_l$ wordt in de plaats afgeleid:
	
	\begin{align*}
	&\forall{c_1}[Klasse_1]\ldots\forall{c_m}[Klasse_m] \exists_{=upperbound_l}o_l[Klasse_l](ClassOneand\ldots{}andClassM\\&(c_1,\ldots,o_l,\ldots,c_m)).
	\end{align*}
	
	Voor \textit{InventoryandItem/2} worden de volgende regels afgeleid:
	
\begin{align*}
		\forall{o_2}[Item](\#{o_1[Inventory]: InventoryandItem(o_1,o_2)} \leq 1).
\end{align*} 
		
\begin{align*}
		\forall{o_1}[Inventory](\#{o_2[Item]: InventoryandItem(o_1,o_2)} \leq 5).
\end{align*}

\subsection{Voorstellen van klassehi\"erarchi\"een}\label{sec:hierarchies}
Aangezien FO($\cdot$) een getypeerde logica is, is het eenvoudig om klassehi\"erarchie\"een voor te stellen. Als volgens het diagram klasse \textit{X} een subklasse is van klasse \textit{Y}, dan is het voldoende om in het uitvoervocabularium te noteren dat logisch type \textit{Y} een supertype is van logisch type \textit{X}. Het is mogelijk om op die manier een keten van overervingsrelaties te modelleren en zo een hi\"erarchie te verkrijgen. In IDP in het bijzonder is het mogelijk dat een logisch type een subtype is van meer dan \'e\'en type, dus kan men ook meervoudige overerving modelleren.

Dit heeft echter een gevolg voor interpretaties van het logisch type overeenkomstig met een superklasse in een mogelijke structuur volgens het uitvoervocabularium. Indien klasse \textit{Z} een subklasse is van klasse \textit{X} uit de vorige paragraaf, dan moeten alle objecten die behoren tot klasse \textit{Z} ook behoren tot klasse \textit{X}, en zo ook moeten alle objecten die behoren tot klasses \textit{Z} en \textit{X} behoren tot klasse \textit{Y}.

Het is mogelijk om de eindgebruiker dit werk te besparen en automatisch de geschikte interpretaties te laten berekenen door af te stappen van een logisch type voor elke klasse en in de plaats twee nieuwe logische types te introduceren: Een algemeen logisch type voor objecten, \textit{Object}; en een \textit{constructed type}\cite{DeCatBroes2014PLaa} dat voor elke klasse in het diagram een overeenkomstig object heeft, \textit{ClassObject}.

Om lidmaatschap van een klasse uit te drukken, zijn er verder twee nieuwe predicaten nodig: \textit{RuntimeClass(ClassObject, Object)}, wat voor elk object uitdrukt tot welke klasse precies het behoort; en \textit{StaticClass(ClassObject, Object)}, wat uitdrukt tot welke klasses een object behoort als gevolg van de klassehi\"erarchie\"en in het diagram.

Er zijn ook twee predicaten nodig om overervingsrelaties tussen klasses te modelleren: \textit{IsDirectSupertypeOf(ClassObject, ClassObject)} om directe overervingsrelaties voor te stellen; en \textit{IsSupertypeOf(ClassObject, ClassObject)} dat de transitieve sluiting voor \textit{IsDirectSupertypeOf} voorstelt. Aangezien het in predicatenlogica onmogelijk is om een algemene voorstelling voor transitieve sluitingen uit te drukken, maken we gebruik van inductieve definities\cite{DeCatBroes2014PLaa}:

\begin{align}
\{
\nonumber &\forall{x}[ClassObject]\forall{y}[ClassObject](\mathit{IsSupertypeOf}(x,y) \leftarrow \\ &\mathit{IsDirectSupertypeOf}(x,y)).\label{def:tc1} \\
\nonumber &\forall{x}[ClassObject]\forall{y}[ClassObject](\mathit{IsSupertypeOf}(y,x) \leftarrow \\
&\exists{z}(\mathit{IsSupertypeOf(y,z)} \land \mathit{IsSupertypeOf}(z,x))).\label{def:tc2}
\}
\end{align}

\sloppy Zin \ref{def:tc1} maakt gebruik van \textit{IsDirectSupertypeOf/2} om een basisgeval voor \\ \textit{IsSupertypeOf/2} op te stellen. Zin \ref{def:tc2} bouwt dan verder de hi\"erarchie\"en op.

We maken in een tweede definitie gebruik van \textit{IsSupertypeOf/2} om \textit{StaticClass/2} in te vullen:

\begin{align*}
\{
&\forall{x}[ClassObject]\forall{o}[Object](StaticClass(x,o) \leftarrow RuntimeClass(x,o)). \\
&\forall{x}[ClassObject]\forall{y}[ClassObject]\forall{o}[Object](StaticClass(y,o) \leftarrow \\ &RuntimeClass(x,o) \land \mathit{IsSupertypeOf}(y,x)).
\}
\end{align*}

In een aparte definitie lijsten we de invulling voor \textit{IsDirectSupertype/2} gebaseerd op het diagram op. Voor het diagram in figuur \ref{fig:diagram-voorbeeld} wordt dit:

\begin{align*}
\{
&\mathit{IsDirectSupertypeOf}(Statistic,Weaponlevel) \leftarrow .\\
&\mathit{IsDirectSupertypeOf}(Statistic,DerivedStatistic) \leftarrow .\\
&\mathit{IsDirectSupertypeOf}(Item,Weapon) \leftarrow .\\
\}
\end{align*}

We hebben echter niet voor deze aanpak gekozen voor twee redenen.
Attribuutpredicaten, operatiepredicaten en associatiepredicaten zouden logisch type \textit{Object} als argumenten moeten hebben. Dit betekent dat er nieuwe regels nodig zijn die afdwingen dat die objecten van het juiste type zijn. Grotere theorie\"en leiden tot een grotere rekentijd en geheugengebruik bij de uitvoering van redeneertaken.
De multipliciteitsregels zouden ook herschreven moeten worden om gebruik te maken van \textit{StaticClass}. Dit maakt zinnen over de hele lijn langer en zorgt ervoor dat ze moeilijker te begrijpen zijn.

\parbreak

In hoofdstuk \ref{sec:rol-idp} wordt de logische theorie die automatisch gegenereerd werd volgens de regels opgelijst in dit hoofdstuk weergegeven en wordt ook uitgelegd hoe die theorie wordt gebruikt om de consistentie van het diagram te controleren.
	\chapter{Simuleren van gedrag op basis van een sequentiediagram}\label{sec:gedrag}
Een ander populair type van UML-diagram is het sequentiediagram\cite{RumbaughJames2005Tuml}. Waar klassediagrammen de informatie bevat in klasses en de verbanden tussen klasses benoemen, beschrijven sequentiediagrammen het gedrag van methodes gedefinieerd voor deze klasses. In deze diagrammen communiceren instanties van klasses via berichten. Doorgaans zijn deze berichten een oproep van een methode, of een toekenning aan een variabele intern aan het diagram of een instantiatie van een nieuwe instantie. De berichten zijn genummerd volgens een bepaalde volgorde en samen modelleren ze het gedrag van een stuk van de software.

In dit hoofdstuk beschouwen we hoe we vocabularia en logische theorie\"en gegenereerd volgens de regels beschreven in sectie \ref{sec:cd-rep-cons} kunnen uitbreiden om het gedrag voorgesteld in een sequentiediagram te modelleren.

\section{Componenten van een sequentiediagram}

In deze sectie beschrijven we welke componenten we beschouwen in deze masterproef.

Figuur \ref{fig:seq-diagram-game} geeft een voorbeeld van een sequentiediagram gebaseerd op het klassediagram voorgesteld in figuur \ref{fig:diagram-voorbeeld}.

Een instantie wordt voorgesteld door een kader met daarin tekst volgens het patroon \textit{instantienaam : klassenaam}. Dit wil zeggen dat bijvoorbeeld \textit{attacker} een instantie is van de klasse \textit{Character}. Vanuit elk kader vertrekt ook een streepjeslijn: De \textbf{levenslijn}. Deze levenslijn kan ingevuld worden door gekleurde balken, welke de duur van een oproep van een methode aan een instantie voorstellen.
Verder zijn er ook kaders die berichten omsluiten. Deze kaders duiden \textbf{gecombineerde fragmenten} aan, en in deze tekst beschouwen we twee soorten:

\begin{enumerate}
	\item Het \textbf{altfragment}: Deze soort duidt een \textit{if-else}-constructie aan. Het bestaat uit twee delen, namelijk het \textit{if}-deel en het \textit{else}-deel, en er staat aangeduid onder welke voorwaarden welk deel wordt uitgevoerd. Figuur \ref{fig:seq-diagram-game} bevat een voorbeeld van een altfragment.
	\item Het \textbf{lusfragment}: Deze soort duidt een lusconstructie aan. Er staat aangeduid onder welke voorwaarden er een iteratie wordt uitgevoerd. Deze voorwaarde wordt gecontroleerd zowel v\'o\'or de eerste keer dat er mogelijks een iteratie wordt uitgevoerd als elke keer dat een iteratie ten einde komt. Indien de voorwaarde niet geldt, wordt de lus overgeslagen. Figuur \ref{fig:seq-diagram-frag-ex} bevat een voorbeeld van een diagram met lusfragmenten.
\end{enumerate}

Berichten worden voorgesteld door pijlhoofden aan ofwel een ononderbroken lijn ofwel een streepjeslijn. Ze kunnen verscheidene betekenissen hebben afhankelijk van de context.

Allereerst zijn er berichten die een variabele defini\"eren en er een waarde aan toekennen. Die waarde kan direct berekend worden, zoals in instructie 12 in figuur \ref{fig:seq-diagram-game}. De waarde kan een directe waarde zijn zoals 3 of ``foo'', een variabele, of een bewerking op directe waarden of variabelen van primitieve types. Als de instructie een bewerking op waarden van een primitief type is, moeten de gebruikte waardes allemaal van hetzelfde type zijn en moet het type een geldig invoertype voor de bewerking zijn, anders is het sequentiediagram ongeldig. De toegekende waarde kan ook het resultaat zijn van een methodeoproep, zoals in instructie 3 in dezelfde figuur. Als de methode die wordt opgeroepen een methode is gedefinieerd voor een klasse door het klassediagram, wordt de waarde eerst berekend in een uitvoering van het bijhorend sequentiediagram voor het wordt toegekend aan de variabele.

Een numerieke variabele kan een waarde krijgen door middel van de functies \textit{chooseEx(lowerBound, upperBound)} en \textit{chooseIn(lowerBound, upperBound)}. Bij uitvoering van die functies geeft de gebruiker een getal $n$ op. Bij \textit{chooseEx} geldt dat $n \geq lowerBound$ en $n < upperBound$ en bij \textit{chooseIn} geldt dat $n \geq lowerBound$ en $n < upperBound$.

Een variabele kan ook gedefinieerd worden door een paar van methodeoproep en terugkeerbericht. Instructies 6 en 7 in figuur \ref{fig:seq-diagram-game} zijn daar een voorbeeld van. Instructie 6 is een methodeoproep. Instructie 7 definieert impliciet de variabele \textit{defenceVal} en kent de waarde berekend in de oproep eraan toe.

Methodeoproepen zijn enkel geldig als de methode die wordt opgeroepen gedefinieerd is voor de klasse waar de ontvanger van het bericht een instantie van is. Als een methode gedefinieerd voor een klasse in het klassediagram wordt opgeroepen, wordt verwacht dat voor elke parameter een geldige waarde van het juiste type wordt opgegeven, hetzij een directe waarde, hetzij een variabele. Indien het verkeerde aantal argumenten wordt opgegeven, is het gedrag van de oproep niet gedefinieerd. Indien een waarde van een fout type wordt opgegeven, is het sequentiediagram niet geldig.

Attributen en associaties defini\"eren voor klasses impliciet enkele methodes. 

We beschouwen enkel attributen met een multipliciteit van $1..1$. Daarvoor defini\"eren we methodes volgens de patronen \textit{getAttribuutnaam()} en \textit{setAttribuutnaam()}. Voor \textit{value} in de klasse \textit{Statistic} in figuur \ref{fig:diagram-voorbeeld} krijgen we \textit{getValue()} en \textit{setValue(int)}. \textit{getValue()} haalt de waarde van \textit{value} op en \textit{setValue(int)} zorgt ervoor dat de waarde van \textit{value} gelijk is aan de gegeven waarde in de volgende tijdstap.

Wat betreft associaties beschouwen we enkel binaire associaties. Ze defini\"eren de volgende soorten van methodes:

\begin{itemize}
	\item De multipliciteit is $1..1$: We defini\"eren een methode volgens het patroom \textit{getKlassenaam()}. De klasse \textit{Character} in figuur \ref{fig:diagram-voorbeeld} krijgt dus de methode \textit{getInventory()} dat de gerelateerde instantie van klasse \textit{Inventory} als resultaat heeft.
	\item De multipliciteit is $0..1$: Zoals hierboven, maar er is mogelijks geen resultaat.
	\item De multipliciteit is van de vorm $0..n$, $0..*$, $1..n$ of $1..*$ waarbij $n \in \mathbb{N}$: We defini\"eren een methode \textit{getKlassenaam(int)} waar het argument de betekenis heeft van een index. Een uiteinde met een multipliciteit met een bovengrens groter dan 1 stellen we immers voor door een verzameling waar ieder lid een volgnummer heeft. Het opgegeven argument moet groter zijn dan 0. Als het argument groter is dan $n$ of als de ondergrens gelijk is aan 0 en de verzameling horend bij het uiteinde is leeg, heeft de oproep geen resultaat. De klasse \textit{Inventory} in figuur \ref{fig:diagram-voorbeeld} krijgt dus de methode \textit{getItem(int)}.
	\item Men kan een instantie aan het andere uiteinde van een associatie opvragen op basis van de waarde van een attribuut. Klasse \textit{Character} krijgt een methode \textit{getStatisticByName(string)}. Het resultaat is de gerelateerde instantie van klasse \textit{Statistic} die het opgegeven argument als waarde heeft van \textit{name}, als die bestaat. Als de opgegeven waarde geen unieke instantie benoemt, is het gedrag van dit soort methode niet gedefinieerd. Instructie 14 in figuur \ref{fig:seq-diagram-game} heeft dus een instantie van klasse \textit{Statistic} gerelateerd aan \textit{target} waarvoor \textit{name} gelijk is aan \textit{``hp''} als resultaat, als die bestaat.
\end{itemize}

%\begin{landscape}
	%\thispagestyle{empty}
	%\resizebox{\textwidth}{!}{
\begin{sidewaysfigure}[htp]
	\includegraphics[height=0.55\textwidth]{chap-gedrag/seq-diagram-game.png}
	\caption{Sequentiediagram gebaseerd op het klassediagram van figuur \ref{fig:diagram-voorbeeld}}
	\label{fig:seq-diagram-game}
\end{sidewaysfigure}
%}
%\end{landscape}

\section{De keuze voor lineaire tijdscalculus}
UML-diagrammen schrijven mogelijke toestanden van softwaresystemen en acties op deze voor. Die systemen kunnen van toestand veranderen tussen tijdstappen. Sequentiediagrammen zijn een manier om te beschrijven hoe zulke veranderingen teweeggebracht kunnen worden. Tijdens de uitvoering van een sequentiediagram mag het systeem enkel veranderen zoals beschreven door de huidige actie. Daarom hebben we een mechanisme nodig binnen FO($\cdot$) dat dynamische systemen en bewerkingen erop kan beschrijven. Tegelijk moet dat mechanisme garanderen dat eigenschappen van het systeem die niet worden be\"invloed door de huidige beschouwde actie van het sequentiediagram niet veranderen. Lineaire tijdscalculus\cite{BogaertsBart2014Sdsu}, oftewel LTC, voldoet aan deze voorwaarden. Daarom zullen we om sequentiediagrammen uitvoerbaar te maken binnen FO($\cdot$) het generatieproces voor het vocabularium en de theorie die we bekomen hebben in hoofdstuk \ref{sec:consistentie} uitbreiden volgens de principes van LTC. Dit betekent dat we de predicaten voor operaties die voorheen gegenereerd werden in hoofdstuk \ref{sec:consistentie} achterwege laten.

In de volgende secties werken we deze uitbreiding uit voor het sequentiediagram in figuur \ref{fig:seq-diagram-game}.

\section{Uitbreiding van het vocabularium}
In LTC is tijd een centraal concept, dus daarom introduceren we allereerst een logisch type $Time \subset \mathbb{N}$. Verder defini\"eren we een parti\"ele functie \textit{Next(Time)} dat voor alle tijdpunten het volgende tijdpunt geeft behalve voor het laatst mogelijke tijdpunt. We defini\"eren ook een constante \textit{Start}, wat het eerst mogelijke tijdpunt aanduidt.

Voor elk tijdpunt is het mogelijk dat er een bepaalde instructie van het sequentiediagram wordt uitgevoerd. We duiden deze instructie aan met zijn volgnummer.
Deze volgnummers gebruiken we als instructieteller, en daarvoor defini\"eren we een logisch type $SDPoint \subset \mathbb{N}$.

Om te garanderen dat de instructievolgorde opgelegd door het sequentiediagram gevolgd wordt, maken we deze instructieteller inertieel en introduceren we deze symbolen:

\begin{itemize}
	\item Het toestandspredicaat: \textit{SDPointAt(Time, SDPoint)}
	\item Het begintoestandspredicaat: \textit{I\_SDPointAt(SDPoint)}
	\item Het causatiepredicaat: \textit{C\_SDPointAt(Time, SDPoint)}
\end{itemize}

We moeten ook de instanties waarop gehandeld wordt in het sequentiediagram kunnen benoemen. Om te garanderen dat de instanties die vernoemd worden altijd verwijzen naar hetzelfde object tenzij een instructie een toekenning doet aan de overeenkomstige variabele, maken we ook de instanties inertieel. Voor \textit{attacker} verkrijgen we dan bijvoorbeeld:

\begin{itemize}
	\item \textit{AttackerT(Time, Character)}
	\item \textit{I\_AttackerT(Character)}
	\item \textit{C\_AttackerT(Time, Character)}
\end{itemize}

Het is ook mogelijk dat in een instructie een variabele intern aan het sequentiediagram wordt gedefinieerd. Zo is er instructie 7 waar een \textit{return}-instructie \textit{defenceVal} definieert en ook instructie 12 die de waarde van \textit{inflicted} definieert als een som van andere variabelen. Deze variabelen willen we ook kunnen benoemen en maken we inertieel. Voor alle zulke variabelen defini\"eren we ook predicaten zoals hierboven voor \textit{attacker}.

We passen ook de predicaten die overeenkomen met klasseattributen aan. Het kan immers zijn dat de waarde van een attribuut wordt aangepast, zoals in instructie 18 die de waarde van \textit{value} van object \textit{hp} van klasse \textit{Statistic} verandert naar 0. Klasseattributen maken we ook inertieel. Voor \textit{value} in \textit{Statistic} krijgen we dan:

\begin{itemize}
	\item \textit{Statisticvalue(Time, Statistic, LimitedInt)}
	\item \textit{I\_Statisticvalue(Statistic, LimitedInt)}
	\item \textit{C\_Statisticvalue(Time, Statistic, LimitedInt)}
	\item En het oncausatiepredicaat: \textit{Cn\_Statisticvalue(Time, Statistic, LimitedInt)}
\end{itemize}

Hier voegen we een oncausatiepredicaat toe omdat het mogelijk is dat een attribuut meer dan \'e\'en waarde heeft op een bepaald tijdstip als de bovengrens voor de multipliciteit groter is dan \'e\'en. Met dit predicaat geven we aan dat bepaalde waardes die voor een bepaalde tijdstap gelden ongedaan moeten worden gemaakt in de volgende tijdstap.

\section{Uitbreiden van de theorie}
Voor elke inerti\"ele eigenschap van het systeem moeten er twee dingen gebeuren: Toestandszinnen opstellen en voorwaardes voor causatiezinnen en oncausatiezinnen specificeren. Het resultaat is een inductieve definitie die de inerti\"ele predicaten definieert en een inductieve definitie die de causatiepredicaten en oncausatiepredicaten definieert.

\subsection{Toestandszinnen opstellen}
Toestandszinnen worden geschreven in termen van begintoestandspredicaten, causatiepredicaten en oncausatiepredicaten. Ze garanderen dat inerti\"ele eigenschappen enkel veranderen wanneer het ook echt de bedoeling is dat ze veranderen.

Als eerste kijken we naar toestandszinnen voor \textit{SDPointAt/2}. \textit{I\_SDPointAt/1} geeft aan welke de eerste instructie is die we willen uitvoeren, en daarom schrijven we een definitie die deze overeenkomst uitdrukt:

\begin{align}
	\forall{s}[SDPoint](SDPointAt(Start, s) \leftarrow I\_SDPointAt(s)).
\end{align}


De volgende definities gebruiken het causatiepredicaat:

\begin{align}
	\forall{t}[Time]\forall{s}[SDPoint](SDPointAt(Next(t), s) \leftarrow C\_SDPointAt(Next(t), s)). \label{eq:sdcauses}
\end{align}
\begin{align}
	\forall{t}[Time]\forall{s}[SDPoint](SDPointAt(Next(t), s) \leftarrow SDPointAt(t, s) \nonumber \\ \land{} \space \lnot{}(\exists{s1}[SDPoint](C\_SDPointAt(Next(t), s1)))). \label{eq:sduncauses}
\end{align}

Zin \ref{eq:sduncauses} zorgt ervoor dat de huidige waarde van \textit{SDPointAt} wordt behouden tenzij er een oorzaak is voor verandering.

We schrijven gelijkaardige definities voor de predicaten die overeenkomen met instanties die vernoemd worden in het sequentiediagram (zoals \textit{attacker}).

\parbreak

Voor klasseattributen verloopt dit ook gelijkaardig, maar we wijken af van het formaat van zin \ref{eq:sduncauses} door als volgt het oncausatiepredicaat te gebruiken:

\begin{align*}
	\forall{t}[Time]\forall{s}[Statistic]\forall{i}[LimitedInt](Statisticvalue(Next(t), s, i) \\ \leftarrow Statisticvalue(t, s, i) \land \lnot Cn\_Statisticvalue(Next(t), s, i)).
\end{align*}

\subsubsection{Voorwaardes voor causatie en oncausatie}
We kijken eerst naar klasseattributen. Een aantal ervan worden niet aangepast, wat we bijvoorbeeld neerschrijven voor \textit{range} in \textit{Weapon} als volgt:

\begin{align*}
	\forall{t}[Time]\forall{w}[Weapon]\forall{i}[LimitedInt](C\_Weaponrange(t, w, i) \leftarrow false).
\end{align*}
\begin{align*}
	\forall{t}[Time]\forall{w}[Weapon]\forall{i}[LimitedInt](Cn\_Weaponrange(t, w, i) \leftarrow false).
\end{align*}

Voor de klasseattributen die wel worden aangepast, kijken we naar de instructies die zulke aanpassingen doorvoeren. Voor \textit{value} in \textit{Statistic} zijn dit instructie 18 en 20. We kijken eerst naar de causatiezin en oncausatiezin die volgen uit instructie 18:

\begin{align*}
	\forall{t}[Time]\forall{s}[Statistic](C\_Statisticvalue(t, s, 0) \leftarrow SDPointAt(t, 18) \land HpT(t, s).
\end{align*}
\begin{align*}
	\forall{t}[Time]\forall{s}[Statistic]\forall{i}[LimitedInt](Cn\_Statisticvalue(Next(t), s, i) \\ \leftarrow SDPointAt(Next(t), 18) \land HpT(t, s) \land Statisticvalue(t, s, i) \land \lnot{}(i = 0).
\end{align*}

Aangezien in instructie 18 de instantie \textit{hp} wordt aangesproken, gebruiken we \textit{HpT/2} om te verzekeren dat de waarde van het juiste logisch object wordt veranderd. \textit{value} kan ook maar \'e\'en waarde tegelijk hebben, en daarom schrijven we een oncausatiezin om te verzekeren dat de vorige waarde wordt gewist.

Kijken we nu naar de definities die voortvloeien uit instructie 20:

\begin{align*}
	&\forall{t}[Time]\forall{s}[Statistic]\forall{i}[LimitedInt](C\_Statisticvalue(t, s, i) \\ &\leftarrow SDPointAt(t, 20) \land HpT(t, s) \land NewHpT(t, i)).
\end{align*}
\begin{align*}
&\forall{t}[Time]\forall{s}[Statistic]\forall{i}[LimitedInt](Cn\_Statisticvalue(Next(t), s, i) \\ &\leftarrow SDPointAt(Next(t), 20) \land HpT(t, s) \land Statisticvalue(t, s, i) \land \\ &\lnot{}NewHpT(Next(t), i)).
\end{align*}

Het verschil hier is dat we \textit{NewHpT} erbij betrekken omdat we de waarde van \textit{hp} veranderen naar de waarde van \textit{newHp} in plaats van het te veranderen naar 0.

\parbreak

Het volgende waar we naar kijken zijn de causatiezinnen voor \textit{SDPointAt/2}. Wat we hier willen uitdrukken is dat normaal gezien tussen instructies de instructieteller telkens met \'e\'en wordt verhoogd, tenzij een grens van een \textit{if-else}-constructie of een lus is bereikt. In dat geval kan het zijn dat de instructieteller verspringt afhankelijk van de voorwaarde die vernoemd wordt voor zulke constructies.

Voor deze sequentiediagram krijgen we:

\begin{align}
	&\forall{t}[Time]\forall{s}[SDPoint](C\_SDPointAt(Next(t), (s+1) \leftarrow SDPointAt(t, s) \nonumber \\ &\land \lnot{}((s = 17) \lor (s = 19))). \label{eq:sdprog} \\
	&\forall{t}[Time](C\_SDPointAt(Next(t), 18) \leftarrow SDPointAt(t, 17) \land \nonumber \\ &(\exists{i}[LimitedInt](NewHpT(t, i) \land i <= 0))). \label{eq:sdif} \\
	&\forall{t}[Time](C\_SDPointAt(Next(t), 20) \leftarrow SDPointAt(t, 17 ) \land \nonumber \\ &(\exists{i}[LimitedInt](NewHpT(t, i) \land i > 0))). \label{eq:sdthen} \\
	&\forall{t}[Time](C\_SDPointAt(Next(t), 21) \leftarrow SDPointAt(t, 19) \lor SDPointAt(t, 20)). \label{eq:sdexit}
\end{align}

Zin \ref{eq:sdprog} verzekert het juiste gedrag van de instructieteller, namelijk dat hij doorgaans met \'e\'en wordt verhoogd tussen tijdstappen. De uitzonderingen worden hier ook opgelijst; in dit geval verspringt de teller wanneer men het begin van de \textit{if-else}-constructie tegenkomt en wanneer het einde van het \textit{if}-deel is bereikt. Zinnen \ref{eq:sdif} en \ref{eq:sdthen} controleren de voorwaarde voor de uitvoering van het \textit{if}- en \textit{else}-deel en selecteren wat correct is. Zin \ref{eq:sdexit} zegt dat zowel het \textit{if}-deel als het \textit{else}-deel uitkomen op de instructie die direct volgt op de \textit{if-else}-constructie.

Voor dit diagram is het eenvoudig om deze zinnen op te stellen aangezien er geen geneste gecombineerde fragmenten aanwezig zijn. We beschrijven de algemene methode om het uitvoeringspad doorheen gecombineerde fragmenten te bepalen in sectie \ref{sec:combined-fragment}.

\parbreak

Als laatste zijn er de causatiezinnen voor de verscheidene variabelen die worden aangemaakt en aangesproken in het sequentiediagram. Een aantal van deze variabelen veranderen niet doorheen de uitvoering van het sequentiediagram en er wordt verondersteld dat deze al bekend zijn v\'o\'or de uitvoering begint. Deze variabelen zijn diegenen die betrokken zijn bij de eerste instructie: \textit{attacker}, de instantie die de eerste oproep ontvangt, en \textit{target} en \textit{weapon}, die als parameter worden opgegeven.

Voor de andere variabelen wordt er een causatiezin toegevoegd voor elke instructie die een waarde toekent aan die variabele. Als voorbeeld bekijken we instructie 3:

\begin{align*}
	&\forall{t}[Time]\forall{d}[Statistic](C\_DefenceT(t, d) \leftarrow SDPointAt(t, 3) \land \\ &\exists{c}[Character](TargetT(t, c) \land CharacterandStatistic(c, d) \\ &\land Statisticname(t, d, "defence"))).
\end{align*}


De zin drukt uit dat de getter wordt opgeroepen op \textit{target} en dat er wordt gevraagd naar een instantie van \textit{Statistic} dat in verband staat met \textit{target} en als naam ``defence'' heeft. Die instantie wordt dan als waarde toegekend aan de variabele \textit{defence}.

Zie bijlage \ref{app:seq-diagram-game} voor het volledig model voor het gedrag van het sequentiediagram in figuur \ref{fig:seq-diagram-game}.


\section{Het uitvoeren van gecombineerde fragmenten}\label{sec:combined-fragment}
Het opstellen van causatiezinnen voor \textit{SDPointAt/2} ten gevolge van gecombineerde fragmenten is niet vanzelfsprekend. In deze sectie beschrijven we onze methode om dit te bewerkstelligen.
Bij de vertaling van sequentiediagrammen houden we in de interne voorstelling binnen onze vertaler van een gecombineerd fragment volgende zaken bij:

\begin{itemize}
	\item Alle gecombineerde fragmenten die kinderen zijn van het fragment.
	\item Het fragment dat de ouder is van het fragment in kwestie, als die bestaat.
	\item Alle berichten die rechtstreeks deel zijn van het fragment. Een bericht is rechtstreeks deel van een fragment als het bericht een deel is van het fragment, maar geen rechtstreeks deel is van een kind of afstammeling van het fragment.
	\item De voorwaarde waaraan voldaan moet zijn om het fragment uit te voeren. 
\end{itemize}

Voor alt-fragmenten maken we het onderscheid tussen kinderen en berichten van het \textit{if}-gedeelte enerzijds en tussen kinderen en berichten van het \textit{else}-gedeelte anderzijds. Ook geldt dat er een voorwaarde is voor de uitvoering van het \textit{if}-gedeelte en voor de uitvoering van het \textit{else}-gedeelte.

In de methode om in de uitvoertheorie het uitvoeringspad doorheen gecombineerde fragmenten correct te vertalen gebruiken we drie procedures: E\'en die bepaalt naar welke berichten wordt gesprongen onder welke voorwaarde bij het binnengaan van een fragment, \'e\'en die bepaalt naar welke berichten wordt gesprongen onder welke voorwaarde bij het buitengaan van een fragment en \'e\'en die de heruitvoering van een lus verzorgt indien de voorwaarde voor de lus nog geldt. We bespreken deze procedures afzonderlijk.

\subsection{Transitie naar een fragment}\label{sec:transition-to}

Het resultaat van deze procedure is een mapping van berichten waarnaar gesprongen kan worden bij het binnengaan van een fragment naar onder welke voorwaarde deze sprong gebeurt. We willen dat deze procedure dit niet enkel doet voor het gegeven fragment, maar ook voor alle kinderen van het fragment. Op die manier worden alle fragmenten verwerkt wanneer de procedure wordt opgeroepen voor alle fragmenten zonder ouders.
We geven een overzicht van de procedure in algoritme \ref{alg:transition-to-frag}.

\parbreak

\begin{algorithm}
	\KwIn{\textit{fragment : gecombineerd fragment}; \textit{aggregateVoorwaarde : string}}
	\KwOut{Een mapping van bericht naar string. De string stelt de voorwaarde voor waaronder naar een bericht wordt gesprongen bij de transitie naar een fragment.}
	$uitvoer \leftarrow \emptyset$; \\
	\textit{gezien $\leftarrow \emptyset$}; \\
	\eIf{eerste bericht is rechtstreeks deel van fragment}{
	$uitvoer \leftarrow uitvoer $\textit{ + \{eerste bericht $\rightarrow$ aggregateVoorwaarde + voorwaarde voor fragment\}}; \\
	\textit{kinderen $\leftarrow$ kinderen van fragment}; \\
	\textit{uitvoer $\leftarrow$ uitvoer $\cup$ vouwLussen(gezien, kinderen, $\epsilon$, \textbf{false}, \textbf{true})}; zie algoritme \ref{alg:wrap-loops} \\
	\ForEach{kind $\in$ kinderen}{
		\If{kind $\notin$ gezien}{
		\textit{uitvoer $\leftarrow$ uitvoer $\cup$ bepaalTransitieIn(kind, ``'')};}}
	 }{
	 \textit{eersteFragment $\leftarrow$ eerste kind van fragment}; \\
	 \eIf{eersteFragment is lusfragment}{
	 	\textit{voorwaarde $\leftarrow$ aggregateVoorwaarde + ``$\land$'' + voorwaarde voor fragment}; \\
	 	\textit{uitvoer $\leftarrow$ uitvoer $\cup$ bepaalTransitieIn(eersteFragment, voorwaarde)}; \\
	 	\textit{voorwaarde $\leftarrow$ voorwaarde + ``$\land \lnot$('' + voorwaarde voor eersteFragment + ``)''}; \\
	 	\textit{kinderen $\leftarrow$ kinderen van fragment}; \\
	 	\textit{uitvoer $\leftarrow$ uitvoer $\cup$ vouwLussen(gezien, kinderen, voorwaarde, \textbf{true}, \textbf{true})};
	 }{
	 \textit{voorwaarde $\leftarrow$ aggregateVoorwaarde + ``$\land$'' + voorwaarde voor fragment}; \\
	 \textit{uitvoer $\leftarrow$ uitvoer $\cup$ bepaalTransitieIn(eersteFragment, voorwaarde)};
 	}
 	\ForEach{kind $\in$ kinderen}{
 		\If{kind $\notin$ gezien}{
 		\textit{uitvoer $\leftarrow$ uitvoer $\cup$ bepaalTransitieIn(kind, ``'')};}
 	}
	}
	\textbf{return} \textit{uitvoer};
	\caption{bepaalTransitieIn}
	\label{alg:transition-to-frag}
\end{algorithm}

\begin{algorithm}
	\KwIn{\textit{gezien : verzameling van gecombineerde fragmenten}; \textit{kinderen : lijst van gecombineerde fragmenten}; \textit{aggregateVoorwaarde : string}; \textit{slaEersteOver : boolean}; \textit{allemaal : boolean}}
	\KwOut{Een mapping van bericht naar string, zoals in algoritme \ref{alg:transition-to-frag}}
	\textit{uitvoer $\leftarrow \emptyset$}; \\
	\ForEach{\textit{kind} $\in$ \textit{kinderen}}{
	\eIf{kind is lusfragment}{
	\eIf{allemaal}{
	\textit{uitvoer $\leftarrow$ uitvoer $\cup$ bepaalTransitieIn(kind, aggregateVoorwaarde)};
	}{
	\textit{uitvoer $\leftarrow$ uitvoer $\cup$ bepaalTransitieInOnvolledig(kind, aggregateVoorwaarde)};
	}
	\textit{aggregateVoorwaarde} $\leftarrow$ \textit{aggregateVoorwaarde + ``$\land{} \lnot$('' + voorwaarde voor kind + ``)''}; \\
	\textit{gezien $\leftarrow$ gezien + kind};
	}{
	\eIf{$\lnot$ allemaal}{
	\textbf{break};}{
	\textit{aggregateVoorwaarde} $\leftarrow \epsilon$;}
	}}
	\textbf{return} \textit{uitvoer};
	\caption{vouwLussen}
	\label{alg:wrap-loops}
\end{algorithm}

In algoritme \ref{alg:wrap-loops} is \textit{bepaalTransitieInOnvolledig} een variant van \textit{bepaalTransitieIn} waar er geen recursieve oproep is voor die kinderen die niet betrokken zijn in het vouwproces voor lusfragmenten.

Merk op dat voor alt-fragmenten algoritme \ref{alg:transition-to-frag} wordt uitgevoerd voor het \textit{if}-gedeelte en het \textit{else}-gedeelte afzonderlijk.

De kern van algoritme \ref{alg:transition-to-frag} is dat in de uitvoertheorie in \'e\'en stap bepaald moet worden welk bericht eerst zal worden uitgevoerd wanneer het uitvoerpad een gecombineerd fragment binnengaat. Daarom bouwen we doorheen de recursie een aggregate voorwaarde op die bestaat uit een conjunctie van voorwaarden voor fragmenten. Enkel wanneer een fragment bereikt wordt waarvoor geldt dat het eerste bericht rechtstreeks deel is van dat fragment wordt er een mapping van dat bericht naar die aggregate voorwaarde toegevoegd. Hierna maken we de aggregate voorwaarde terug leeg om dan het proces voort te zetten voor de kinderen.

Algoritme \ref{alg:wrap-loops} is nodig omdat lusfragmenten die elkaar opvolgen een speciaal geval vormen. Voor een lusfragment geldt immers dat het wordt uitgevoerd indien de voorwaarde voor dat lusfragment vervuld is en de voorwaarden voor alle voorgaande lusfragmenten niet vervuld zijn. Figuur \ref{fig:seq-diagram-frag-ex} stelt bijvoorbeeld dat de uitvoering springt naar de lus aangeduid met \fragname{loop4} als aan de voorwaarde voor de lus aangeduid met \fragname{loop3} niet voldaan is. Algoritme \ref{alg:wrap-loops} markeert alle fragmenten die het voor zijn rekening neemt zodat ze niet opnieuw worden behandeld in algoritme \ref{alg:transition-to-frag}.

\subsection{Transitie uit een fragment}\label{sec:transitions-out}

Het doel is om te bepalen welke berichten dienen als punten waar het uitvoerpad een gecombineerd fragment verlaat (verder een `verlaatpunt'), en naar welke berichten de uitvoer springt onder welke voorwaarden. Daarom is het resultaat van deze procedure een mapping van verlaatpunt naar een verzameling van paren van bericht en voorwaarde die moet gelden om naar dat bericht te springen. We geven een overzicht van de procedure in algoritme \ref{alg:calcExitForMessages}.

\begin{algorithm}
	\thispagestyle{empty}
	\KwIn{\textit{fragment : gecombineerd fragment; uitvoer : lege verzameling van bericht $\rightarrow$ verzameling van \{bericht $\rightarrow$ string\}}}
	\KwOut{\textit{Verzameling van bericht $\rightarrow$ \{bericht $\rightarrow$ string\}}}
	\textit{\{laatsteBericht, aggregateVoorwaarde\} $\leftarrow$ bepaalLaatsteBericht(fragment)}; zie algoritme \ref{alg:determineFinalMessage} \\
	\If{laatsteBericht is niet leeg}{
	\textit{mapPaar $\leftarrow \emptyset$}; \\
	\eIf{fragment heeft geen ouder}{
	\textit{transitieNaarBuiten(fragment, mapPaar, (negatie van lusvoorwaarde als fragment een lusfragment is, anders $\epsilon$))}; zie algoritme \ref{alg:exitToOutside} \\
	\textit{uitvoer $\leftarrow$ uitvoer + (laatsteBericht $\rightarrow$ mapPaar)};}{
	\textit{concateneer aggregateVoorwaarde met negatie van lusvoorwaarde voor fragment als fragment een lus is} \\
	\textit{fragment $\leftarrow$ ouder van fragment}; \\
	\If{bericht na laatsteBericht is deel van fragment}{
	\textit{berichtNa $\leftarrow$ bericht na laatsteBericht}; \\
	\textit{kind $\leftarrow$ voorouder van fragment waar berichtNa rechtstreeks deel van is dat fragment als ouder heeft}; \\
	\textit{kinderenNa $\leftarrow$ \{kind\} $\cup$ alle kinderen van fragment die na kind komen}; \\
	\textit{als het eerste lid van kinderenNa geen lusfragment is, hou enkel dat eerste lid over; anders, hou eerste lid, alle lusfragmenten die meteen volgen op het eerste lid, en het meteen daaropvolgende fragment, als er \'e\'en is, over} \\
	\textit{transitiesIn $\leftarrow \emptyset$}; \\
	\ForEach{kindFragment $\in$ kinderenNa}{
	\textit{negaties $\leftarrow$ conjunctie van negaties van lusvoorwaarden van voorgaande lussen als die er zijn, anders ``''}; \\
	\textit{transitiesIn $\leftarrow$ transitiesIn $\cup$ bepaalTransitiesIn(kindFragment, negaties)}; \\}
	\textit{concateneer alle sprongvoorwaarden in transitiesIn met aggregateVoorwaarde} \\
	\textit{voeg alle elementen van transitiesIn toe aan mapPaar} \\
	\textit{uitvoer $\leftarrow$ uitvoer + (laatsteBericht $\rightarrow$ mapPaar);} \\
	\textit{ga naar stap 31 als het laatste lid van kinderenNa geen lus is, anders naar stap 23}}
	\eIf{fragment heeft een ouder}{
	\textit{concateneer aggregateVoorwaarde met negatie van lusvoorwaarde voor fragment als fragment een lus is} \\
	\textit{fragment $\leftarrow$ ouder van fragment;} \\
	\textit{ga naar stap 10}}{
	\textit{transitieNaarBuiten(fragment, mapPaar, aggregateVoorwaarde)}; \\
	\textit{uitvoer $\leftarrow$ uitvoer + (laatsteBericht $\rightarrow$ mapPaar)}; \\
	\textit{ga naar stap 31}}
	}
	}
	\ForEach{kind $\in$ kinderen van fragment dat initieel als argument werd gegeven}{
	\textit{bepaalTransitieUit(kind, uitvoer)}}
	\textbf{return};
	\caption{bepaalTransitieUit}
	\label{alg:calcExitForMessages}
\end{algorithm}

\begin{algorithm}
	\KwIn{fragment : gecombineerd fragment}
	\KwOut{Paar van bericht en string}
	\textit{aggregateVoorwaarde $\leftarrow \epsilon$}; \\
	\textit{containers $\leftarrow$ gesorteerde lijst van berichtcontainers van fragment}; \tcc{een berichtcontainer is ofwel \'e\'en enkel bericht of een gecombineerd fragment---op deze manier kunnen we een gecombineerd fragment beschouwen als een verzameling van berichtcontainers die bestaat uit de berichten die rechtstreeks deel zijn van het fragment en de kinderen van het fragment}
	\ForEach{container $\in$ containers, beginnend vanaf de laatste}{
	\If{container is een bericht}{
	\textbf{return} \textit{\{container, aggregateVoorwaarde\}};
	}
	\eIf{container is geen lus}{
	\textbf{return} \textit{\{$\emptyset$, aggregateVoorwaarde\}};}
	{
	\textit{aggregateVoorwaarde $\leftarrow$ aggregateVoorwaarde + ``$\land \lnot$('' + lusvoorwaarde van container + ``)''};
	}}
	\textbf{return} \textit{\{$\emptyset$, $\epsilon$\}};
	\caption{bepaalLaatsteBericht}
	\label{alg:determineFinalMessage}
\end{algorithm}

\begin{algorithm}
	\KwIn{\textit{fragment : gecombineerd fragment; mapPaar : verzameling van \{bericht $\rightarrow$ string\}; transitieVoorwaarde : string}}
	\If{bericht meteen na dit fragment is deel van een lusfragment zonder ouder}{
	\textit{volgenden $\leftarrow$ alle lussen die meteen volgen op fragment en het meteen daaropvolgende fragment, als er \'e\'en is}; \\
	\textit{aggregateVoorwaarde $\leftarrow \epsilon$}; \\
	\ForEach{volgendFragment $\in$ volgenden}{
	\textit{transitiesIn $\leftarrow$ bepaalTransitieIn(volgendFragment, aggregateVoorwaarde)}; \\
	\textit{concateneer alle sprongvoorwaarden in transitiesIn met transitieVoorwaarde}; \\
	\textit{mapPaar $\leftarrow$ mapPaar $\cup$ transitiesIn}; \\
	\If{volgendFragment is een lusfragment}{\textit{aggregateVoorwaarde $\leftarrow$ aggregateVoorwaarde + ``$\land \lnot$('' + voorwaarde voor lus + ``)''};}}
	\If{laatste lid van volgenden is geen lus}{
	\textbf{return};}
	}
	\textit{volgend $\leftarrow$ bericht dat meteen volgt op fragment}; \\
	\If{volgend is deel van gecombineerd fragment}{
	\textit{topFragment $\leftarrow$ voorouder van fragment waar volgend deel van is die zelf geen ouder heeft}; \\
	\textit{transitiesIn $\leftarrow$ bepaalTransitieIn(topFragment, ``'')}; \\
	\textit{concateneer alle sprongvoorwaarden in transitiesIn met transitieVoorwaarde}; \\
	\textit{mapPaar $\leftarrow$ mapPaar $\cup$ transitiesIn}; \\
	\textbf{return};}
	\textit{mapPaar $\leftarrow$ mapPaar + \{volgend $\rightarrow$ transitieVoorwaarde\}}; \\
	\textbf{return};
	\caption{transitieNaarBuiten}
	\label{alg:exitToOutside}
\end{algorithm}

Merk op in stap 3 dat \textit{laatsteBericht} later wordt gemapt naar een mapping van berichten naar de voorwaarde waaronder naar die berichten wordt gesprongen vanaf \textit{laatsteBericht}.
Algoritme \ref{alg:calcExitForMessages} gebruikt eerst algoritme \ref{alg:determineFinalMessage} om te bepalen of het laatste bericht van het fragment rechtstreeks deel is ervan. Zoniet, gaat het algoritme voort met de kinderen van het fragment. Zoja, controleren we of het fragment een ouder heeft. Als dat zo is, dan kan het zijn dat die ouder een bericht heeft dat in het diagram meteen na het laatste bericht van het fragment komt en dat het uitvoerpad dus mogelijk naar dat bericht springt. Als dat bericht rechtstreeks deel is van de ouder, dan wordt genoteerd dat de uitvoering vanaf het laatste bericht springt naar dat bericht en gaat het algoritme verder vanaf stap 31 (door ruimtegebrek is deze mogelijkheid niet neergeschreven in deze weergave van algoritme \ref{alg:calcExitForMessages}). Als het bericht in de plaats deel is van een kind van de ouder, haalt het algoritme het fragment op waarvan dat bericht rechtstreeks deel is, klimt het naar boven in de boom tot het een kind van de ouder tegenkomt, bundelt opeenvolgende lussen als dat kind een lusfragment is (en ook het fragment na die lussen, als er \'e\'en bestaat) en roept dan algoritme \ref{alg:transition-to-frag} op op alle gebundelde fragmenten. In de uitvoer wordt dan genoteerd dat het uitvoerpad kan springen naar de berichten die deel zijn van de uitvoer van algoritme \ref{alg:transition-to-frag}. Als op het laatste lusfragment een bericht volgt in plaats van een ander type fragment, noteren we dat naar dat bericht kan worden gesprongen (deze mogelijkheid staat niet in het algoritme door ruimtegebrek). Het algoritme gaat verder met de kinderen van het oorspronkelijk fragment.

Als het bericht na het laatste bericht geen deel is van de ouder, of als het laatste lid van \textit{kinderenNa} een lus is, of als het laatste lid van \textit{kinderenNa} geen lus is en er volgt geen bericht op, controleren we of de ouder zelf een ouder heeft. Zoja, neemt het algoritme een stap naar boven in de fragmentenboom. Bij deze stap markeren we eerst het huidige fragment zodat deze niet meer beschouwd wordt in verdere oproepen van algoritme \ref{alg:transition-to-frag} en concateneren we \textit{aggregateVoorwaarde} met de negaties van de lusvoorwaardes van de lussen op het einde van het fragment, als die er zijn (niet weergegeven in algoritme door plaatsgebrek). Zonee, betekent dit dat de top van de fragmentenboom is bereikt en dat het uitvoerpad de boom moet verlaten. Algoritme \ref{alg:exitToOutside} zorgt voor deze stap uit de boom. We controleren of de fragmentenboom wordt opgevolgd door een lusfragment en bundelen alle volgende lusfragmenten (en het meteen daaropvolgend fragment, als er \'e\'en bestaat) als dat zo is. We roepen algoritme \ref{alg:transition-to-frag} op met die fragmenten om de beurt als argument en registreren de uitvoer als berichten waarnaar gesprongen kan worden. Als er geen lus volgt op de boom, roepen we algoritme \ref{alg:transition-to-frag} op als er een fragment volgt, en anders noteren we het bericht dat meteen volgt op de boom als een bericht waarnaar gesprongen kan worden.

Merk op dat voor alt-fragmenten algoritme \ref{alg:calcExitForMessages} afzonderlijk wordt uitgevoerd voor het \textit{if}-gedeelte en het \textit{else}-gedeelte afzonderlijk.

\subsection{Het herhaaldelijk uitvoeren van een lus}\label{sec:loop-reentry}
Een laatste aspect is dat we ervoor moeten zorgen dat een lus terug wordt uitgevoerd als de laatste instructie bereikt is en de lusvoorwaarde nog geldt. We gebruiken algoritme \ref{alg:calculateLoopReentry} om te bepalen vanaf welke berichten terug wordt gesprongen naar de mogelijke beginpunten van een lusfragment.

\begin{algorithm}
	\KwIn{fragment : gecombineerd fragment}
	\KwOut{Verzameling van bericht $\rightarrow$ \{bericht $\rightarrow$ string\}}
	\textit{uitvoer $\leftarrow \emptyset$}; \\
	\textit{uitgesloten $\leftarrow \emptyset$}; \\
	\textit{aggregateVoorwaarde $\leftarrow \epsilon$}; \\
	\ForEach{laatsteBericht $\in$ mogelijke laatste berichten van fragment}{
	\Do{fragment heeft een ouder}{
	\If{fragment is een lusfragment en laatsteBericht is een mogelijk laatst bericht van fragment}{
	\textit{containers $\leftarrow$ containers van fragment}; \\
	\eIf{eerste container is lus}{
	\textit{bundelVoorwaarde $\leftarrow$ voorwaarde van lus als de laatste container een fragment is en laatsteBericht is daar een laatste bericht van, anders $\epsilon$}; \\
	\ForEach{container $\in$ containers}{
	\eIf{container is een fragment en container $\notin$ uitgesloten}{
	\textit{transities $\leftarrow$ bepaalTransitieInOnvolledig(fragment, $\epsilon$, uitgesloten)}; \\
	\textit{concateneer alle sprongvoorwaardes in transities met aggregateVoorwaarde en bundelVoorwaarde}; \\
	\ForEach{bericht---voorwaarde-paar $\in$ transities}{
	\textit{uitvoer $\leftarrow$ uitvoer + \{laatsteBericht $\rightarrow$ \{bericht $\rightarrow$ voorwaarde\}\}};
	}
	}{
	\textit{uitvoer $\leftarrow$ uitvoer + \{laatsteBericht $\rightarrow$ \{container $\rightarrow$ aggregateVoorwaarde + bundelVoorwaarde\}\};}
	}
	\eIf{container is een lus}{
	\textit{bundelVoorwaarde $\leftarrow$ bundelVoorwaarde + ''$\land \lnot$(``voorwaarde voor container + ``)''};
	}{
	\textbf{break};}
	}}
	{
	\textit{transities $\leftarrow$ bepaalTransitieInOnvolledig(fragment, $\epsilon$, uitgesloten)}; \\
	\textit{concateneer alle sprongvoorwaardes in transities met aggregateVoorwaarde}; \\
	\ForEach{bericht---voorwaarde-paar $\in$ transities}{
		\textit{uitvoer $\leftarrow$ uitvoer + \{laatsteBericht $\rightarrow$ \{bericht $\rightarrow$ voorwaarde\}\}};
	}
	}
	}
	\If{fragment is een lusfragment}{
	\textit{uitgesloten $\leftarrow$ uitgesloten + fragment};}
	\If{fragment heeft een ouder}{
	\If{fragment is een lusfragment}{\textit{aggregateVoorwaarde $\leftarrow$ aggregateVoorwaarde + ``$\land \lnot$('' + voorwaarde voor lusfragment + ``)''};}
	\textit{fragment $\leftarrow$ ouder van fragment};
	}}}
	\caption{bepaalLusHeruitvoering}
	\label{alg:calculateLoopReentry}
\end{algorithm}

We gebruiken hier een variant van \textit{bepaalTransitieInOnvolledig} waarbij de fragmenten doorgegeven in \textit{uitgesloten} worden overgeslagen.
De mogelijke laatste berichten van een fragment zijn deze waarbij het uitvoerpad het fragment verlaat nadat ze zijn uitgevoerd, bijvoorbeeld de laatste berichten van de \textit{if}- en \textit{else}-gedeeltes van een alt-fragment of een bericht dat voorafgaat aan een lusfragment.

Vanaf zulk een laatste bericht gaan we van onderaf naar boven in de boom, beginnend bij het fragment dat als argument wordt gegeven van de oproep. Telkens we een lusfragment tegenkomen waarvan dat bericht een laatste bericht is, moeten we voor dat fragment alle mogelijke beginpunten vinden. We beschouwen de \textit{containers} voor het fragment. Indien de eerste \textit{container} een lus is, controleren we of de laatste \textit{container} een lus is en of \textit{laatsteBericht} een laatst bericht ervan is. In dat geval moeten we verzekeren dat er rekening wordt gehouden met de lusvoorwaarde van de lus op dit niveau wanneer we \textit{bepaalTransitieInOnvolledig} gebruiken. Verder gaan we de \textit{containers} af: We bundelen de negaties van de voorwaardes van voorafgaande lusfragmenten samen, we bepalen de transitiepunten ervan en we gaan naar boven in de boom nadat we een \textit{container} zijn tegengekomen die geen lus is of als we alle \textit{containers} hebben beschouwd.

Als de eerste \textit{container} geen lus is, bepalen we de transitiepunten van het fragment en gaan dan naar boven in de boom.

\subsection{De vertaling van transities}\label{sec:process-frag}

Nu volgt een beschrijving van hoe we de voorgaande algoritmes gebruiken om een verzameling van mappings van berichten waarnaar gesprongen wordt naar een mapping van berichten waarvan gesprongen wordt en onder welke voorwaarde vanaf die berichten wordt gesprongen bij te houden. We geven een beschrijving op hoog niveau in algoritme \ref{alg:processCombinedFragment}. Bij de uitvoering ervan werken we vanuit de veronderstelling dat alle berichten en fragmenten van een diagram toegankelijk zijn.

\begin{algorithm}
	\KwIn{fragment : gecombineerd fragment; uitvoer : verzameling van bericht $\rightarrow$ verzameling van \{bericht $\rightarrow$ string\}}
	\If{fragment is een lusfragment, heeft geen ouder en wordt voorgegaan door een bericht dat geen deel is van een gecombineerd fragment}{
	\textit{noteer in uitvoer dat dit fragment en de lussen die er direct op volgen worden overgeslagen als de lusvoorwaarde voor geen enkele ervan geldt}
	}
	\eIf{er is een fragment dat vlak v\'o\'or fragment komt}{
	\textit{vorig $\leftarrow$ fragment dat vlak v\'o\'or fragment komt}; \\
	\textit{verwerkFragment(vorig, uitvoer)}; \\
	\textit{bepaalVerlaatpunten(fragment, uitvoer)}; \\
	\textit{verzekerLusHeruitvoering(fragment, uitvoer)};
	}{
	\textit{transitieIn $\leftarrow$ bepaalTransitieIn(fragment, $\epsilon$)}; \\
	\ForEach{\{bericht---voorwaarde\} $\in$ transitieIn}{
	\textit{berichtVoor $\leftarrow$ bericht dat v\'o\'or bericht komt}; \\
	\eIf{berichtVoor is geen deel van een fragment}{
	\textit{uitvoer $\leftarrow$ uitvoer + \{berichtVoor $\rightarrow$ \{bericht $\rightarrow$ voorwaarde\}\}};
	}
	{
	\textit{vorigFragment $\leftarrow$ fragment waar berichtVoor deel van is}; \\
	\textit{transitiesUit $\leftarrow$ bepaalTransitieUit(vorigFragment);} \\
	\textit{voor alle voorkomens van bericht in transitiesUit als bericht waarnaar gesprongen wordt, doe uitvoer $\leftarrow$ uitvoer + \{sprongbericht $\rightarrow$ \{bericht $\rightarrow$ sprongvoorwaarde horend bij bericht\}\}
	}}
	}
	\textit{bepaalVerlaatpunten(fragment, uitvoer);} \\
	\textit{verzekerLusHeruitvoering(fragment, uitvoer)};
	}
	\caption{verwerkFragment}
	\label{alg:processCombinedFragment}
\end{algorithm}

\textit{bepaalVerlaatPunten} gebruikt algoritme \ref{alg:calcExitForMessages} om te bepalen naar welke berichten wordt gesprongen onder welke voorwaarde vanuit het fragment. \textit{verzekerLusHeruitvoering} bouwt een fragmentenboom op met het gegeven fragment als wortel en roept algoritme \ref{alg:calculateLoopReentry} op met elke knoop in de boom om de beurt als argument om te verzekeren dat lussen worden heruitgevoerd wanneer de lusvoorwaarde nog geldt.

Bij het vertalen van sequentiediagrammen roepen we algoritme \ref{alg:processCombinedFragment} op met alle fragmenten in het diagram om de beurt als argument. De resulterende informatie geeft ons:

\begin{enumerate}
	\item De berichten waarnaar gesprongen wordt
	\item De berichten vanwaar gesprongen wordt
	\item Onder welke voorwaarde zulk een sprong gebeurt
\end{enumerate}

We gebruiken dit om de eigenlijke vertaling van gecombineerde fragmenten naar FO($\cdot$) te bepalen.

\subsection{Voorbeeld van gecombineerd fragment vertaling}

We defini\"eren eerst notatie die we verder gebruiken bij het uitwerken van een voorbeeld:

\begin{align*}
	\transitionentry{\fragcond{voorwaarde}}{\fragmessage{$bericht_a$}}
\end{align*}

Dit betekent dat de uitvoering kan springen naar $bericht_a$ onder \textit{\fragcond{voorwaarde}}.

\begin{align*}
	\transitionentry[\fragmessage{$bericht_b$}]{\fragcond{voorwaarde}}{\fragmessage{$bericht_a$}}
\end{align*}

Dit betekent dat de uitvoering vanaf \fragmessage{$bericht_b$} springt naar \fragmessage{$bericht_a$} onder \textit{\fragcond{voorwaarde}}.

\subsubsection{De uitvoering van de voorgestelde algoritmes}

In deze sectie gebruiken we figuur \ref{fig:seq-diagram-frag-ex} om een voorbeeld te geven van de vertaling van gecombineerde fragmenten volgens de algoritmes beschreven in secties \ref{sec:transition-to}, \ref{sec:transitions-out}, \ref{sec:loop-reentry} en \ref{sec:process-frag}.

\begin{figure}
	\includegraphics[width=1\textwidth]{chap-gedrag/seq-diagram-frag-ex.png}
	\caption{Sequentiediagram voor een voorbeeldvertaling van gecombineerde fragmenten}
	\label{fig:seq-diagram-frag-ex}
\end{figure}

In wat volgt gebruiken we de voorwaarde voor een fragment als naam voor het fragment zelf. Om het onderscheid te maken tussen deze twee manieren waarop we de voorwaarde gebruiken, schrijven we \fragname{loop1} om aan te duiden dat de voorwaarde als naam wordt gebruikt en \fragcond{loop1} wanneer we het effectief als voorwaarde voor de uitvoering van een fragment gebruiken. \fragname{alt1} verwijst naar het gehele alt-fragment en \fragname{alt1a} en \fragname{alt1b} verwijzen respectievelijk naar het \textit{if}-deel en het \textit{else}-deel.

We roepen algoritme \ref{alg:processCombinedFragment} eerst op op met \fragname{loop1} als argument. Aangezien het geen ouder heeft en \fragmessage{message1}, wat geen fragment heeft, eraan voorafgaat, noteren we in de uitvoer:

\begin{align*}
	&\transitionentry[\fragmessage{message1}]{\lnot \fragcond{loop1}}{\fragmessage{message10}}
\end{align*}

Er is geen fragment dat v\'o\'or dit fragment komt, dus roepen we algoritme \ref{alg:transition-to-frag} op. \fragmessage{message2} is rechtstreeks deel van \fragname{loop1}, dus noteren we in de uitvoer voor dit algoritme:

\begin{align*}
	\transitionentry{\fragcond{loop1}}{\fragmessage{message2}}
\end{align*}

Hierna gaan we verder naar algoritme \ref{alg:wrap-loops} met \fragname{alt1} en \fragname{loop2} als invoerfragmenten en \textit{allemaal} gezet naar \textbf{true}. \fragname{alt1} is geen lusfragment en slaan we over. Vervolgens roepen we algoritme \ref{alg:transition-to-frag} op met \fragname{loop2} als argument.
\fragmessage{message6} is niet rechtstreeks deel van \fragname{loop2}, dus zetten we \textit{voorwaarde} naar \fragcondt{loop2} en roepen eerst algoritme \ref{alg:transition-to-frag} op met \fragname{loop3} als argument. Op dit niveau wordt \textit{voorwaarde} gezet naar ``\fragcond{loop2} $\land$ \fragcond{loop3}'' en roepen we algoritme \ref{alg:transition-to-frag} op met \fragname{alt2} als argument. \fragmessage{message6} en \fragmessage{message7} zijn rechtstreeks deel van \fragname{alt2}, dus hebben we:

\begin{align*}
	&\transitionentry{\fragcond{loop2} \land \fragcond{loop3} \land \fragcond{alt2a}}{\fragmessage{message6}} \\
	&\transitionentry{\fragcond{loop2} \land \fragcond{loop3} \land \fragcond{alt2b}}{\fragmessage{message7}}
\end{align*}

We keren terug naar het niveau van \fragname{loop2}. \fragname{loop3} was een lus, dus zetten we \textit{voorwaarde} naar ``\fragcond{loop2} $\land \lnot$ \fragcond{loop3}'' en roepen we algoritme \ref{alg:wrap-loops} op met als argument de verzameling met enkel \fragname{loop4} als lid. Het resultaat van die oproep is dat we neerschrijven:

\begin{align*}
	\transitionentry{\fragcond{loop2} \land \lnot \fragcond{loop3} \land \fragcond{loop4}}{\fragmessage{message8}}
\end{align*}

We keren terug naar het niveau van \fragname{loop2}. Dit fragment heeft verder geen kinderen, dus eindigt deze oproep van algoritme \ref{alg:wrap-loops} hier.

We keren terug naar stap 7 in algoritme \ref{alg:transition-to-frag} voor \fragname{loop1}. \fragname{alt1} is niet eerder gemarkeerd door algoritme \ref{alg:wrap-loops}, dus gebruiken we het als argument voor een oproep van algoritme \ref{alg:transition-to-frag}. Het resultaat van die oproep is:

\begin{align*}
	&\transitionentry{\fragcond{alt1a}}{\fragmessage{message3}} \\
	&\transitionentry{\fragcond{alt1b}}{\fragmessage{message5}}
\end{align*}

Dit markeert het einde van algoritme \ref{alg:transition-to-frag} voor \fragname{loop1}. In algoritme \ref{alg:processCombinedFragment} gaan we nu over naar stap 10. Na het uitvoeren van de lus verkrijgen we:

\begin{align*}
	&\transitionentry[\fragmessage{message1}]{\fragcond{loop1}}{\fragmessage{message2}} \\
	&\transitionentry[\fragmessage{message2}]{\fragcond{alt1a}}{\fragmessage{message3}} \\
	&\transitionentry[\fragmessage{message2}]{\fragcond{alt1b}}{\fragmessage{message5}} \\
	&\transitionentry[\fragmessage{message4}]{\fragcond{loop2} \land \fragcond{loop3} \land \fragcond{alt2a}}{\fragmessage{message6}} \\
	&\transitionentry[\fragmessage{message5}]{\fragcond{loop2} \land \fragcond{loop3} \land \fragcond{alt2a}}{\fragmessage{message6}} \\
	&\transitionentry[\fragmessage{message4}]{\fragcond{loop2} \land \fragcond{loop3} \land \fragcond{alt2b}}{message7} \\
	&\transitionentry[\fragmessage{message5}]{\fragcond{loop2} \land \fragcond{loop3} \land \fragcond{alt2b}}{message7} \\
	&\transitionentry[\fragmessage{message6}]{\fragcond{loop2} \land \lnot \fragcond{loop3} \land \fragcond{loop4}}{message8} \\
	&\transitionentry[\fragmessage{message7}]{\fragcond{loop2} \land \lnot \fragcond{loop3} \land \fragcond{loop4}}{message8}
\end{align*}

Nu bereiken we stap 18 in algoritme \ref{alg:processCombinedFragment}. Deze stap houdt in dat we algoritme \ref{alg:calcExitForMessages} oproepen met \fragname{loop1} als argument.

In stap 1 roepen we eerst algoritme \ref{alg:determineFinalMessage} op met \fragname{loop1} als argument. De uitkomst daarvan is dat \fragmessage{message9} herkend wordt als laatste bericht. \fragname{loop1} heeft geen ouder, dus gebruiken we algoritme \ref{alg:exitToOutside} met \fragname{loop1} als fragment en ``$\lnot$ \fragcond{loop1}'' als transitievoorwaarde als invoer. Er volgen geen fragmenten op \fragname{loop1} en \fragmessage{message10} is geen deel van een fragment, dus noteren we:

\begin{align*}
	\transitionentry{\lnot \fragcond{loop1}}{\fragmessage{message10}}
\end{align*}

Voor stap 6 in algoritme \ref{alg:calcExitForMessages} noteren we:

\begin{align*}
	\transitionentry[\fragmessage{message9}]{\lnot \fragcond{loop1}}{\fragmessage{message10}}
\end{align*}

Algoritme \ref{alg:calcExitForMessages} gaat nu verder met de kinderen van \fragname{loop1}. \fragname{alt1} komt eerst aan bod, en we bekijken eerst het \textit{if}-deel. Het resultaat van algoritme \ref{alg:determineFinalMessage} is \textit{\{\fragmessage{message4}, $\epsilon$\}}. \fragname{loop1} is de ouder van \fragname{alt1} en \fragmessage{message6} is deel van \fragname{loop1}, dus we gaan naar stap 11. In wat volgt roepen we algoritme \ref{alg:transition-to-frag} op met \fragname{loop2} als argument. Het resultaat daarvan gebruiken we om te noteren:

\begin{align*}
	&\transitionentry[\fragmessage{message4}]{\fragcond{loop2} \land \fragcond{loop3} \land \fragcond{alt2a}}{\fragmessage{message6}} \\
	&\transitionentry[\fragmessage{message4}]{\fragcond{loop2} \land \fragcond{loop3} \land \fragcond{alt2b}}{\fragmessage{message7}} \\
	&\transitionentry[\fragmessage{message4}]{\fragcond{loop2} \land \lnot \fragcond{loop3} \land \fragcond{loop4}}{\fragmessage{message8}}
\end{align*}

Als gevolg van het feit dat \fragmessage{message9} volgt op \fragname{loop2}, noteren we ook:

\begin{align*}
	&\transitionentry[\fragmessage{message4}]{\lnot \fragcond{loop2}}{message9}
\end{align*}

Het voorgaande wordt herhaald voor het \textit{else}-deel. Op gelijkaardige wijze voor het \textit{if}-deel leidt dit tot:

\begin{align*}
	&\transitionentry[\fragmessage{message5}]{\fragcond{loop2} \land \fragcond{loop3} \land \fragcond{alt2a}}{\fragmessage{message6}} \\
	&\transitionentry[\fragmessage{message5}]{\fragcond{loop2} \land \fragcond{loop3} \land \fragcond{alt2b}}{\fragmessage{message7}} \\
	&\transitionentry[\fragmessage{message5}]{\fragcond{loop2} \land \lnot \fragcond{loop3} \land \fragcond{loop4}}{\fragmessage{message8}} \\
	&\transitionentry[\fragmessage{message5}]{\lnot \fragcond{loop2}}{message9}
\end{align*}

We gaan terug naar stap 31 voor \fragname{loop1}. Algoritme \ref{alg:calcExitForMessages} wordt nu opgeroepen op \fragname{loop2}. Via algoritme \ref{alg:determineFinalMessage} concluderen we dat \fragname{loop2} geen laatste bericht heeft, dus roepen we algoritme \ref{alg:calcExitForMessages} op met elk kind om de beurt als argument. \fragname{loop3} komt eerst. Op gelijkaardige wijze gaan we verder naar \fragname{alt2}.

Het \textit{if}-deel van \fragname{alt2} komt eerst. Algoritme \ref{alg:determineFinalMessage} besluit dat \fragmessage{message6} het laatste bericht is. \fragmessage{message8} is geen deel van \fragname{loop3}, dus zetten we \textit{aggregateVoorwaarde} naar ``$\lnot$ \fragcond{loop3}'' en \textit{fragment} naar \fragname{loop2} en gaan naar stap 10. \fragname{message8} is deel van \fragname{loop2}, dus roepen we algoritme \ref{alg:transition-to-frag} op met \fragname{loop4} als argument en concateneren \textit{aggregateVoorwaarde}, wat resulteert in:

\begin{align*}
	\transitionentry[\fragmessage{message6}]{\fragcond{loop4} \land \lnot \fragcond{loop3}}{\fragmessage{message8}}
\end{align*}

\fragname{loop4} is een lus, dus zetten we \textit{fragment} naar \fragname{loop1} en gaan naar stap 10. \fragmessage{message8} is deel van \fragname{loop1}, maar \fragname{loop2} is gemarkeerd in de vorige iteratie en slaan we dus over. We zien wel dat \fragname{loop1} een bericht heeft na \fragmessage{message8}, namelijk \fragmessage{message9}, en daarom noteren we:

\begin{align*}
	\transitionentry[\fragmessage{message6}]{\lnot \fragcond{loop3} \land \lnot \fragcond{loop4} \land \lnot \fragcond{loop2}}{\fragmessage{message9}}
\end{align*}

Met het vinden van dat bericht na \fragmessage{message8}, eindigt het algoritme voor het \textit{if}-deel van \fragname{alt2}. Nu komt het \textit{else}-deel van \fragname{alt2} aan bod, en gelijkaardig voor het \textit{if}-deel krijgen we:

\begin{align*}
		&\transitionentry[\fragmessage{message7}]{\fragcond{loop4} \land \lnot \fragcond{loop3}}{\fragmessage{message8}} \\
		&\transitionentry[\fragmessage{message7}]{\lnot \fragcond{loop3} \land \lnot \fragcond{loop4} \land \lnot \fragcond{loop2}}{\fragmessage{message9}}
\end{align*}

\fragname{loop3} is nu volledig behandeld, en we gaan terug naar stap 31 voor \fragname{loop2}. We roepen algoritme \ref{alg:calcExitForMessages} op met \fragname{loop4} als argument. Algoritme \ref{alg:determineFinalMessage} besluit dat \fragmessage{message8} het laatste bericht is en we zetten \textit{aggregateVoorwaarde} naar ``$\lnot$ \fragcond{loop4}''. \fragname{loop2} heeft geen bericht na \fragmessage{message8}, dus zetten we fragment naar \fragname{loop1} en gaan terug naar stap 8. \fragmessage{message9} komt in \fragname{loop1} meteen na \fragmessage{message8} en is rechtstreeks deels van \fragname{loop1}, dus noteren we:

\begin{align*}
	\transitionentry[\fragmessage{message8}]{\lnot \fragcond{loop4} \land \lnot \fragcond{loop2}}{\fragmessage{message9}}
\end{align*}

Hier stopt de uitvoering van het algoritme voor \fragname{loop4}, en hiermee meteen ook voor \fragname{loop2} en \fragname{loop1}.

\parbreak

We bereiken stap 19 in algoritme \ref{alg:processCombinedFragment}. In plaats van \fragname{loop1} te gebruiken als argument zoals het zou zijn in een echte uitvoering, gebruiken we \fragname{loop4} als illustratiever voorbeeld.

\fragmessage{message8} is vanzelfsprekend het laatste bericht van \fragname{loop4}. We roepen de variant van algoritme \ref{alg:transition-to-frag} gebruikt in dit algoritme op met \fragname{loop4} als argument. De uitkomst van deze iteratie is:

\begin{align*}
	\transitionentry[\fragmessage{message8}]{\fragcond{loop4}}{\fragmessage{message8}}
\end{align*}

We voegen \fragname{loop4} toe aan \textit{uitgesloten} en zetten \textit{aggregateVoorwaarde} naar ``$\lnot$ \fragcond{loop4}'' en \textit{fragment} naar \fragname{loop2}. De laatste container van \fragname{loop2} is een fragment en \fragmessage{message8} is daar het laatste bericht van, dus zetten we \textit{bundelVoorwaarde} naar ``\fragcond{loop2}''. \fragname{loop4} is lid van \textit{uitgesloten}, dus noteren we:

\begin{align*}
	&\transitionentry[\fragmessage{message8}]{\lnot \fragcond{loop4} \land \fragcond{loop2} \land \fragcond{loop3} \land \fragcond{alt2a}}{\fragmessage{message6}} \\
	&\transitionentry[\fragmessage{message8}]{\lnot \fragcond{loop4} \land \fragcond{loop2} \land \fragcond{loop3} \land \fragcond{alt2b}}{\fragmessage{message7}}
\end{align*}

We voegen \fragname{loop2} toe aan \textit{uitgesloten}, concateneren \textit{aggregateVoorwaarde} met ``$\land \lnot$ \fragcond{loop2}'' en gaan verder met \fragname{loop1}. \fragmessage{message8} is geen laatste bericht van \fragname{loop1}, en \fragname{loop1} heeft geen ouder. De uitvoering van algoritme \ref{alg:calculateLoopReentry} stopt.

\parbreak

Nadat we algoritme \ref{alg:processCombinedFragment} hebben uitgevoerd op alle fragmenten, rest de taak van de uitvoer van dat algoritme te vertalen naar logica.

\subsubsection{De uitvoer van de algoritmes vertalen naar logica}

Het is eenvoudig om de uitvoer te vertalen naar logica. We zoeken naar gevallen waar het bericht waarnaar gesprongen wordt en de voorwaarde waaronder die sprong gebeurt overeenkomen en combineren ze. Als voorbeeld:

\begin{align*}
	&\transitionentry[\fragmessage{$message_b$}]{\fragcond{voorwaarde}}{\fragmessage{$message_a$}} \\
	&\transitionentry[\fragmessage{$message_c$}]{\fragcond{voorwaarde}}{\fragmessage{$message_a$}}
\end{align*}

Dit vertalen we naar:

\begin{align*}
	\forall{t}[Time](C\_SDPointAt(Next(t), message_a) \leftarrow (SDPointAt(t, message_b) \\ \lor SDPointAt(t, message_c)) \land \fragcond{voorwaarde}).
\end{align*}

De uitvoer van algoritme \ref{alg:processCombinedFragment} voor alle fragmenten in figuur \ref{fig:seq-diagram-frag-ex} vertalen we op die manier als volgt naar logica:

\begin{align*}
	&\forall{t}[Time](C\_SDPointAt(Next(t), 2) \leftarrow (SDPointAt(t, 1) \\ &\lor SDPointAt(t, 9)) \land \fragcond{loop1}). \\
	&\forall{t}[Time](C\_SDPointAt(Next(t), 3) \leftarrow SDPointAt(t, 2) \\ &\land \fragcond{alt1a}). \\
	&\forall{t}[Time](C\_SDPointAt(Next(t), 5) \leftarrow SDPointAt(t, 2) \\ &\land \fragcond{alt1b}). \\
	&\forall{t}[Time](C\_SDPointAt(Next(t), 6) \leftarrow (SDPointAt(t, 6) \\ &\lor SDPointAt(t, 7)) \land \fragcond{loop3} \land \fragcond{alt2a})). \\
	&\forall{t}[Time](C\_SDPointAt(Next(t), 6) \leftarrow SDPointAt(t, 8) \\ &\land \lnot \fragcond{loop4} \land \fragcond{loop2} \land \fragcond{loop3} \land \fragcond{alt2a}). \\
	&\forall{t}[Time](C\_SDPointAt(Next(t), 6) \leftarrow (SDPointAt(t, 4) \\ &\lor SDPointAt(t, 5)) \land \fragcond{loop2} \land \fragcond{loop3} \land \fragcond{alt2a}). \\
	&\forall{t}[Time](C\_SDPointAt(Next(t), 7) \leftarrow (SDPointAt(t, 6) \\ &\lor SDPointAt(t, 7)) \land \fragcond{loop3} \land \fragcond{alt2b})). \\
	&\forall{t}[Time](C\_SDPointAt(Next(t), 7) \leftarrow SDPointAt(t, 8) \\ &\land \lnot \fragcond{loop4} \land \fragcond{loop2} \land \fragcond{loop3} \land \fragcond{alt2b}). \\
	&\forall{t}[Time](C\_SDPointAt(Next(t), 7) \leftarrow (SDPointAt(t, 4) \\ &\lor SDPointAt(t, 5)) \land \fragcond{loop2} \land \fragcond{loop3} \land \fragcond{alt2b}).
\end{align*}

\begin{align*}
	&\forall{t}[Time](C\_SDPointAt(Next(t), 8) \leftarrow (SDPointAt(t, 6) \\ &\lor SDPointAt(t, 7)) \land \lnot \fragcond{loop3} \land \fragcond{loop4}). \\
	&\forall{t}[Time](C\_SDPointAt(Next(t), 8) \leftarrow (SDPointAt(t, 4) \\ &\lor SDPointAt(t, 5)) \land \fragcond{loop2} \land \lnot \fragcond{loop3} \\ &\land \fragcond{loop4}). \\
	&\forall{t}[Time](C\_SDPointAt(Next(t), 8) \leftarrow SDPointAt(t, 8) \\ &\land \fragcond{loop4}). \\
	&\forall{t}[Time](C\_SDPointAt(Next(t), 9) \leftarrow (SDPointAt(t, 6) \\ &\lor SDPointAt(t, 7)) \land \lnot \fragcond{loop3} \land \lnot \fragcond{loop4} \\ &\land \lnot \fragcond{loop2}). \\
	&\forall{t}[Time](C\_SDPointAt(Next(t), 9) \leftarrow SDPointAt(t, 8) \\ &\land \lnot \fragcond{loop4} \land \lnot \fragcond{loop2}). \\
	&\forall{t}[Time](C\_SDPointAt(Next(t), 9) \leftarrow (SDPointAt(t, 4) \\ &\lor SDPointAt(t, 5)) \land \lnot \fragcond{loop2}). \\
	&\forall{t}[Time](C\_SDPointAt(Next(t), 10) \leftarrow (SDPointAt(t, 1) \\ &\lor SDPointAt(t, 9)) \land \lnot \fragcond{loop1}).
\end{align*}

\section{Interactie tussen meerdere sequentiediagrammen}\label{sec:interaction}
In een project stelt men doorgaans meerdere sequentiediagrammen op die elkaar ook kunnen oproepen. Men gebruikt soms ook recursie in deze diagrammen. In deze sectie beschrijven we hoe we het oproepen van andere sequentiediagrammen en een recursiemechanisme ondersteunen.

\begin{figure}
	\centering
	\includegraphics[width=0.25\textwidth]{chap-gedrag/recursion-class.png}
	\caption{Klasse gebruikt in voorbeeld over recursie}
	\label{fig:recursion-class}
\end{figure}

%\begin{landscape}
%\thispagestyle{empty}
%	\begin{sidewaysfigure}[htp]
%		\centering
%		\begin{subfigure}{\textwidth}
%			\includegraphics[height=0.4\textwidth]{chap-gedrag/methodOne.png}
%			\caption{Sequentiediagram voor methodOne()}
%			\label{fig:methodOne}
%		\end{subfigure}%
%		\begin{subfigure}{\textwidth}
%			\includegraphics[height=0.4\textwidth]{chap-gedrag/methodTwo.png}
%			\caption{Sequentiediagram voor methodTwo()}
%			\label{fig:methodtwo}
%		\end{subfigure}
%		\caption{Sequentiediagrammen voor klasse A in figuur \ref{fig:recursion-class}}
%		\label{fig:seq-recursion}
%	\end{sidewaysfigure}
%\end{landscape}

\begin{figure}[htp]
	\centering
	\includegraphics[width=0.5\textwidth]{chap-gedrag/methodOne.png}
	\caption{Sequentiediagram voor methodOne()}
	\label{fig:methodOne}
\end{figure}%

\begin{figure}[htp]
	\centering
	\includegraphics[width=\textwidth]{chap-gedrag/methodTwo.png}
	\caption{Sequentiediagram voor methodTwo()}
	\label{fig:methodtwo}
\end{figure}

In de volgende subsecties gebruiken we het voorbeeld uitgebeeld in figuren \ref{fig:recursion-class} en \ref{fig:methodOne} en \ref{fig:methodtwo} om de gebruikte principes te illustreren.

\subsection{Aanpassingen aan \textit{SDPoint}}
Het is niet meer voldoende om \textit{SDPoint}s te modelleren als natuurlijke getallen aangezien elk diagram zijn eigen \textit{SDPoint}s heeft. Om de \textit{SDPoint}s horende bij elk diagram van elkaar te kunnen onderscheiden, maken we van het logisch type \textit{SDPoint} nu een \textit{constructed type} in IDP. We gebruiken als patroon voor de naamgeving van elk logisch object $<diagramnaam>\_<instructienummer>$. Voor het voorbeeld krijgen we o.a. $methodOne\_2$ en $methodTwo\_3$. We voegen ook een nieuwe functie toe aan het vocabularium dat gegeven een \textit{SDPoint} het volgende \textit{SDPoint} teruggeeft, namelijk $NextSD(SDPoint) : SDPoint$.

Een andere aanpassing is dat we een virtuele \textit{SDPoint} inpassen na elke instructie voor een oproep. Het naamgevingspatroon hiervoor is $<diagramnaam>\_<instructienummer>post$. Aangezien de derde instructie in figuur \ref{fig:methodOne} een oproep is, krijgen we $methodOne\_3post$. Met deze toevoeging ontkoppelen we twee zaken die bij een oproep komen kijken: Enerzijds dat het mogelijk is dat het resultaat van een oproep wordt toegekend aan een variabele; en anderzijds dat na een oproep er mogelijks een alt-fragment of lusfragment volgt, en dat het dus v\'o\'or de oproep niet noodzakelijk duidelijk is welke de volgende instructie is na de uitvoering van de oproep. Het zou niet mogelijk zijn om voor deze twee zaken het correcte gedrag te verkrijgen zonder een oproepinstructie op deze manier op te splitsen.

\subsection{Het stapelmechanisme}
Om oproepen van andere sequentiediagrammen correct uit te voeren, moet de theorie bijhouden welke variabelen er bestaan tijdens een bepaalde oproep. Bovendien moet de theorie voor recursieve oproepen ook de waardes van een set variabelen kunnen bewaren v\'o\'or een oproep en die waardes herstellen na een oproep. Hiertoe ontwerpen we een stapelmechanisme. Er gebeuren volgende aanpassingen aan het vocabularium:

\begin{itemize}
	\item Toevoeging van het logisch type $StackLevel \subset \mathbb{N}$: Dit stelt de oproepdiepte van een oproep voor.
	\item Toevoeging van een inerti\"ele functie $CurrentStackLevel(Time) : StackLevel$: De oproepdiepte op een bepaald tijdstip.
	\item Toevoeging van een intertieel predicaaat \\ $ReturnPoint(Time, StackLevel, SDPoint)$: Op een bepaald tijdstip, de \textit{SDPoint} waarnaar de uitvoering moet terugkeren wanneer de laatste instructie voor de gegeven \textit{StackLevel} bereikt is.
	\item Alle diagramvariabelen worden nu gemodelleerd door een ternair predicaat dat nu ook de oproepdiepte in rekening neemt. Voor het voorbeeld krijgen we dus bijvoorbeeld $FinishedT(Time, StackLevel, bool)$.
\end{itemize}

De volgende subsecties beschrijven hoe we in het definitieblok voor de causatiezinnen dit stapelmechanisme gebruiken.

\subsubsection{Causatiezinnen voor \textit{SDPointAt/2}}\label{sec:sd-rec-cause}
Oproepinstructies betekenen bijkomende uitzonderingen op het normale verloop van \textit{SDPoints} naast deze die voortkomen uit alt-- en lusfragmenten. In dit geval zijn $methodOne\_3$, $methodTwo\_3$, $methodOne\_5$, $methodTwo\_5$ en $finished$ de nieuwe uitzonderingen. $methodOne\_5$ en $methodTwo\_5$ zijn ook \textit{SDPoint}s die niet in het diagram terug te vinden zijn, maar die we zelf toevoegen. Dit zijn impliciete terugkeerinstructies die we toevoegen voor diagrammen die \textit{void} als resultaat hebben. $finished$ is een speciale \textit{SDPoint} die het einde van de uitvoering aanduidt. De zin die het normale verloop van \textit{SDPoint}s regelt wordt dus:

\begin{align}
	& \nonumber \forall{t}[Time]\forall{s}[SDPoint](C\_SDPointAt(Next(t), NextSD(s) \leftarrow \\ \nonumber &SDPointAt(t,s) \land \lnot((s = methodOne\_3) \lor (s = methodOne\_5) \\ \nonumber &\lor (s = methodTwo\_1) \lor (s = methodTwo\_3) \lor (s = methodTwo\_3post) \\ &\lor (methodTwo\_4) \lor (methodTwo\_5) \lor (s = finished)).
\end{align}

De uitzonderingen als resultaat van een oproep worden als volgt gemodelleerd:

\begin{align}
	 \forall{t}[Time](C\_SDPointAt(Next(t), methodTwo\_1) \leftarrow SDPointAt(t, methodOne\_3).\label{eq:callOne} \\
	 \forall{t}[Time](C\_SDPointAt(Next(t), methodTwo\_1) \leftarrow SDPointAt(t, methodTwo\_3).\label{eq:callTwo}
\end{align}

Zin \ref{eq:callOne} resulteert uit instructie 3 van het diagram voor \textit{methodOne} en zin \ref{eq:callTwo} resulteert uit instructie 3 van het diagram voor \textit{methodTwo}.

De tweede aanpassing is dat er een terugkeer moet gebeuren wanneer het einde van een sequentiediagram is bereikt. Hiervoor maken we gebruik van $ReturnPoint/3$:

\begin{align}
	&\nonumber \forall{t}[Time]\forall{s}[SDPoint](C\_SDPointAt(Next(t), s) \leftarrow \\ \nonumber &ReturnPoint(t, CurrentStackLevel(t), s) \land (SDPointAt(t, methodOne\_5) \\ &\lor SDPointAt(t, methodTwo\_5))).\label{eq:sd-return}
\end{align}

Deze zin drukt uit dat het terugkeerpunt dat is genoteerd voor deze oproepdiepte wordt genomen als de volgende \textit{SDPoint} wanneer het einde van een sequentiediagram is bereikt, in dit geval $methodOne\_5$ of $methodTwo\_5$. We zetten $finished$ hier niet bij omdat er niets op volgt.

\subsubsection{Causatiezinnen voor \textit{ReturnPoint/3}}
Wanneer de uitvoering een oproep bereikt, willen we voor de nieuwe oproepdiepte dat het terugkeerpunt wordt gezet naar de \textit{SDPoint} direct na de oproepinstructie. Daarmee krijgen we de volgende twee zinnen:

\begin{align}
	\nonumber &\forall{t}[Time]\forall{st}[StackLevel](C\_ReturnPoint(Next(t), st, methodOne\_3post) \\ &\leftarrow (CurrentStackLevel(t) = (st-1)) \land SDPointAt(t, methodOne\_3)). \\
	\nonumber &\forall{t}[Time]\forall{st}[StackLevel](C\_ReturnPoint(Next(t), st, methodTwo\_3post) \\ &\leftarrow (CurrentStackLevel(t) = (st-1)) \land SDPointAt(t, methodTwo\_3)).
\end{align}

We willen ook dat een terugkeerpunt verdwijnt eenmaal dat het wordt gebruikt aan het einde van een diagram. Daarom schrijven we de volgende voorwaarde neer voor het oncausatiepredicaat voor \textit{ReturnPoint/3}:

\begin{align}
 \nonumber &\forall{t}[Time]\forall{st}[StackLevel]\forall{sd}[SDPoint](Cn\_ReturnPoint(Next(t), st, sd) \\ \nonumber &\leftarrow (CurrentStackLevel(t) = st) \land ReturnPoint(t, st, sd) \\ &\land (SDPointAt(t, methodOne\_5) \lor SDPointAt(t, methodTwo\_5))).\label{eq:return-uncauses}
\end{align}

Zinnen \ref{eq:sd-return} en \ref{eq:return-uncauses} samen garanderen dat terugkeerpunten gebruikt worden en verdwijnen wanneer het einde van een diagram is bereikt.

\subsubsection{Causatiezinnen voor \textit{CurrentStackLevel(Time) : StackLevel}}

De oproepdiepte moet toenemen wanneer een oproepinstructie wordt uitgevoerd en afnemen wanneer het einde van een diagram is bereikt. Deze respectievelijke gevallen modelleren we als volgt:

\begin{align}
	\nonumber &\forall{t}[Time]\forall{st}[StackLevel](C\_CurrentStackLevel(Next(t), st) \leftarrow \\ \nonumber &(CurrentStackLevel(t) = (st-1)) \land (SDPointAt(t, methodOne\_3) \\ &\lor SDPointAt(t, methodTwo\_3))). \\
	\nonumber &\forall{t}[Time]\forall{st}[StackLevel](C\_CurrentStackLevel(Next(t), st) \leftarrow \\ \nonumber &(CurrentStackLevel(t) = (st+1)) \land (SDPointAt(t, methodOne\_5) \\ &\lor SDPointAt(t, methodTwo\_5))).
\end{align}

\subsubsection{Causatiezinnen voor oproepobjecten en parameters}
Er komen twee nieuwe soorten variabelen bij: Objecten waarvan een methode wordt opgeroepen en parameters van een methode. In het sequentiediagram voor \textit{methodTwo} is \textit{obj2} de naam van het object dat het eerste bericht ontvangt. Wanneer het diagram voor \textit{methodOne} deze methode oproept, moet \textit{obj2} dus gezet worden naar de juiste waarde, in dit geval \textit{obj} omdat \textit{obj} de methode oproept op zichzelf. Een gelijkaardig geval doet zich voor bij de recursieve oproep in het diagram voor \textit{methodTwo}. Daarom krijgen we de volgende zinnen voor \textit{C\_Obj2T/3}:

\begin{align}
	\nonumber &\forall{t}[Time]\forall{s}[StackLevel]\forall{obj}[A](C\_Obj2T(Next(t), s, obj) \leftarrow
	\\ \nonumber &(CurrentStackLevel(t) = (s-1)) \land SDPointAt(t, methodOne\_3) \\ &\land ObjT(t, (s-1), obj)). \\
	\nonumber &\forall{t}[Time]\forall{s}[StackLevel]\forall{obj}[A](C\_Obj2T(Next(t), s, obj) \leftarrow
	\\ \nonumber &(CurrentStackLevel(t) = (s-1)) \land SDPointAt(t, methodTwo\_3) \\ &\land Obj2T(t, (s-1), obj)).
\end{align}

\textit{mTwoArg} in het diagram voor \textit{methodTwo} is een parameter van \textit{methodTwo} dat ook aangesproken wordt in het diagram zelf. In \textit{methodOne} wordt \textit{mTwoArg} gelijkgesteld aan 1 terwijl in \textit{methodTwo} deze eerst met \'e\'en wordt verhoogd. Daarom krijgen we de drie volgende zinnen:

\begin{align}
	\nonumber &\forall{t}[Time]\forall{s}[StackLevel](C\_MTwoArgT(Next(t), s, 1) \leftarrow \\ &(CurrentStackLevel(t) = (s-1)) \land SDPointAt(t, methodOne\_3)). \\
	\nonumber &\forall{t}[Time]\forall{s}[StackLevel]\forall{n}[int](C\_MTwoArgT(Next(t), s, n) \leftarrow \\ \nonumber &(CurrentStackLevel(t) = (s-1)) \land SDPointAt(t, methodTwo\_3) \\ &\land MTwoArg(t, (s-1), n)). \\
	\nonumber &\forall{t}[Time]\forall{s}[StackLevel]\forall{n}[int](C\_MTwoArgT(Next(t), s, n) \leftarrow \\ \nonumber &(CurrentStackLevel(t) = s) \land SDPointAt(t, methodTwo\_2) \\ &\land (\exists{n1}[int](MTwoArg(t, s, n1) \land (n = n1 + 1)))).\label{eq:mtwoarg-inc}
\end{align}

Zin \ref{eq:mtwoarg-inc} demonstreert dat ook buiten oproepinstructies of een terugkeer uit een diagram wordt gekeken naar het oproepniveau. Er wordt immers enkel gekeken of geschreven naar de waarde van de 'versie' van de variabele die overeenkomt met het huidige oproepniveau. Dit komt ook terug bij de variabelen die niet het oproepobject of een parameter van een methode voorstellen.

\subsubsection{Uitkomst van de beschreven procedure}
Er zijn geen noemenswaardige veranderingen aan hoe we de toestandszinnen opstellen. Bijlage \ref{app:seq-recursion} bevat de uitkomst van de procedure die we hebben beschreven in deze sectie. 

\section{Extra veronderstellingen over het ontwerp van sequentiediagrammen}\label{sec:beperkingen}
Om de implementatie van theoriegeneratie zoals beschreven in dit hoofdstuk te vergemakkelijken, zijn er een aantal veronderstellingen ten aanzien van sequentiediagrammen die de gebruiker als invoer geeft. In deze subsectie sommen we deze op:

\begin{itemize}
	\item Er wordt een logisch symbool voorzien voor elke variabele in een sequentiediagram. De naam van dat logisch symbool is gebaseerd op de naam van de variabele, en daarom moeten alle variabelen over alle diagrammen heen een unieke naam hebben. De naam van het diagram waarin de variabele voorkomt kan dienen als prefix om de naam van de variabele toch uniek te maken zonder dat tussenkomst van de ontwerper nodig is. Door tijdsgebrek is dit echter niet ge\"implementeerd.
	\item Het type van een variabele kan niet veranderen over instructies heen. Ofwel komt een variabele overeen met een levenslijn en wordt het type dus bepaald door de levenslijn, ofwel wordt het type bepaald door de instructie die de variabele instantieert.
	\item Een variabele kan geen verzameling voorstellen.
	\item Per instructie kan maar \'e\'en associatielink tegelijkertijd genavigeerd worden. Beschouw het voorbeelddiagram van figuur \ref{fig:diagram-voorbeeld}. Als variabele \textit{x} een object van klasse \textit{Item} is, dan zal een oproep van \textit{getInventory()} op \textit{x} een object van klasse \textit{Inventory} opleveren (mocht die associatie zijn ingevuld voor \textit{x}), maar is \textit{getInventory().getCharacter()} geen geldige instructie.
	\item Er mogen geen aanpassingen worden aangebracht aan de associatiestructuur voor een model van een klassediagram. Voor een \textit{Character}---\textit{Inventory}-paar geldt bijvoorbeeld dat die verbinding niet verbroken kan worden en dat geen van beide kanten vervangen mag worden door een andere instantie van de respectievelijke klasse. Aangezien associaties van meervoudige multipliciteit als lijsten worden ge\"implementeerd, betekent dit ook dat men geen elementen kan toevoegen aan of verwijderen uit een lijst.
	\item Men kan geen nieuwe objecten instanti\"eren.
	\item Er zijn geen bewerkingen op booleaanse waarden behalve de NOT-functie, die men kan oproepen als \textit{flipBool(boolean)}.
	\item UML legt op dat men voor beide takken van een alt-fragment de voorwaarde voor de uitvoering die tak moet specificeren. Ook hier wordt verwacht dat de ontwerper dit telkens doet.
	\item Als een diagram een uitvoer heeft, dan mag de instructie die bepaalt welke variabele als uitvoer wordt gebruikt geen onderdeel zijn van een gecombineerd fragment.
	\item In een fragmentvoorwaarde of een oproep gebeurt er voor de gebruikte waardes geen evaluatie tenzij om de waarde van een variabele op te halen. \textit{call(1 + 2)} en \textit{call(var1 + var2)} worden bijvoorbeeld niet correct vertaald, maar \textit{call(3)} en \textit{call(var1)} wel.
\end{itemize}

\section{Extra soorten instructies voor sequentiediagrammen}\label{sec:newlang}
UML is een modelleertaal die wordt gebruikt ter ondersteuning van softwareontwerp in imperatieve programmeertalen. In sequentiediagrammen vertaalt zich dat tot een veelvoud aan instructies voor eenvoudige taken zoals het selecteren van het eerste getal verschillend van 0 in een lijst van getallen. Aangezien elke nieuwe instructie in een sequentiediagram leidt tot minstens \'e\'en zin in de gegenereerde theorie, impliceert dit een grote kost in zowel rekentijd als geheugengebruik voor modelexpansie en progressie\"inferentie. Daarom geven we in deze sectie een aanzet tot het integreren van logica in het ontwerp van sequentiediagrammen met het zicht op twee doelen:

\begin{enumerate}
	\item Het totale aantal instructies over alle sequentiediagrammen verminderen
	\item Het aantal tijdstappen die nodig zijn bij modelexpansie of progressie\"inferentie om een bepaalde taak te verrichten verminderen
\end{enumerate}

We houden vast aan de structuur van klasses, associaties, levenslijnen en berichten. Wat volgt is een overzicht van nieuwe soorten instructies die gebruikt kunnen worden in berichten.

\subsection{Nieuwe soorten instructies}

Voorheen kon een variabele maar \'e\'en object of waarde van een primitief type voorstellen. Nu laten we toe dat een variabele een verzameling kan voorstellen. In een sequentiediagram wordt dit voorgesteld als een multi-object zoals in figuur \ref{fig:multi-object}. In de theorie behouden we voor namen het patroon $<variabelenaam>T/3$.

We laten op variabelen die een verzameling voorstellen enkele nieuwe instructies toe:

\begin{figure}
	\includegraphics{chap-gedrag/seq-multi-object.png}
	\centering
	\caption{Een multi-object}
	\label{fig:multi-object}
\end{figure}

\begin{itemize}
	\item Voor alle leden van een verzameling tegelijkertijd kan een associatie genavigeerd worden. Als op een verzameling S van objecten van klasse X \textit{getY()} wordt uitgevoerd waar Y de naam is van een klasse geassocieerd met X, dan is het resultaat een verzameling van objecten van klasse Y. Die verzameling bestaat uit alle objecten die in verband staan met minstens \'e\'en object uit S.
	\item $getNumX()$ waar X de naam is van een klasse die in verband staat met de klasse van de verzameling. Het resultaat is de som van het aantal instanties van klasse X waarmee elk lid van de verzameling in verband staat.
	\item $EXISTS\ ONE\ WHERE\ [query]$: Geeft \textbf{true} als en slechts als \textit{query} geldt voor minstens \'e\'en lid van de verzameling. \textit{query} betekent hier een instructie die bestaat uit getters van klassevariabelen verbonden met booleaanse connectieven (klassevariabelen kunnen hierbij rechtstreeks vergeleken worden met een getal of string, waar toepasselijk).
	\item $EXISTS\ n\ TO\ m\ WHERE\ [query]$ waar $n \leq m$: Geeft \textbf{true} als en slechts als voor minstens $n$ en ten hoogste $m$ leden van de verzameling geldt dat \textit{query} waar is.
	\item $FOR\ ALL\ APPLIES\ [query]$: Geeft \textbf{true} als en slechts als voor alle leden van de verzameling \textit{query} geldt.
	\item $NOT\ [query]$: geeft \textbf{true} als en slechts als \textit{query} \textbf{false} geeft. \textit{query} kan hier \'e\'en van de voorgaande soorten instructies zijn.
	\item $CHOOSE\ ALL\ WHERE\ APPLIES\ [query]$: Geeft een nieuwe verzameling die bestaat uit alle leden van de originele verzameling waarvoor \textit{query} waar is.
	\item $CHOOSE\ ONE\ WHERE\ APPLIES\ [query]$: Geeft \'e\'en object uit de verzameling waarvoor \textit{query} waar is. Dit vormt een keuzepunt bij simulatie.
	\item $CHOOSE\ n\ TO\ m\ WHERE\ APPLIES\ [query]$ waar $n \leq m$: Geeft een nieuwe verzameling die bestaat uit minstens $n$ en ten hoogste $m$ leden uit de originele verzameling waarvoor \textit{query} waar is. Dit vormt een keuzepunt bij simulatie.
	\item $CHOOSE\ <getalnaam>\ WHERE\ APPLIES\ [query\ dat\ <getalnaam>\ bindt]$: Geeft een nieuwe variabele met als naam \textit{getalnaam} waarvoor geldt dat de waarde voldoet aan \textit{query}. Dit soort instructie kan ook toegepast worden op een variabele dat maar \'e\'en object voorstelt. Dit vormt een keuzepunt bij simulatie.
	\item $SET <klassevariabele> TO <waarde>$: Voor alle leden van de verzameling wordt de waarde van de genoemde klassevariabele veranderd naar de genoemde waarde.
\end{itemize}

We illustreren het gebruik van deze nieuwe instructies door middel van het voorbeeld van figuur \ref{fig:new-nim}, dat een modellering van het bekende spel Nim voorstelt. De beurt gaat eerst aan de gekozen speler. Dan wordt gecontroleerd of alle stapels leeg zijn. Als dat niet het geval is, kiest de speler eerst een stapel en neemt dan minstens \'e\'en object weg van die stapel. Daarna wordt de beurt gegeven aan de volgende speler. Als alle stapels leeg zijn, is de huidige speler de winnaar. Dit stelt dus een versie van Nim voor waar de speler die als laatste een object wegneemt verliest.

\begin{figure}[H]
	\includegraphics[width=\textwidth]{chap-gedrag/seq-new-nim.png}
	\caption{Een modellering van het spel Nim}
	\label{fig:new-nim}
\end{figure}

De lusvoorwaarde wordt als volgt gebruikt:

\begin{align}
	\nonumber &\forall{t}[Time]\forall{st}[StackLevel](C\_SDPointAt(Next(t), play\_3) \leftarrow \\ \nonumber &(CurrentStackLevel(t) = st) \land (SDPointAt(t, play\_2) \lor SDPointAt(t, play\_6)) \land \\ \nonumber &\lnot{}\exists{g}[Game](GameT(t, st, g) \land{}\forall{h}[Heap](GameandHeap(g, h) \\ &\Rightarrow \exists{n}[LimitedInt](HeapamountObjects(t, h, n) \land n = 0)))). \\
	\nonumber &\forall{t}[Time]\forall{st}[StackLevel](C\_SDPointAt(Next(t), play\_7) \leftarrow \\ \nonumber &(CurrentStackLevel(t) = st) \land SDPointAt(t, play\_6) \land \exists{g}[Game](GameT(t, st, g) \land \\ \nonumber &\forall{h}[Heap](GameandHeap(g, h) \\ &\Rightarrow \exists{n}[LimitedInt](HeapamountObjects(t, h, n) \land n = 0)))).
\end{align}

In instructie 3 wordt de $CHOOSE\ ONE\ WHERE\ APPLIES\ [query]$ instructie gebruikt. We introduceren hier een open predicaat, $ChosenHeap(Time, Heap)$, om ons te helpen deze instructie te vertalen. We voegen de volgende zin toe aan het definitieblok voor diagramvariabelen:

\begin{align}
	\nonumber &\forall{t}[Time]\forall{st}[StackLevel]\forall{h}[Heap](C\_HeapT(Next(t), st, h) \\ &\leftarrow (CurrentStackLevel(t) = st) \land SDPointAt(t, play\_3) \land ChosenHeap(t, h)).
\end{align}

Om ervoor te zorgen dat exact \'e\'en Heap-object wordt gekozen, moeten we voorwaardes leggen op $ChosenHeap/2$. Dit doen we als volgt:

\begin{align}
	&\forall{t}[Time](SDPointAt(t, play\_3) \Rightarrow \exists_{=1}{h}[Heap](ChosenHeap(t, h))).\label{form:uniqueheap} \\
	\nonumber &\forall{t}[Time]\forall{h}[Heap](ChosenHeap(t, h) \Rightarrow (SDPointAt(t, play\_3) \\ \nonumber &\land \exists{g}[Game]\exists{st}[StackLevel]\exists{n}[LimitedInt]((CurrentStackLevel(t) = st) \\ &\land GameT(t, st, g) \land GameandHeap(g, h) \land HeapamountObjects(t, h, n) \land \lnot(n = 0)))).\label{form:correctsd}
\end{align}

Zin \ref{form:uniqueheap} garandeert dat er maar \'e\'en \textit{Heap}-object mag gekozen worden.

Zin \ref{form:correctsd} garandeert dat er enkel een keuze mag gemaakt worden als de uitvoering instructie 3 bereikt en dat er enkel een \textit{Heap}-object wordt gekozen waarvoor geldt dat \textit{amountObjects} groter is dan 0.

Voor instructie 4 introduceren we op een soortgelijke manier een open predicaat $ChosenTake/2$. De zin die we toevoegen aan het definitieblok voor variabelen en de voorwaarden die we opleggen op $ChosenTake/2$ zien er gelijkaardig uit.

Bijlage \ref{code:new-nim} bevat de vertaling van het diagram in figuur \ref{fig:new-nim}.
	\chapter{Evaluatie}\label{sec:evaluatie}

In dit hoofdstuk ontwerpen we een klassediagram en sequentiediagrammen waarmee we het spel Nim modelleren zonder gebruik te maken van de nieuwe instructies uit sectie \ref{sec:newlang}. We beoordelen de volgende zaken:

\begin{itemize}
	\item Hoe gemakkelijk het is om tot een ontwerp te komen.
	\item Of we het spel kunnen spelen met de uitvoertheorie.
	\item Of we bepaalde gewenste eigenschappen kunnen verifi\"eren.
	\item Performantie: tijd- en geheugengebruik, deze laatste in de vorm van de grootte van de \textit{grounding}\cite{DeCatBroes2014PLaa}. We zetten in onze IDP-bestanden \textit{stdoptions.verbosity.groundingstats} op 1 om IDP deze statistieken zelf te laten rapporteren.
\end{itemize}

\section{Ontwerp}

Figuur \ref{fig:nim-cd} bevat het klassediagram voor ons ontwerp van Nim.

De klasse \textit{Game} stelt het concept van een spel voor. \textit{p1Win} geeft aan of speler 1 heeft gewonnen en \textit{gameFinished} geeft aan of het spel volledig gespeeld is. \textit{allHeapsEmpty} vraagt op of alle \textit{Heap}s leeg zijn. \textit{takeTurn} doet een speler zijn beurt spelen. \textit{play(boolean)} is de methode die het verloop van het spel regelt.

De klasse \textit{Heap} stelt stapels van objecten voor. \textit{numObjects} is het aantal objecten dat een stapel bevat. \textit{isEmpty} geeft aan of de stapel leeg is en \textit{take(int)} neemt een aantal objecten weg van een stapel.

Figuren \ref{fig:nim-play}, \ref{fig:nim-allHeapsEmpty}, \ref{fig:nim-isEmpty}, \ref{fig:nim-takeTurn} en \ref{fig:nim-take} bevatten de sequentiediagrammen die het gedrag van hun bijhorende methodes modelleren.

Diagram \ref{fig:nim-play} bepaalt eerst of alle stapels leeg zijn. Zo ja, is het spel voorbij en is de speler die nu aan beurt is de winnaar. Zo nee, speelt de huidige speler een ronde en geeft de beurt door aan de andere speler.

Diagram \ref{fig:nim-allHeapsEmpty} vraagt aan alle stapels om de beurt of ze leeg zijn. Als het een stapel tegenkomt die niet leeg is, is het antwoord meteen nee. Als het enkel stapels tegenkomt die leeg zijn, is het antwoord ja.

Diagram \ref{fig:nim-isEmpty} antwoordt ja als de stapel leeg is en nee als de stapel niet leeg is.

Diagram \ref{fig:nim-takeTurn} vraagt aan de speler van welke stapel hij objecten wil nemen. Er wordt vanaf nul geteld, dus moet hij nul antwoorden als hij de eerste stapel wil selecteren. Als de geselecteerde stapel leeg is, wordt er opnieuw gevraagd om een stapel te selecteren. Als de speler een stapel heeft gekozen die niet leeg is, moet hij kiezen om minstens \'e\'en en maximaal het aantal objecten dat nog overblijft op de stapel weg te nemen.

Diagram \ref{fig:nim-take} berekent hoeveel objecten er nog overblijven op een stapel gegeven het aantal objecten. Als het verschil kleiner is dan nul, wordt de teller op nul gezet.

Appendix \ref{code:nim-eval} bevat het IDP-bestand met de automatisch gegenereerde uitvoertheorie voor deze diagrammen.

\begin{figure}[h]
	\centering
	\includegraphics[width=0.25\textwidth]{chap-evaluatie/ClassDiagram1.png}
	\caption{Klassediagram voor Nim}
	\label{fig:nim-cd}
\end{figure}

%\begin{landscape}
%	\newpage
%	\thispagestyle{plain}
%	\begin{figure}-
%		\centering
%		\includegraphics[width=0.4\textwidth]{chap-evaluatie/play.png}
%		\caption{Sequentiediagram voor play()}
%		\label{fig:nim-play}
%	\end{figure}
%	\begin{figure}
%		\includegraphics[width=0.75\textwidth]{chap-evaluatie/allHeapsEmpty.png}
%		\caption{Sequentiediagram voor allHeapsEmpty()}
%		\label{fig:nim-allHeapsEMpty}
%	\end{figure}
%\end{landscape}
\begin{landscape}
	\newpage
	\thispagestyle{empty}
	\begin{figure}
		\vspace*{-2cm}
		\begin{subfigure}{\textwidth}
			\includegraphics[width=0.9\textwidth]{chap-evaluatie/play.png}
			\caption{Sequentiediagram voor \textit{play(boolean)}}
			\label{fig:nim-play}
		\end{subfigure}%
		\begin{subfigure}{\textwidth}
			\includegraphics[width=0.8\textwidth]{chap-evaluatie/allHeapsEmpty.png}
			\caption{Sequentiediagram voor \textit{allHeapsEmpty()}}
			\label{fig:nim-allHeapsEmpty}
		\end{subfigure}
		\caption{Sequentiediagrammen voor \textit{play(boolean)} en \textit{allHeapsEmpty()}}
		\label{fig:nim-play-ahe}
	\end{figure}
\end{landscape}
\begin{landscape}
	\newpage
	\thispagestyle{empty}
	
	\begin{figure}
		\begin{subfigure}{\textwidth}
			\includegraphics[width=0.75\textwidth]{chap-evaluatie/isEmpty.png}
			\caption{Sequentiediagram voor \textit{isEmpty()}}
			\label{fig:nim-isEmpty}
		\end{subfigure}%
		\begin{subfigure}{\textwidth}
			\includegraphics[width=0.8\textwidth]{chap-evaluatie/takeTurn.png}
			\caption{Sequentiediagram voor \textit{takeTurn()}}
			\label{fig:nim-takeTurn}
		\end{subfigure}
		\caption{Sequentiediagrammen voor \textit{isEmpty()} en \textit{takeTurn()}}
		\label{fig:nim-isempty-tt}
	\end{figure}
\end{landscape}

\begin{figure}[h]
	\includegraphics[width=0.8\textwidth]{chap-evaluatie/take.png}
	\caption{Sequentiediagram voor \textit{take(int)}}
	\label{fig:nim-take}
\end{figure}

\subsection{Evaluatie van het ontwerpproces}

Door de beperkingen opgesomd in sectie \ref{sec:beperkingen} zijn er een aantal onintu\"itieve elementen. Als het mogelijk was geweest om de associaties van een object aan te passen tijdens de uitvoering, zou er een klasse \textit{Player} zijn geweest en een associatie \textit{Game}---\textit{Player} die de winnaar aanduidt. Aangezien er maar twee spelers zijn, is het echter voldoende om de winnaar voor te stellen met een booleaanse waarde. Als \textit{gameFinished} waar is, kan toch de winnaar ondubbelzinnig aangeduid worden.

Nim is een simpel spel, dus waren er geen noemenswaardige moeilijkheden bij het ontwerp van de diagrammen. Omdat men maar relatief simpele taken kan doen met \'e\'en enkele instructie, leidt dit er wel toe dat er een veelvoud aan instructies nodig zijn, wat de tijd die nodig is om het ontwerp te maken verhoogt.

\section{Evaluatie van de uitvoertheorie}

Alle testen worden uitgevoerd op een machine die Ubuntu 16.04.3 LTS draait. Deze machine heeft als CPU de Intel(R) Core(TM)2 CPU 6600 en heeft 3969 MB aan geheugen beschikbaar. We gebruiken versie 3.7.0 van IDP.

De volgende testen worden uitgevoerd:

\begin{enumerate}
	\item Simulatie van het verloop van een spel met twee stapels, waarvan \'e\'en twee objecten heeft en de andere drie, door middel van progressie\"inferentie.
	\item Verificatie van de correctheid van het diagram voor \textit{isEmpty}
	\item Verificatie van de correctheid van het diagram voor \textit{allHeapsEmpty} 
\end{enumerate}

\subsection{Evaluatie van de simulatie}

In deze simulatie neemt de eerste speler tijdens zijn beurt alle drie de objecten van de stapel met drie objecten. Vervolgens neemt de tweede speler \'e\'en object van de andere stapel. Nu heeft de eerste speler geen keuze en moet hij het overblijvend object nemen, waardoor hij het spel verliest.

Figuur \ref{fig:ms} bevat een grafiek van de cumulatieve rekentijd per simulatiestap. We zien dat de rekentijd een lineair verloop kent. Het duurt echter ongeveer achttien minuten om het spel volledig uit te spelen. Stap 23, 67 en 111 waren de stappen wanneer de speler een stapel kon kiezen. Het duurde ongeveer twee en een halve minuut om stap 23 te bereiken. Tussen stap 23 en 67 en tussen stap 67 en 111 was er telkens een tijdspanne van ongeveer vijf minuten.

\begin{figure}
	\includegraphics[width=1.05\textwidth]{chap-evaluatie/ms.png}
	\caption{Cumulatieve rekentijd per stap}
	\label{fig:ms}
\end{figure}

Figuur \ref{fig:groundingsize} bevat een grafiek van de grootte van de \textit{grounding} per simulatiestap. De grootte is voor de meeste stappen gelijk aan \'e\'en of drie, met zes gevallen die daarvan afwijken.

\begin{figure}
	\includegraphics[width=1.05\textwidth]{chap-evaluatie/groundingsize.png}
	\caption{Grootte van de \textit{grounding} per stap}
	\label{fig:groundingsize}
\end{figure}

Uit deze resultaten blijkt dat we met de uitvoertheorie het spel kunnen spelen, maar het tijdsinterval tussen beurten lang is. De grootte van de \textit{grounding} blijft beperkt.

\section{Verificatie van de correctheid van \textit{isEmpty}}

We willen controleren of het diagram in figuur \ref{fig:nim-isEmpty} een positief antwoord geeft als en slechts als de gegeven stapel leeg is. We werken met een invoerstructuur die stelt dat de uitvoering begint met de eerste instructie van \textit{isEmpty} en eindigt na de laatste instructie van hetzelfde diagram. Om de correctheid te controleren we vier zinnen op:

\begin{align}
	\nonumber&\exists{t}[Time]\exists{st}[StackLevel]\exists{h}[Heap]\exists[n][LimitedInt](SDPointAt(t, finished) \\ &\land IeToReturnT(t, st, F) \land HeapamountObjects(t, h, n) \land n \leq 0).\label{form:ie_fnonempty} \\
	\nonumber&\exists{t}[Time]\exists{st}[StackLevel]\exists{h}[Heap]\exists[n][LimitedInt](SDPointAt(t, finished) \\ &\land IeToReturnT(t, st, T) \land HeapamountObjects(t, h, n) \land n > 0).\label{form:ie_fempty} \\
	\nonumber&\exists{t}[Time]\exists{st}[StackLevel]\exists{h}[Heap]\exists[n][LimitedInt](SDPointAt(t, finished) \\ &\land IeToReturnT(t, st, T) \land HeapamountObjects(t, h, n) \land n \leq 0).\label{form:ie_cempty} \\
	\nonumber&\exists{t}[Time]\exists{st}[StackLevel]\exists{h}[Heap]\exists[n][LimitedInt](SDPointAt(t, finished) \\ &\land IeToReturnT(t, st, F) \land HeapamountObjects(t, h, n) \land n > 0).\label{form:ie_cnonempty}
\end{align}

Zin \ref{form:ie_fnonempty} schrijft voor dat het antwoord negatief is, maar dat de stapel nul of minder objecten heeft.

Zin \ref{form:ie_fempty} is gelijkaardig, maar controleert op een positief antwoord met een stapel met meer dan nul objecten.

Beide zinnen moeten leiden tot een onvervulbare theorie.

Zin \ref{form:ie_cempty} schrijft voor dat het antwoord positief is en dat de stapel nul of minder objecten heeft.

Zin \ref{form:ie_cnonempty} controleert op een negatief antwoord met een stapel met meer dan nul objecten.

Beide zinnen moeten leiden tot een vervulbare theorie.

We voeren IDP viermaal uit met de voorgaande zinnen om de beurt toegevoegd aan de invoertheorie. Voor de twee eerste zinnen krijgen we inderdaad als antwoord dat de theorie onvervulbaar is. Voor zin \ref{form:ie_cempty} krijgen we een uniek model dat overeenkomt met een lege stapel. Voor zin \ref{form:ie_cnonempty} krijgen we alle modellen met een stapel die niet leeg is.

Er is een gelijkaardig tijdsverloop en \textit{grounding}-grootte voor alle vier de uitvoeringen. Het duurt ongeveer 21,5 seconden en de \textit{grounding}-grootte is 578 306.

We merken op dat het relatief lang duurt om modelexpansie uit te voeren voor dit diagram. De \textit{grounding} is ook omvangrijk.

\section{Verificatie van de correctheid van \textit{allHeapsEmpty}}

We willen controleren of het diagram in figuur \ref{fig:nim-allHeapsEmpty} een positief antwoord geeft als alle stapels leeg zijn en een negatief antwoord als minstens \'e\'en stapel niet leeg is. We werken met een invoerstructuur die aangeeft dat de uitvoering begint met de eerste instructie van \textit{allHeapsEmpty} en eindigt na de laatste instructie van dat diagram. We stellen vier zinnen op:

\begin{align}
	\nonumber&\exists{t}[Time]\exists{st}[StackLevel](SDPointAt(22, allHeapsEmpty\_8) \\ &\land \lnot\exists{h}[Heap]\exists{n}[LimitedInt](HeapamountObjects(t, h, n) \land n > 0)).\label{form:ahe_fnonempty} \\
	\nonumber&\exists{t}[Time]\exists{st}[StackLevel]\exists{h}[Heap]\exists[n][LimitedInt] \\ &(SDPointAt(22, allHeapsEmpty\_7) \land HeapamountObjects(t, h, n) \land n > 0).\label{form:ahe_fempty} \\
	\nonumber&\exists{t}[Time]\exists{st}[StackLevel](SDPointAt(22, allHeapsEmpty\_7) \\ &\land \forall{h}[Heap]\forall{n}[LimitedInt](HeapamountObjects(t, h, n) \Rightarrow n \leq 0)).\label{form:ahe_cempty} \\
	\nonumber&\exists{t}[Time]\exists{st}[StackLevel]\exists{h}[Heap]\exists[n][LimitedInt] \\ &(SDPointAt(22, allHeapsEmpty\_8) \land HeapamountObjects(t, h, n) \land n > 0).\label{form:ahe_cnonempty}
\end{align}

We benoemen expliciet tijdstap 22 omdat de machine net voldoende geheugen heeft om modelexpansie uit te voeren met dat aantal tijdstappen. 22 tijdstappen zijn voldoende om gegeven twee stapels een punt te bereiken in het diagram waarop het antwoord eigenlijk al bekend is. Instructie 7 komt overeen met het geval waar het antwoord positief is; instructie 8 komt overeen met het geval waar het antwoord negatief is.

Zin \ref{form:ahe_fnonempty} stelt voor dat het antwoord negatief is, maar dat er geen stapel bestaat met meer dan nul objecten.

Zin \ref{form:ahe_fempty} stelt voor dat het antwoord positief is, maar dat er een stapel bestaat met meer dan nul objecten.

Deze twee zinnen moeten leiden tot de conclusie dat de theorie onvervulbaar is.

Zin \ref{form:ahe_cempty} stelt voor dat het antwoord positief is en dat alle stapels leeg zijn.

Zin \ref{form:ahe_cnonempty} stelt voor dat het antwoord negatief is en dat er een stapel bestaat met meer dan nul objecten.

De conclusie moet hier zijn dat de theorie vervulbaar is.

Wanneer we IDP viermaal runnen met deze vier zinnen om de beurt toegevoegd aan de uitvoertheorie, is de theorie inderdaad onvervulbaar voor de eerste twee zinnen. Zin \ref{form:ahe_cempty} leidt tot een uniek model dat overeenkomt met het geval dat alle stapels leeg zijn. Zin \ref{form:ahe_cnonempty} geeft alle modellen waar minstens \'e\'en stapel meer dan nul objecten heeft.

De uitvoeringstijd en \textit{grounding}-grootte zijn gelijkaardig voor alle vier uitvoeringen. De uitvoering duurt ongeveer 24 seconden en de \textit{grounding}-grootte is 2 249 098.

\subsection{Samenvatting van de bevindingen}
Men kan Nim spelen met de theorie die automatisch werd gegenereerd uit de gegeven diagrammen. Tussen elke beurt is er wel een wachttijd van ongeveer vijf minuten, met twee en een halve minuut voordat de eerste beurt gespeeld kan worden.

Het is mogelijk om voor een diagram bepaalde gewenste eigenschappen te controleren. Dit gaat echter gepaard met grote \textit{groundings}.

Met de machine die ons ter beschikking stond konden we niet controleren op welke manieren het spel kan verlopen die ertoe leiden dat een bepaalde speler wint beginnend van een bepaald scenario omwille van een tekort aan geheugen.
	\chapter{Het integreren van logica in de ontwerptaal voor sequentiediagrammen}\label{sec:decl-seq}
UML is een modelleertaal die wordt gebruikt ter ondersteuning van softwareontwerp in imperatieve programmeertalen. In sequentiediagrammen vertaalt zich dat tot een veelvoud aan instructies voor eenvoudige taken zoals het selecteren van het eerste getal verschillend van 0 in een lijst van getallen. Aangezien elke nieuwe instructie in een sequentiediagram leidt tot minstens \'e\'en zin in de gegenereerde theorie, impliceert dit een grote kost in zowel rekentijd als geheugengebruik voor modelexpansie en progressie\"inferentie, zoals we hebben aangetoond in hoofdstuk \ref{sec:evaluatie}. Daarom geven we in dit hoofdstuk een aanzet tot het integreren van logica in het ontwerp van sequentiediagrammen.

\section{Vergelijking tussen lineaire tijdcalculus en onze voorstellingsmethode}

We modelleren eerst Nim rechtstreeks in lineaire tijdcalculus en vergelijken de resulterende theorie met de theorie gegenereerd volgens de regels in hoofdstuk \ref{sec:gedrag}. Het volgende codefragment bevat een LTC-theorie die Nim modelleert. 

\lstinputlisting[caption=Nim gemodelleerd in lineaire tijdcalculus]{chap-declaratieve-seq/ltc-nim.idp}\label{code:ltc-nim}

Bij het opstellen van deze theorie werden de volgende stappen gevolgd:

\begin{enumerate}
	\item Bepaal de variabelen die de toestand van het systeem beschrijven. Zo krijgen we de logische types \textit{Heap}, die stapels voorstellen; \textit{Size}, wat het aantal objecten op een stapel voorstelt; en \textit{Player}, voorgesteld door exact twee getallen voor de twee spelers. We onderscheiden ook de functies \textit{NumObj(Time, Heap) : Size}, wat de grootte van een stapel op een bepaald tijdstip voorstelt; en \textit{PlayerTurn(Time) : Player}, die voor elk tijdstip aangeeft welke speler aan beurt is. Deze functies maken we inertieel.
	\item Bepaal welke bewerkingen er kunnen gebeuren op het systeem op elk tijdstip die er mogelijks voor zorgen dat de toestand van het systeem verandert in toekomstige tijdstappen. Definieer de \textit{constructed type Action} die deze bewerkingen benoemt. Voor Nim is er maar \'e\'en actie mogelijk: het nemen van een aantal objecten van een bepaalde stapel. Daarom bestaat \textit{Action} uit domeinobjecten gegenereerd uit de specificatie \textit{Take(Heap, Size)}, dat voor alle mogelijke combinaties van \textit{Heap} en \textit{Size} een domeinobject genereert.
	\item Leg geschikte voorwaarden op de uitvoering van de bewerkingen bepaald in de vorige stap. In Nim moeten we ervoor zorgen dat een speler objecten neemt van een stapel die niet leeg is en dat een speler minstens \'e\'en object wegneemt en niet meer objecten dan aanwezig op de stapel. Verder dwingen we af dat een speler een niet-lege stapel kiest.
	\item Schrijf de causatiezinnen. Specificeer hoe een bewerking de toestand van het systeem verandert. We leiden af dat \textit{NumObj} verandert ten gevolge van een \textit{Take} actie. Verder zorgen we ervoor dat er elke tijdstap een beurtwissel gebeurt.
	\item Schrijf de toestandzinnen. Deze hebben een vast patroon voor elk inertieel logisch symbool.
\end{enumerate}

Het meest opmerkelijk verschil tussen deze logische specificatie en de specificatie in bijlage \ref{code:nim-eval} is dat zowel het vocabularium en de theorie significant compacter zijn. De combinatie van kleinere zoekruimte en minder zinnen om rekening mee te houden zorgt ervoor dat zowel rekentijd als geheugengebruik sterk verkleinen. Inderdaad, voor een spel met een stapel met twee objecten en een stapel met drie objecten duurt het iets meer dan twee seconden om alle 37 mogelijke spelverlopen te vinden. Daarbij is de \textit{grounding}-grootte 3855. Vergelijk met de verificatie van \textit{allHeapsEmpty} in hoofdstuk \ref{sec:evaluatie}, waar de rekentijd 31,6 seconden was en de \textit{grounding}-grootte 142 510.

We merken op dat een belangrijk nadeel dat sequentiediagrammen hebben ten opzichte van rechtstreekse LTC is dat de toestand van maar \'e\'en variabele kan opgevraagd of aangepast worden. De LTC-theorie heeft bijvoorbeeld \'e\'en zin nodig om te verzekeren dat het spel voorbij is indien alle stapels leeg zijn. Er is echter een heel diagram nodig om voor elke stapel om de beurt te controleren of die leeg is.

Verder kan de LTC-theorie ook meteen alle lege stapels uitsluiten wanneer de speler moet kiezen van welke stapel hij objecten neemt. In figuur \ref{fig:nim-takeTurn} moet eerst gecontroleerd worden of de stapel die de speler kiest effectief leeg is. De aanwezigheid van lussen die een berekening implementeren in plaats van dat ze iteratief proces in het probleemdomein voorstellen zorgt er dus voor dat modelexpansie voor de resulterende theorie minder performant is. Dit leidt namelijk tot logische symbolen die de variabelen betrokken bij de berekening voorstellen en zinnen in de theorie die overeenkomen met instructies op die variabelen. In de context van Nim is de beurtenwissel wanneer nog niet alle stapels leeg zijn een voorbeeld van een iteratief proces binnen het probleemdomein. In het ontwerp gepresenteerd in sectie \ref{sec:nim-design} zijn er echter meerdere lussen aanwezig die geen iteratief proces in het probleemdomein voorstellen:

\begin{itemize}
	\item De lus in diagram \ref{fig:nim-allHeapsEmpty} die voor elke stapel nagaat of hij leeg is.
	\item De lus in diagram \ref{fig:nim-takeTurn} die de speler een stapel laat selecteren tot hij een niet-lege stapel kiest.
\end{itemize}

Deze observaties leiden ertoe dat we nieuwe soorten instructies voor sequentiediagrammen beschikbaar willen stellen met het oog op deze doelen:

\begin{enumerate}
	\item Het totale aantal instructies over alle sequentiediagrammen verminderen
	\item Het aantal LTC-tijdstappen die nodig zijn bij modelexpansie of progressie\"inferentie om een bepaalde taak te verrichten verminderen
\end{enumerate}

Concreet willen we instructies die ons toelaten om de toestand van alle objecten in een verzameling op te vragen, te controleren of te wijzigen. Verder willen we ook instructies die bij simulatie toelaten om te kiezen uit bepaalde leden uit een verzameling op basis van een criterium op de toestand van de leden.

\section{Nieuwe soorten instructies}

Voorheen kon een variabele maar \'e\'en object of waarde van een primitief type voorstellen. Nu laten we toe dat een variabele een verzameling kan voorstellen. In een sequentiediagram wordt dit voorgesteld als een multi-object zoals in figuur \ref{fig:multi-object}. In de theorie behouden we voor namen het patroon $<variabelenaam>T/3$.

We laten op variabelen die een verzameling voorstellen enkele nieuwe instructies toe:

\begin{figure}
	\includegraphics{chap-gedrag/seq-multi-object.png}
	\centering
	\caption{Een multi-object}
	\label{fig:multi-object}
\end{figure}

\begin{itemize}
	\item Voor alle leden van een verzameling tegelijkertijd kan een associatie genavigeerd worden. Als op een verzameling S van objecten van klasse X \textit{getY()} wordt uitgevoerd waar Y de naam is van een klasse geassocieerd met X, dan is het resultaat een verzameling van objecten van klasse Y. Die verzameling bestaat uit alle objecten die in verband staan met minstens \'e\'en object uit S.
	\item $getNumX()$ waar X de naam is van een klasse die in verband staat met de klasse van de verzameling. Het resultaat is de som van het aantal instanties van klasse X waarmee elk lid van de verzameling in verband staat.
	\item $EXISTS\ ONE\ WHERE\ [query]$: Geeft \textbf{true} als en slechts als \textit{query} geldt voor minstens \'e\'en lid van de verzameling. \textit{query} betekent hier een instructie die bestaat uit getters van klassevariabelen verbonden met booleaanse connectieven (klassevariabelen kunnen hierbij rechtstreeks vergeleken worden met een getal of string, waar toepasselijk).
	\item $EXISTS\ n\ TO\ m\ WHERE\ [query]$ waar $n \leq m$: Geeft \textbf{true} als en slechts als voor minstens $n$ en ten hoogste $m$ leden van de verzameling geldt dat \textit{query} waar is.
	\item $FOR\ ALL\ APPLIES\ [query]$: Geeft \textbf{true} als en slechts als voor alle leden van de verzameling \textit{query} geldt.
	\item $NOT\ [query]$: geeft \textbf{true} als en slechts als \textit{query} \textbf{false} geeft. \textit{query} kan hier \'e\'en van de voorgaande soorten instructies zijn.
	\item $CHOOSE\ ALL\ WHERE\ APPLIES\ [query]$: Geeft een nieuwe verzameling die bestaat uit alle leden van de originele verzameling waarvoor \textit{query} waar is.
	\item $CHOOSE\ ONE\ WHERE\ APPLIES\ [query]$: Geeft \'e\'en object uit de verzameling waarvoor \textit{query} waar is. Dit vormt een keuzepunt bij simulatie.
	\item $CHOOSE\ n\ TO\ m\ WHERE\ APPLIES\ [query]$ waar $n \leq m$: Geeft een nieuwe verzameling die bestaat uit minstens $n$ en ten hoogste $m$ leden uit de originele verzameling waarvoor \textit{query} waar is. Dit vormt een keuzepunt bij simulatie.
	\item $CHOOSE\ <getalnaam>\ WHERE\ APPLIES\ [query\ dat\ <getalnaam>\ bindt]$: Geeft een nieuwe variabele met als naam \textit{getalnaam} waarvoor geldt dat de waarde voldoet aan \textit{query}. Dit soort instructie kan ook toegepast worden op een variabele dat maar \'e\'en object voorstelt. Dit vormt een keuzepunt bij simulatie.
	\item $SET <klassevariabele> TO <waarde>$: Voor alle leden van de verzameling wordt de waarde van de genoemde klassevariabele veranderd naar de genoemde waarde.
\end{itemize}

In een \textit{query} mogen er geen methodes voorkomen waarvan het gedrag gespecificeerd wordt in een sequentiediagram.

Vergelijk de \textit{CHOOSE}-instructies met de $\pi$-operator van GOLOG\cite{levesque1997golog}. GOLOG is een logische programmeertaal gebaseerd op de situatiecalculus. Net zoals de lineaire tijdscalculus is de situatiecalculus een formalisme binnen de eerste-orde-predicatenlogica om dynamische systemen te modelleren. De basisblokken van de situatiecalculus zijn:

\begin{itemize}
	\item De acties die uitgevoerd kunnen worden in het systeem.
	\item De \textit{fluents} die de toestand van het systeem beschrijven.
	\item De situaties. Een situatie is een geschiedenis van uitgevoerde acties.
	
\end{itemize}

De $pi$-operator is de non-deterministische keuzeoperator, gebruikt als $(\pi{}x)[\delta{}(x)]$, waar $\delta{}(x)$ \'e\'en of meerdere acties die de variabele $x$ gebruiken voorstelt. De $pi$-operator bepaalt non-deterministisch de waarde van het invoerargument voor de acties waar het op wordt toegepast. De \textit{CHOOSE}-instructies bepalen daarentegen non-deterministsch de waarde van een variabele.

We illustreren het gebruik van deze nieuwe instructies door middel van het voorbeeld van figuur \ref{fig:new-nim}, dat Nim volledig modelleert. De beurt gaat eerst naar de gekozen speler. Dan wordt gecontroleerd of alle stapels leeg zijn. Als dat niet het geval is, kiest de speler eerst een stapel en neemt dan minstens \'e\'en object weg van die stapel. Daarna wordt de beurt gegeven aan de volgende speler. Als alle stapels leeg zijn, is de huidige speler de winnaar.

\begin{figure}[H]
	\includegraphics[width=\textwidth]{chap-gedrag/seq-new-nim.png}
	\caption{Een modellering van het spel Nim}
	\label{fig:new-nim}
\end{figure}

De lusvoorwaarde wordt als volgt voorgesteld:

\begin{align}
\nonumber &\forall{t}[Time]\forall{st}[StackLevel](C\_SDPointAt(Next(t), play\_3) \leftarrow \\ \nonumber &(CurrentStackLevel(t) = st) \land (SDPointAt(t, play\_2) \lor SDPointAt(t, play\_6)) \land \\ \nonumber &\lnot{}\exists{g}[Game](GameT(t, st, g) \land{}\forall{h}[Heap](GameandHeap(g, h) \\ &\Rightarrow \exists{n}[LimitedInt](HeapamountObjects(t, h, n) \land n = 0)))). \\
\nonumber &\forall{t}[Time]\forall{st}[StackLevel](C\_SDPointAt(Next(t), play\_7) \leftarrow \\ \nonumber &(CurrentStackLevel(t) = st) \land SDPointAt(t, play\_6) \land \exists{g}[Game](GameT(t, st, g) \land \\ \nonumber &\forall{h}[Heap](GameandHeap(g, h) \\ &\Rightarrow \exists{n}[LimitedInt](HeapamountObjects(t, h, n) \land n = 0)))).
\end{align}

In instructie 3 wordt de $CHOOSE\ ONE\ WHERE\ APPLIES\ [query]$ instructie gebruikt. We introduceren hier een open predicaat, $ChosenHeap(Time, Heap)$, om ons te helpen deze instructie te vertalen. We voegen de volgende zin toe aan het definitieblok voor diagramvariabelen:

\begin{align}
\nonumber &\forall{t}[Time]\forall{st}[StackLevel]\forall{h}[Heap](C\_HeapT(Next(t), st, h) \\ &\leftarrow (CurrentStackLevel(t) = st) \land SDPointAt(t, play\_3) \land ChosenHeap(t, h)).
\end{align}

Om ervoor te zorgen dat exact \'e\'en Heap-object wordt gekozen, moeten we voorwaardes leggen op $ChosenHeap/2$. Dit doen we als volgt:

\begin{align}
&\forall{t}[Time](SDPointAt(t, play\_3) \Rightarrow \exists_{=1}{h}[Heap](ChosenHeap(t, h))).\label{form:uniqueheap} \\
\nonumber &\forall{t}[Time]\forall{h}[Heap](ChosenHeap(t, h) \Rightarrow (SDPointAt(t, play\_3) \\ \nonumber &\land \exists{g}[Game]\exists{st}[StackLevel]\exists{n}[LimitedInt]((CurrentStackLevel(t) = st) \\ &\land GameT(t, st, g) \land GameandHeap(g, h) \land HeapamountObjects(t, h, n) \land \lnot(n = 0)))).\label{form:correctsd}
\end{align}

Zin \ref{form:uniqueheap} garandeert dat er maar \'e\'en \textit{Heap}-object mag gekozen worden.

Zin \ref{form:correctsd} garandeert dat er enkel een keuze mag gemaakt worden als de uitvoering instructie 3 bereikt en dat er enkel een \textit{Heap}-object wordt gekozen waarvoor geldt dat \textit{amountObjects} groter is dan 0.

Voor instructie 4 introduceren we op een soortgelijke manier een open predicaat $ChosenTake/2$. De zin die we toevoegen aan het definitieblok voor variabelen en de voorwaarden die we opleggen op $ChosenTake/2$ zien er gelijkaardig uit.

Bijlage \ref{code:new-nim} bevat de vertaling van het diagram in figuur \ref{fig:new-nim}.

\section{Performantie van modelexpansie en progressie\"inferentie}\label{sec:dec-performance}

We beoordelen voor de theorie in bijlage \ref{code:new-nim} de rekentijd en het geheugengebruik voor modelexpansie en progressie\"inferentie.

Voor modelexpansie specificeren we in de invoerstructuur \'e\'en stapel met twee objecten en \'e\'en stapel met drie objecten. Het scenario is dus analoog aan die in hoofdstuk \ref{sec:evaluatie}. Verder defini\"eren we 11 tijdstappen, wat genoeg is voor twee beurten. Het duurt ongeveer 5,8 seconden om een volledig spelverloop te berekenen, met een \textit{grounding}-grootte van 15 249 en virtueel geheugengebruik van 71,18 MB. Modelexpansie is dus significant performanter voor deze theorie dan voor de theorie ge\"evalueerd in hoofdstuk \ref{sec:evaluatie}.

We simuleren weer een spelverloop waar de eerste speler alle drie objecten van de tweede stapel neemt, dan de tweede speler \'e\'en object van de eerste stapel, en dan de eerste speler het laatste object. In totaal is de simulatie 16 tijdstappen lang, met een totale simulatietijd van 93,11 seconden en een totaal geheugengebruik van 242,3 MB. Als we dit vergelijken met de resultaten in tabel \ref{tab:sim-mem}, zien we dat zowel de simulatietijd als het geheugengebruik significant kleiner zijn.

\section{Variabele associaties}

Sectie \ref{sec:eval-design} beschouwde hoe het feit dat de invulling van een associatie niet kan veranderen over de tijd ontwerpen onintu\"itief maakt. Deze sectie onderzoekt de impact op de performantie in termen van rekentijd en geheugengebruik als dat nu wel toegelaten is.

Associaties defini\"eren nu impliciet twee operaties. Voor een associatie \textit{X}---\textit{Y} wordt voor de klasse \textit{X} de operatie \textit{setY(Y)} en de operatie \textit{unset(Y)} gedefinieerd. \textit{setY(Y)} relateert het object van klasse \textit{X} tot het gegeven object van klasse \textit{Y}. \textit{unsetY(Y)} zorgt ervoor dat het object niet meer gerelateerd is aan het gegeven object.

We maken weer een nieuw ontwerp voor Nim.

Figuur \ref{fig:nim-assoc-cd} toont het klassediagram. Er is een nieuwe klasse vergeleken met het ontwerp in sectie \ref{sec:nim-design}, namelijk \textit{Player}. De associatie \textit{Game}---\textit{Player} duidt de winnaar aan.

Figuur \ref{fig:nim-assoc-play} geeft het sequentiediagram voor \textit{play(Player, Player)}. De lus wordt uitgevoerd zolang nog niet alle stapels leeg zijn. Eerst krijgt \textit{player1} de beurt en daarna \textit{player2} als er nog een niet-lege stapel is. Instructie 3 markeert \textit{player2} als de winnaar indien \textit{player1} het laatste object neemt. Instructie 5 doet hetzelfde voor \textit{player1}.

Figuur \ref{fig:nim-assoc-taketurn} toont het sequentiediagram voor \textit{takeTurn(Game)}. De speler kiest een niet-lege stapel en neemt minstens \'e\'en object ervan weg.

Bijlage \ref{app:nim-assoc} bevat het IDP-bestand dat deze diagrammen modelleert. De predicaten die de associaties voorstellen, namelijk \textit{GameandHeap} en \textit{GameandPlayer}, zijn inertieel. Zo laten we toe dat associaties variabel zijn over de tijd.

\begin{figure}
	\centering
	\includegraphics[width=0.75\textwidth]{chap-declaratieve-seq/nim-assoc-cd.png}
	\caption{Klassediagram voor Nim, met een extra klasse \textit{Player}}
	\label{fig:nim-assoc-cd}
\end{figure}

\begin{figure}
	\centering
	\begin{subfigure}{\textwidth}
		\includegraphics[width=\textwidth]{chap-declaratieve-seq/play.png}
		\caption{Sequentiediagram voor \textit{play(Player, Player)}}
		\label{fig:nim-assoc-play}
	\end{subfigure}
	\begin{subfigure}{\textwidth}
		\includegraphics[width=\textwidth]{chap-declaratieve-seq/takeTurn.png}
		\caption{Sequentiediagram voor \textit{takeTurn(Game)}}
		\label{fig:nim-assoc-taketurn}
	\end{subfigure}
	\caption{Sequentiediagrammen voor \textit{play(Player, Player) en \textit{takeTurn(Game)}}}
	\label{fig:nim-assoc-seq}
\end{figure}

We evalueren modelexpansie en progressie\"inferentie.

In de invoerstructuur geven we twee stapels, \'e\'en van twee objecten en \'e\'en van drie objecten. We defini\"eren 21 tijdstappen, wat genoeg is om drie beurten te spelen. Modelexpansie vindt alle modellen na 15,58 seconden, met een \textit{grounding}-grootte van 424 682 en virtueel geheugengebruik van 592,55 GB.

Via progressie\"inferentie spelen we het spel in drie beurten. De simulatie is 22 stappen lang. De simulatie duurt drie minuten en gebruikt 462,02 MB aan virtueel geheugen.

Vergeleken met sectie \ref{sec:dec-performance} is er een aanzienlijke impact op performantie. Dit is niet enkel te wijten aan grotere zoekdomeinen omwille van de inerti\"ele predicaten voor associaties, maar ook aan de aanwezigheid van een groter aantal variabelen en aan de meer omvangrijke theorie. Toch blijft de performantie significant beter dan voor de theorie ge\"evalueerd in hoofdstuk \ref{sec:evaluatie}.

\section{Verdere uitbreidingen aan sequentiediagrammen}

In principe is het in LTC mogelijk om meerdere acties tegelijk uit te voeren op het systeem in \'e\'en tijdstap. Dit suggereert dat het voor complexere problemen dan Nim interessant kan zijn om parallellisme in sequentiediagrammen voor te kunnen stellen in FO($\cdot$)-theorie\'en. Er zijn twee mogelijke paden daarin om verdere uitbreidingen te onderzoeken.

\subsection{Het parallel gecombineerd fragment}

UML biedt het parallel gecombineerd fragment aan. Figuur \ref{fig:seq-par} bevat een voorbeeld. De streepjeslijn scheidt de twee delen van het fragment. Het is toegelaten dat een parallel gecombineerd fragment meer dan twee delen heeft. Dit diagram specificeert dat \textit{m1()} wordt uitgevoerd v\'o\'or \textit{m2()} en dat \textit{m3()} wordt uitgevoerd v\'o\'or \textit{m4()}. Er zijn geen verdere voorwaarden op de uitvoeringsvolgorde van \textit{m1()}, \textit{m2()}, \textit{m3()} en \textit{m4()}. De uitvoering kan wel pas uit het fragment springen als zowel \textit{m3()} als \textit{m4()} zijn uitgevoerd.

\begin{figure}
	\centering
	\includegraphics[width=0.6\textwidth]{chap-declaratieve-seq/seq-par.png}
	\caption{Een voorbeeld van het parallel gecombineerd fragment}
	\label{fig:seq-par}
\end{figure}

Zulk een fragment zou dus kunnen toelaten dat meer dan \'e\'en instructie tegelijk wordt uitgevoerd. Om zulke fragmenten te vertalen, zouden er toestandszinnen moeten komen voor \textit{SDPointAt/2} om bij binnenkomst in het fragment \textit{SDPointAt/2} waar te maken voor alle mogelijke eerste instructies van het fragment. Verder moet er een zin komen die stelt dat de uitvoering het fragment enkel kan verlaten als alle mogelijke laatste instructies zijn uitgevoerd.

\subsection{Een methode oproepen op meerdere objecten tegelijk}

Een andere denkpiste is het oproepen van een bepaalde methode op alle leden van een verzameling of de leden die aan een bepaald criterium voldoen. Zulke functionaliteit zou bijvoorbeeld toegepast kunnen worden om een sequentiediagram voor een booleaanse methode op te roepen op alle leden van een verzameling en om alle leden te selecteren waarvoor het resultaat \textit{true} is.

Er zijn meerdere mogelijke aanpakken om zulke functionaliteit te modelleren in LTC:

\begin{itemize}
	\item Laat toe dat alle methodes parallel kunnen worden uitgevoerd. Hiervoor is het echter noodzakelijk om objecten bij te houden als eigenaar van een bepaalde oproep aangezien de resultaten van een parallele oproep enkel betekenis hebben binnen de context waarin de parallele oproep gebeurt. Dit kan men doen door instanties een extra argument te maken van \textit{SDPointAt} en een nieuw predicaat te introduceren dat voor alle objecten betrokken in een parallele oproep bijhoudt welk object de eigenaar is van de oproep die de bron is van die parallele oproep. Op die manier worden de resultaten van een parallele oproep steeds doorgegeven aan het juiste object. Men moet er echter voor zorgen dat de uitvoering niet springt naar de instructie direct na de parallelle oproep totdat de parallele oproep een resultaat heeft voor alle betrokken objecten. Deze aanpak zal waarschijnlijk een grote impact hebben op rekentijd en ruimtegebruik, maar het biedt de meest ruime mogelijkheden bij het ontwerpen van sequentiediagrammen.
	\item Markeer een methode expliciet als parallel uitvoerbaar en laat niet toe dat er een tijdens een parallele oproep een andere parallelle oproep gebeurt. Op deze manier vermijdt men de toevoeging van een extra argument aan \textit{SDPointAt/2} en extra predicaten die eigenaarschap van een oproep bijhouden. Men moet er enkel voor zorgen dat de resultaten van een parallele oproep hun juiste bestemming bereiken, bijvoorbeeld als er leden van een verzameling worden geselecteerd op basis van het resultaat van een booleaanse oproep of het toekennen van resultaten van een oproep van een methode die \textit{int} als resultaat heeft op alle leden van een verzameling aan een nieuwe verzameling. Deze aanpak zal waarschijnlijk een minder grote impact hebben op rekentijd en ruimtegebruik, maar dit beperkt de mogelijkheden bij het ontwerpen van sequentiediagrammen.
\end{itemize}

\section{Conclusie}

We vergeleken een rechtstreekse implementatie van Nim in LTC met de theorie gegenereerd uit het ontwerp in sectie \ref{sec:nim-design}. We merkten dat de LTC-theorie significant compacter was. Dit komt onder meer omdat sequentiediagrammen toelaten dat de toestand van maar \'e\'en variabele tegelijk wordt gecontroleerd. Verder heeft LTC enkel lussen in de vorm van iteratieve processen binnen het probleemdomein, waar sequentiediagrammen ook lussen moeten bevatten voor procedurale berekeningen. Deze observaties dienden als basis voor het ontwerpen van nieuwe soorten instructies voor sequentiediagrammen. De instructies integreren logica in de ontwerptaal voor sequentiediagrammen.

We maakten een nieuw ontwerp voor Nim waar we gebruikmaakten van deze instructies. Modelexpansie en progressie\"inferentie waren significant performanter voor de resulterende theorie. Deze winst blijft grotendeels bewaard wanneer we ook toelaten dat de invulling van een associatie verandert over de tijd.

We gaven tenslotte een aanzet voor toekomstig onderzoek naar mogelijkheden om een methode op te roepen op meerdere objecten tegelijkertijd.
	\chapter{Conclusie en verder onderzoek}\label{sec:conclusie}
We hebben getoond hoe we een klassediagram kunnen vertalen naar FO($\cdot$). Die vertaling kunnen we gebruiken om de consistentie van het diagram te controleren. We hebben ook een meer algemene logische theorie opgesteld die men kan gebruiken om te controleren op een aantal mogelijke kwaliteitsgebreken.

We hebben ook een procedure beschreven om een verzameling sequentiediagrammen die elkaar onderling kunnen oproepen te vertalen naar een logische theorie. Speciale aandacht ging hier naar het in \'e\'en stap kunnen bepalen naar welke instructie precies moet gesprongen worden indien de uitvoering een gecombineerd fragment tegenkomt. Die vertaling en de vertaling van de klassediagram kan men tezamen gebruiken om de uitvoering van het gemodelleerde systeem of een deel ervan te simuleren. Het is ook mogelijk om modelexpansie te gebruiken om te verifi\"eren of een diagram voldoet aan vooropgelegde vereisten. In termen van rekentijd en \textit{grounding}-grootte kunnen simulatie en verificatie echter duur zijn om uit te voeren als de sequentiediagrammen samen een zekere omvang bereiken.

Wat betreft verder onderzoek is het duidelijk dat een vertaler voor de integratie van logica met UML-sequentiediagrammen voorgesteld in hoofdstuk \ref{sec:decl-seq} significant kortere theorie\"en---omdat instructies krachtiger zijn---en kleinere vocabularia---omdat er minder variabelen nodig zijn---kan opleveren. Dit zou vanzelfsprekend een grote invloed hebben op rekentijd en \textit{grounding}-grootte. Er zouden zo meer opties opengaan voor verscheidene verificaties. Als men een stel diagrammen heeft dat een spel modelleert, zou men bijvoorbeeld kunnen controleren of die diagrammen toelaten dat een speler een ongeldige zet doet.

Een verdere uitdaging bestaat erin om gegeven een structuur te controleren of die structuur consistent is met een volledige of gedeeltelijke uitvoering van een sequentiediagram. Zo kan men bijvoorbeeld controleren of een structuur die overeenkomt met de zetten die een bepaalde speler doet geldig zijn volgens een bepaald stel diagrammen dat een spel modelleert. Liefst zou zulk een structuur zoveel mogelijk onafhankelijk zijn van interne boekhouding van de diagrammen. De procedure moet kunnen afleiden welke logische symbolen overeenkomen met logische symbolen van de vertaling van de diagrammen en welke symbolen keuzes van de speler aanduiden. Progressie\"inferentie moet kunnen gebruikt worden om die keuzes in rekening te brengen en zo tot een antwoord te komen of de structuur overeenkomt met een model voor de theorie die als resultaat wordt gegeven door het diagramvertaalproces.
	
	\appendixpage*
	\appendix
	\chapter{IDP-bestand resulterend uit de procedure beschreven in hoofdstuk \ref{sec:consistentie}}\label{app:consistentie}

\lstinputlisting[title=Codebestand 0.4.3]{chap-rol-idp/generatedtheory.idp}\label{code:consistentie}

\chapter{IDP-bestand resulterend uit de procedure beschreven in hoofdstuk \ref{sec:kwaliteitsgebrek}}\label{app:kwaliteitsgebrek}

\lstinputlisting[title=Theorie voor het opsporen van kwaliteitsgebreken]{chap-rol-idp/defs.idp}\label{code:kwaliteitsgebrek}

\chapter{IDP-bestand voor sequentiediagram van het spelvoorbeeld}\label{app:seq-diagram-game}

\lstinputlisting[title=Modellering van het gedrag van het sequentiediagram in figuur \ref{fig:seq-diagram-game}]{idp-sources/seq-diagram-game.idp}\label{code:seq-diagram-game}

\chapter{IDP-bestand voor sequentiediagram voor het voorbeeld over recursie}\label{app:seq-recursion}

\lstinputlisting[title=Modellering van het gedrag van de sequentiediagrammen in figuur \ref{fig:seq-recursion}]{idp-sources/recursion.idp}\label{code:seq-recursion}

\chapter{IDP-bestand voor sequentiediagram voor het voorbeeld over extra instructies}\label{app:new-nim}

\lstinputlisting[title=Modellering van Nim gebruikmakend van nieuwe soorten instructies]{idp-sources/new-lang.idp}\label{code:new-nim}

\chapter{IDP-bestand voor het ontwerp van Nim}

\lstinputlisting[title=Modellering van Nim voor hoofdstuk \ref{sec:evaluatie}]{chap-evaluatie/nimmodel.idp}\label{code:nim-eval}

\chapter{Wetenschappelijk artikel}

\includepdf[pages=-]{IEEEtran/article.pdf}
%
%% bare_conf.tex
%% V1.4b
%% 2015/08/26
%% by Michael Shell
%% See:
%% http://www.michaelshell.org/
%% for current contact information.
%%
%% This is a skeleton file demonstrating the use of IEEEtran.cls
%% (requires IEEEtran.cls version 1.8b or later) with an IEEE
%% conference paper.
%%
%% Support sites:
%% http://www.michaelshell.org/tex/ieeetran/
%% http://www.ctan.org/pkg/ieeetran
%% and
%% http://www.ieee.org/

%%*************************************************************************
%% Legal Notice:
%% This code is offered as-is without any warranty either expressed or
%% implied; without even the implied warranty of MERCHANTABILITY or
%% FITNESS FOR A PARTICULAR PURPOSE! 
%% User assumes all risk.
%% In no event shall the IEEE or any contributor to this code be liable for
%% any damages or losses, including, but not limited to, incidental,
%% consequential, or any other damages, resulting from the use or misuse
%% of any information contained here.
%%
%% All comments are the opinions of their respective authors and are not
%% necessarily endorsed by the IEEE.
%%
%% This work is distributed under the LaTeX Project Public License (LPPL)
%% ( http://www.latex-project.org/ ) version 1.3, and may be freely used,
%% distributed and modified. A copy of the LPPL, version 1.3, is included
%% in the base LaTeX documentation of all distributions of LaTeX released
%% 2003/12/01 or later.
%% Retain all contribution notices and credits.
%% ** Modified files should be clearly indicated as such, including  **
%% ** renaming them and changing author support contact information. **
%%*************************************************************************


% *** Authors should verify (and, if needed, correct) their LaTeX system  ***
% *** with the testflow diagnostic prior to trusting their LaTeX platform ***
% *** with production work. The IEEE's font choices and paper sizes can   ***
% *** trigger bugs that do not appear when using other class files.       ***                          ***
% The testflow support page is at:
% http://www.michaelshell.org/tex/testflow/



\documentclass[conference]{IEEEtran}
% Some Computer Society conferences also require the compsoc mode option,
% but others use the standard conference format.
%
% If IEEEtran.cls has not been installed into the LaTeX system files,
% manually specify the path to it like:
% \documentclass[conference]{../sty/IEEEtran}





% Some very useful LaTeX packages include:
% (uncomment the ones you want to load)


% *** MISC UTILITY PACKAGES ***
%
%\usepackage{ifpdf}
% Heiko Oberdiek's ifpdf.sty is very useful if you need conditional
% compilation based on whether the output is pdf or dvi.
% usage:
% \ifpdf
%   % pdf code
% \else
%   % dvi code
% \fi
% The latest version of ifpdf.sty can be obtained from:
% http://www.ctan.org/pkg/ifpdf
% Also, note that IEEEtran.cls V1.7 and later provides a builtin
% \ifCLASSINFOpdf conditional that works the same way.
% When switching from latex to pdflatex and vice-versa, the compiler may
% have to be run twice to clear warning/error messages.






% *** CITATION PACKAGES ***
%
\usepackage{cite}
% cite.sty was written by Donald Arseneau
% V1.6 and later of IEEEtran pre-defines the format of the cite.sty package
% \cite{} output to follow that of the IEEE. Loading the cite package will
% result in citation numbers being automatically sorted and properly
% "compressed/ranged". e.g., [1], [9], [2], [7], [5], [6] without using
% cite.sty will become [1], [2], [5]--[7], [9] using cite.sty. cite.sty's
% \cite will automatically add leading space, if needed. Use cite.sty's
% noadjust option (cite.sty V3.8 and later) if you want to turn this off
% such as if a citation ever needs to be enclosed in parenthesis.
% cite.sty is already installed on most LaTeX systems. Be sure and use
% version 5.0 (2009-03-20) and later if using hyperref.sty.
% The latest version can be obtained at:
% http://www.ctan.org/pkg/cite
% The documentation is contained in the cite.sty file itself.






% *** GRAPHICS RELATED PACKAGES ***
%
\ifCLASSINFOpdf
  \usepackage[pdftex]{graphicx}
  % declare the path(s) where your graphic files are
  % \graphicspath{{../pdf/}{../jpeg/}}
  % and their extensions so you won't have to specify these with
  % every instance of \includegraphics
  \DeclareGraphicsExtensions{.pdf,.jpeg,.png}
\else
  % or other class option (dvipsone, dvipdf, if not using dvips). graphicx
  % will default to the driver specified in the system graphics.cfg if no
  % driver is specified.
  % \usepackage[dvips]{graphicx}
  % declare the path(s) where your graphic files are
  % \graphicspath{{../eps/}}
  % and their extensions so you won't have to specify these with
  % every instance of \includegraphics
  % \DeclareGraphicsExtensions{.eps}
\fi
% graphicx was written by David Carlisle and Sebastian Rahtz. It is
% required if you want graphics, photos, etc. graphicx.sty is already
% installed on most LaTeX systems. The latest version and documentation
% can be obtained at: 
% http://www.ctan.org/pkg/graphicx
% Another good source of documentation is "Using Imported Graphics in
% LaTeX2e" by Keith Reckdahl which can be found at:
% http://www.ctan.org/pkg/epslatex
%
% latex, and pdflatex in dvi mode, support graphics in encapsulated
% postscript (.eps) format. pdflatex in pdf mode supports graphics
% in .pdf, .jpeg, .png and .mps (metapost) formats. Users should ensure
% that all non-photo figures use a vector format (.eps, .pdf, .mps) and
% not a bitmapped formats (.jpeg, .png). The IEEE frowns on bitmapped formats
% which can result in "jaggedy"/blurry rendering of lines and letters as
% well as large increases in file sizes.
%
% You can find documentation about the pdfTeX application at:
% http://www.tug.org/applications/pdftex





% *** MATH PACKAGES ***
%
\usepackage{amsmath}
% A popular package from the American Mathematical Society that provides
% many useful and powerful commands for dealing with mathematics.
%
% Note that the amsmath package sets \interdisplaylinepenalty to 10000
% thus preventing page breaks from occurring within multiline equations. Use:
\interdisplaylinepenalty=2500
% after loading amsmath to restore such page breaks as IEEEtran.cls normally
% does. amsmath.sty is already installed on most LaTeX systems. The latest
% version and documentation can be obtained at:
% http://www.ctan.org/pkg/amsmath





% *** SPECIALIZED LIST PACKAGES ***
%
%\usepackage{algorithmic}
% algorithmic.sty was written by Peter Williams and Rogerio Brito.
% This package provides an algorithmic environment fo describing algorithms.
% You can use the algorithmic environment in-text or within a figure
% environment to provide for a floating algorithm. Do NOT use the algorithm
% floating environment provided by algorithm.sty (by the same authors) or
% algorithm2e.sty (by Christophe Fiorio) as the IEEE does not use dedicated
% algorithm float types and packages that provide these will not provide
% correct IEEE style captions. The latest version and documentation of
% algorithmic.sty can be obtained at:
% http://www.ctan.org/pkg/algorithms
% Also of interest may be the (relatively newer and more customizable)
% algorithmicx.sty package by Szasz Janos:
% http://www.ctan.org/pkg/algorithmicx




% *** ALIGNMENT PACKAGES ***
%
%\usepackage{array}
% Frank Mittelbach's and David Carlisle's array.sty patches and improves
% the standard LaTeX2e array and tabular environments to provide better
% appearance and additional user controls. As the default LaTeX2e table
% generation code is lacking to the point of almost being broken with
% respect to the quality of the end results, all users are strongly
% advised to use an enhanced (at the very least that provided by array.sty)
% set of table tools. array.sty is already installed on most systems. The
% latest version and documentation can be obtained at:
% http://www.ctan.org/pkg/array


% IEEEtran contains the IEEEeqnarray family of commands that can be used to
% generate multiline equations as well as matrices, tables, etc., of high
% quality.




% *** SUBFIGURE PACKAGES ***
%\ifCLASSOPTIONcompsoc
%  \usepackage[caption=false,font=normalsize,labelfont=sf,textfont=sf]{subfig}
%\else
\usepackage[caption=false,font=footnotesize]{subfig}
%\fi
% subfig.sty, written by Steven Douglas Cochran, is the modern replacement
% for subfigure.sty, the latter of which is no longer maintained and is
% incompatible with some LaTeX packages including fixltx2e. However,
% subfig.sty requires and automatically loads Axel Sommerfeldt's caption.sty
% which will override IEEEtran.cls' handling of captions and this will result
% in non-IEEE style figure/table captions. To prevent this problem, be sure
% and invoke subfig.sty's "caption=false" package option (available since
% subfig.sty version 1.3, 2005/06/28) as this is will preserve IEEEtran.cls
% handling of captions.
% Note that the Computer Society format requires a larger sans serif font
% than the serif footnote size font used in traditional IEEE formatting
% and thus the need to invoke different subfig.sty package options depending
% on whether compsoc mode has been enabled.
%
% The latest version and documentation of subfig.sty can be obtained at:
% http://www.ctan.org/pkg/subfig




% *** FLOAT PACKAGES ***
%
%\usepackage{fixltx2e}
% fixltx2e, the successor to the earlier fix2col.sty, was written by
% Frank Mittelbach and David Carlisle. This package corrects a few problems
% in the LaTeX2e kernel, the most notable of which is that in current
% LaTeX2e releases, the ordering of single and  column floats is not
% guaranteed to be preserved. Thus, an unpatched LaTeX2e can allow a
% single column figure to be placed prior to an earlier double column
% figure.
% Be aware that LaTeX2e kernels dated 2015 and later have fixltx2e.sty's
% corrections already built into the system in which case a warning will
% be issued if an attempt is made to load fixltx2e.sty as it is no longer
% needed.
% The latest version and documentation can be found at:
% http://www.ctan.org/pkg/fixltx2e


%\usepackage{stfloats}
% stfloats.sty was written by Sigitas Tolusis. This package gives LaTeX2e
% the ability to do double column floats at the bottom of the page as well
% as the top. (e.g., "\begin{figure*}[!b]" is not normally possible in
% LaTeX2e). It also provides a command:
%\fnbelowfloat
% to enable the placement of footnotes below bottom floats (the standard
% LaTeX2e kernel puts them above bottom floats). This is an invasive package
% which rewrites many portions of the LaTeX2e float routines. It may not work
% with other packages that modify the LaTeX2e float routines. The latest
% version and documentation can be obtained at:
% http://www.ctan.org/pkg/stfloats
% Do not use the stfloats baselinefloat ability as the IEEE does not allow
% \baselineskip to stretch. Authors submitting work to the IEEE should note
% that the IEEE rarely uses double column equations and that authors should try
% to avoid such use. Do not be tempted to use the cuted.sty or midfloat.sty
% packages (also by Sigitas Tolusis) as the IEEE does not format its papers in
% such ways.
% Do not attempt to use stfloats with fixltx2e as they are incompatible.
% Instead, use Morten Hogholm'a dblfloatfix which combines the features
% of both fixltx2e and stfloats:
%
% \usepackage{dblfloatfix}
% The latest version can be found at:
% http://www.ctan.org/pkg/dblfloatfix




% *** PDF, URL AND HYPERLINK PACKAGES ***
%
%\usepackage{url}
% url.sty was written by Donald Arseneau. It provides better support for
% handling and breaking URLs. url.sty is already installed on most LaTeX
% systems. The latest version and documentation can be obtained at:
% http://www.ctan.org/pkg/url
% Basically, \url{my_url_here}.




% *** Do not adjust lengths that control margins, column widths, etc. ***
% *** Do not use packages that alter fonts (such as pslatex).         ***
% There should be no need to do such things with IEEEtran.cls V1.6 and later.
% (Unless specifically asked to do so by the journal or conference you plan
% to submit to, of course. )


% correct bad hyphenation here

\usepackage{todo}

\hyphenation{op-tical net-works semi-conduc-tor}


\begin{document}
%
% paper title
% Titles are generally capitalized except for words such as a, an, and, as,
% at, but, by, for, in, nor, of, on, or, the, to and up, which are usually
% not capitalized unless they are the first or last word of the title.
% Linebreaks \\ can be used within to get better formatting as desired.
% Do not put math or special symbols in the title.
\title{Translating UML Class Diagrams and Sequence Diagrams to FO($\cdot$) to Facilitate Simulation and Verification}


% author names and affiliations
% use a multiple column layout for up to three different
% affiliations
\author{\IEEEauthorblockN{Thomas Vochten}
\IEEEauthorblockA{Department of Computer Science\\
Katholieke Universiteit Leuven}}
% conference papers do not typically use \thanks and this command
% is locked out in conference mode. If really needed, such as for
% the acknowledgment of grants, issue a \IEEEoverridecommandlockouts
% after \documentclass

% for over three affiliations, or if they all won't fit within the width
% of the page, use this alternative format:
% 
%\author{\IEEEauthorblockN{Michael Shell\IEEEauthorrefmark{1},
%Homer Simpson\IEEEauthorrefmark{2},
%James Kirk\IEEEauthorrefmark{3}, 
%Montgomery Scott\IEEEauthorrefmark{3} and
%Eldon Tyrell\IEEEauthorrefmark{4}}
%\IEEEauthorblockA{\IEEEauthorrefmark{1}School of Electrical and Computer Engineering\\
%Georgia Institute of Technology,
%Atlanta, Georgia 30332--0250\\ Email: see http://www.michaelshell.org/contact.html}
%\IEEEauthorblockA{\IEEEauthorrefmark{2}Twentieth Century Fox, Springfield, USA\\
%Email: homer@thesimpsons.com}
%\IEEEauthorblockA{\IEEEauthorrefmark{3}Starfleet Academy, San Francisco, California 96678-2391\\
%Telephone: (800) 555--1212, Fax: (888) 555--1212}
%\IEEEauthorblockA{\IEEEauthorrefmark{4}Tyrell Inc., 123 Replicant Street, Los Angeles, California 90210--4321}}




% use for special paper notices
%\IEEEspecialpapernotice{(Invited Paper)}




% make the title area
\maketitle

% As a general rule, do not put math, special symbols or citations
% in the abstract
\begin{abstract}
Representing UML diagrams in logic can benefit a designer since the designer can easily miss errors or inefficiencies in the design if the diagrams grow sufficiently elaborate or numerous. In this article, we show a method to translate class diagrams and a corresponding set of sequence diagrams that model the behavior of a desired software system to FO($\cdot$), an extension of first-order predicate logic with inductive definitions, partial functions, aggregates and types. We show how the output theory may be used to verify consistency of a class diagram and detect the presence of certain design flaws in the class diagram. In addition, we show that the theory may be used to simulate system behavior as modelled in the sequence diagrams and how to verify that requirements regarding the output of a diagram are satisfied. We also evaluate performance in terms of execution time and the size of the grounding for each of these tasks.
\end{abstract}

% no keywords




% For peer review papers, you can put extra information on the cover
% page as needed:
% \ifCLASSOPTIONpeerreview
% \begin{center} \bfseries EDICS Category: 3-BBND \end{center}
% \fi
%
% For peerreview papers, this IEEEtran command inserts a page break and
% creates the second title. It will be ignored for other modes.
\IEEEpeerreviewmaketitle



\section{Introduction}
% no \IEEEPARstart
UML\cite{RumbaughJames2005Tuml} is a modelling language used in software engineering to graphically model a design for a software system. Popular types of diagrams are the class diagram and the sequence diagram. However, when a class diagram grows too elaborate or the sequence diagrams too numerous, the designer can easily miss any design errors or flaws or unintended behavior. Since errors made in the design phase can be costly to correct if found late in the software production process, automated verification of consistency and detection of the presence of design flaws might help the designer spot issues they might otherwise miss.

In this article we show a method to translate class diagrams and sequence diagrams that specify the behavior of methods defined in the corresponding class diagram to FO($\cdot$)\cite{DeCatBroes2014PLaa}, an extension of first-order logic with inductive definitions, aggregates, partial functions and types.

We will use a translation of a class diagram to verify whether the diagram is consistent, i.e. whether the FO($\cdot$) theory corresponding to the diagram has a model. We will also use an alternative translation to detect certain types of design flaws which make it more difficult to gain a clear understanding of a class diagram.

We will use a translation of a set of sequence diagrams to simulate the behavior modelled therein. We will then use that translation to verify whether the diagrams fulfill certain requirements.

The software we use to verify consistency, detect design flaws, simulate sequence diagrams and verify requirements for sequence diagrams is IDP\cite{DeCatBroes2014PLaa}, a knowledge base system for FO($\cdot$). IDP can perform model expansion, progression inference and other forms of inference given one or more input vocabularies, theories and structures. 

The rest of this article is structured as follows. Section \ref{sec:related-work} gives an overview of related work with regard to translating UML diagrams to logic. Section \ref{sec:class-diagram} introduces a method to translate class diagrams to FO($\cdot$) and verifies consistency of and detects design flaws in an example diagram. Section \ref{sec:seq} describes our method to translate a set of sequence diagrams to FO($\cdot$). Section \ref{sec:evaluation} evaluates the use of our method to design a set of diagrams that models the game Nim, whether the translation can be used to simulate the game and to verify whether certain requirements hold, and performance in terms of execution time and grounding size. Finally, section \ref{sec:conclusion} concludes the article.
% You must have at least 2 lines in the paragraph with the drop letter
% (should never be an issue)

\section{Related work}\label{sec:related-work}

This section provides an overview of previous work on translating UML diagrams to several kinds of logic.

In \cite{BerardiDaniela2005RoUc}, the authors first provide a method to translate class diagrams to FO-theories. The vocabulary first defines: Predicates that express membership of a class; predicates that relate instances of a class to their respective values for class attributes; predicates that model operations on a class by relating the callee object, values for parameters and the result value corresponding to that callee object and those parameters values; and predicates that relate objects of a class to objects of another class according to associations defined in the class diagram. The rules contained in the output theory make use of the membership predicates to enforce correct typing for the other kinds of predicates and ensure that the predicates for attributes, operations and associations obey the multiplicites imposed on them by the diagrams. Finally, the theory models inheritance by introducing rules that state that all members of a subclass must also be members of the superclass. Note that these inheritance rules do not prevent members of one class being members of another class even if the diagram defines no inheritance relationship between those two classes, regardless of whether the developer wants to allow for that possibility. The rest of the paper describes a method to translate class diagrams to several kinds of description logics and how to use those translations to verify consistency of the diagram and to detect implicit properties which unintentionally hold in the diagram. In this article, we adapt the method for translating class diagrams to FO introduced by the authors to make use of logical types, thereby removing the need to introduce rules in the output theory that enforce the correct typing and rules that model inheritance.

In \cite{KuhlmannMirco2012FUaO}, the authors briefly introduce relational logic and outline how to translate class diagrams to relational logic by defining constants that represent instances of a class and binary relations that relate constants according to attribute and association relationships. The translation imposes conditions on membership of the relations in order to enforce correct typing and multiplicites defined in the diagram. They also show how to translate constraints expressed in OCL\cite{WarmerJosB1999Ocl:}. The authors describe how they use Kodkod\cite{10.1007/978-3-540-71209-1_49} to calculate instances of an output relational model. Kodkod translates relational models to SAT-formulas and translates solutions in SAT format back to solutions for the relational model.

In \cite{LIMA2009143}, the authors describe how to represent sequence diagrams in PROMELA\cite{neumann2014using}, a modelling language that models processes. PROMELA is used by the SPIN\cite{holzmann2004spin} model checker to verify whether properties expressed in linear temporal logic (LTL) hold in the modelled processes. It is possible to simulate sequence diagrams with the proposed representation. The paper also describes how to represent state transitions during the execution of a sequence diagram by listing the performed action, the lifeline that performed the action, the message and the lifeline that sent/received the message for each time step. The authors present a sequence diagram that models the behavior of an ATM in order to prove they can verify certain properties for a sequence diagram. They express four properties in LTL. SPIN determines that only one property holds. For the other three properties, SPIN generates a counterexample of an execution of the sequence diagram that violates the property. However, the authors consider sequence diagrams independently from any class diagram. Therefore, their method cannot truly represent a system of which the state may change over time. They also do not consider sequence diagrams that call other sequence diagrams. Full understanding of a system may only be gained by considering that type of sequence diagram.

Every work discussed in this section uses a subset of FO to perform verification and certain forms of inference on UML class diagrams. Sequence diagrams are considered separately. Our contribution is to instead use FO($\cdot$) to both model class and sequence diagrams and to perform verification on those diagrams.

One cannot express transitive closure in FO in a general manner and hence also not in these subsets of FO. However, transitive closures are easily expressible in FO($\cdot$), and hence it can also correctly specify inheritance relationships. 

We prefer sequence diagrams to activity diagrams since the former allow the use of variables internal to a specific diagram.


\section{Translating class diagrams to FO($\cdot$)}\label{sec:class-diagram}

In this section, we adapt the method to translate class diagrams to FO introduced in \cite{BerardiDaniela2005RoUc} to make use of logical types. We will use the diagram in figure \ref{fig:game-class} to illustrate how we translate class diagrams to FO($\cdot$).

\subsection{Building a theory for checking consistency}\label{sec:consistency}

\begin{figure*}[!t]
\centering
\includegraphics[width=0.75\textwidth]{diagram-voorbeeld}
\caption{Example class diagram}
\label{fig:game-class}
\end{figure*}

Every class is represented by a logical type. In order to enforce correct typing for every predicate we introduce, we need merely specify the correct logical type corresponding to the class expected for each predicate argument.

We introduce a binary predicate for every class attribute and we name it according to the pattern \textit{Classnameattributename}. For example, the predicate corresponding to \textit{name} in \textit{Character} is \textit{Charactername(Character, string)}. In addition, we introduce rules which enforce the multiplicity imposed on the pattern according to the following general pattern:

\begin{align*}
	&\forall{o1}[ClassType](lowerBound \leq \#\{o2[attributeType] : \\ &Classnameattributename(o1, o2)\} \leq upperBound).
\end{align*}

For \textit{name} in \textit{Character}, we get:

\begin{align*}
	\forall{o1}[Character]\exists!{o2}[string](Charactername(o1, o2)).
\end{align*}

We deviate from \cite{BerardiDaniela2005RoUc} by not introducing predicates corresponding to class operations, since sequence diagrams should model the behavior of operations.

For each association, we introduce an $m$-ary predicate where $m$ equals the amount of classes involved in the association. We name them according to the pattern \textit{ClassOneand$\cdots$andClassM(ClassOne, $\cdots$, ClassM)}. The association between \textit{Inventory} and \textit{Item} thus yields \textit{InventoryandItem(Inventory, Item)}. Again, we enforce multiplicity by introducing rules for each role of each association according to the following general pattern:

\begin{align*}
	&\forall{c_1}[Class_1]\cdots{}\forall{c_m}[Class_m](lowerBound_l \leq \\ &\#\{o_l[Class_l] : ClassOneand\cdots{}andClassM\\&(c_1, \cdots, o_l, \cdots, c_m)\} \leq upperBound_l).
\end{align*}

where $l$ indicates the role for which the multiplicity is being enforced. This yields the following two rules for the \textit{Inventory}---\textit{Item} association:

\begin{align*}
	&\forall{o2}[Item](\#\{o1[Inventory] : \\ &InventoryandItem(o1, o2)\} \leq 1). \\
	&\forall{o1}[Inventory](\#\{o2[Item] : \\ &InventoryandItem(o1, o2)\} \leq 5).
\end{align*}

To model inheritance, the vocabulary must simply declare that logical types corresponding to a subclass are subtypes of the logical type corresponding to the superclass. This means that every input structure for the output theory must include all members of the interpretation of a subtype in the interpretation of all the corresponding supertypes in order to be valid input for model expansion. It is possible to spare the user this inconvenience by abandoning the principle of one logical type per class and instead introducing a general logical type \textit{Object} and introducing another logical type \textit{ClassObject} which has one object for every class. We would then add a binary predicate \textit{StaticClass(ClassObject, Object)} in order to express membership of a class. Inductive definitions would then use the inheritance relationships contained in the diagram to calculate the correct class memberships for every object. However, this approach would require rewriting every rule introduced earlier in this section to be longer by enforcing correct typing, which would make the output theory more difficult to understand. There would also have to be additional rules to enforce the correct type for every argument of every attribute and association predicate. For these reasons, we chose not to implement this approach.

We have used these rules to automatically generate an FO($\cdot$) theory based on the diagram in figure \ref{fig:game-class}. Given a valid input structure, IDP found a model. The conclusion is therefore that the diagram is consistent.

\subsection{Building a theory for detecting design flaws}\label{sec:design-flaw}

In order to detect design flaws, we need a separate theory where the classes themselves take center stage instead of specific instances of those classes. Therefore, we remove the logical types for each class and use the \textit{ClassObject} logical type we rejected in the previous subsection. We also introduce the predicates \textit{IsDirectSupertypeOf(ClassObject, ClassObject)} and \textit{IsSupertypeOf(ClassObject, ClassObject)} predicates. The former expresses a direct inheritance relationship while the latter is the transitive closure over \textit{IsDirectSupertypeOf/2}. We model that transitive closure in the following inductive definition:

\begin{align*}
	\{ &\forall{x}[ClassObject]\forall{y}[ClassObject](IsSupertypeOf(x, y) \\ &\leftarrow IsDirectSupertypeOf(x, y)). \\
	&\forall{x}[ClassObject]\forall{y}[ClassObject](IsSupertypeOf(y, x) \\ &\leftarrow \exists{z}[ClassObject](IsSupertypeOf(y, z)  \\ &\land IsSupertypeOf(z, x))).\}
\end{align*}

A separate inductive definition then fills in \textit{IsDirectSupertypeOf/2} with direct inheritance relationships read from the diagram.

In this article, we consider three different kinds of design flaws related to associations:

\begin{itemize}
	\item \textbf{Many-to-many associations}: The presence of a many-to-many association is generally an indication that there is a class missing from the design.
	\item \textbf{Loose class}: If a class cannot be reached by navigating associations, that class is essentially useless and should be eliminated.
	\item \textbf{Insufficiently precise upper bound in a multiplicity}\cite{Balaban2015}: Consider figure \ref{fig:design-flaw}. The multiplicities imply that classes \textit{Alice}, \textit{Bob} and \textit{Charlie} contain an equal number of instances. This renders the upper bound of 2 on \textit{Alice} in \textit{Charlie}---\textit{Alice} insufficiently precise since the other bounds imply that every instance of \textit{Charlie} is associated with exactly one instance of \textit{Alice}. Since this fact is not immediately apparent, such a bound should be corrected to improve the understandability of the diagram. Either set the upper bound on \textit{Alice} to 1, set the lower bound on \textit{Alice} to 0 or increase the upper bound on \textit{Charlie}.
\end{itemize}

\begin{figure}[!t]
	\centering
	\includegraphics[width=2in]{cycle}
	\caption{Example of loose class and insufficiently precise multiplicity upper bound}
	\label{fig:design-flaw}
\end{figure}

We introduce three other predicates in order to encode information about binary associations.

\textit{BiAssoc(ClassObject, ClassObject)} expresses that a binary association exists between two classes.

\textit{BiAssocLow(ClassObject, ClassObject, ClassObject, nat)} indicates that the role corresponding to the class named in the third argument has the lower bound specified in the fourth argument.

Similarly, \textit{BiAssocHigh(ClassObject, ClassObject, ClassObject, nat)} specifies the upper bound for the role corresponding to the argument parameter.

For the first two named design flaws, we introduce the following two rules:

\begin{align}
	\nonumber &\forall{x}[ClassObject]\forall{y}[ClassObject](ManyToMany(x, y) \\ \nonumber &\Leftrightarrow BiAssoc(x, y) \land \lnot \exists{z}[nat](BiAssocHigh(x, y, x, z)) \\ &\land \lnot \exists{z}[nat](BiAssocHigh(x, y, y, z))).\label{form:flaw1} \\
	\nonumber &\forall{x}[ClassObject](LooseClass(x) \Leftrightarrow \\ \nonumber &\lnot (\exists{y}[ClassObject](\lnot(x = y) \land (BiAssoc(x, y) \\ \nonumber  &\lor \exists{s}[ClassObject]\exists{y}[ClassObject]\\ &(IsSupertypeOf(s, x) \land BiAssoc(s, y)))))).\label{form:flaw2} \\
	&\nonumber\forall{x}[ClassObject]\forall{y}[ClassObject] \\
	&\nonumber(CycleImpreciseUpperBound(x, y, x) \\ &\nonumber\Leftrightarrow EqualNbInstances(x, y) \land BiAssocLow(x, y, x) \\ &\nonumber= BiAssocHigh(x, y, y) = BiAssocLow(x, y, y) \\ &\nonumber\land (BiAssocHigh(x, y, x) > BiAssocLow(x, y, x) \\ &\lor \lnot\exists{z}[nat](BiAssocHigh(x, y, x) = z))).\label{form:flaw3}
\end{align}

Rule \ref{form:flaw1} designates an association as many-to-many if neither role specifies an upper bound.

Rule \ref{form:flaw2} designates a class as loose if it is not directly associated with another class or if it does not have a superclass that is directly associated with another class.

Rule \ref{form:flaw3} designates an upper bound as insufficiently precise if the association is between two classes with an equal number of instances and if the other three bounds are both equal to each other and less than the considered upper bound. \textit{EqualNbInstances/2} is a helper predicate that expresses that two classes must have an equal number of instances. We construct the interpretation of this predicate by way of transitive closure over associations for which all bounds are equal.

We combine the information from the diagrams in figures \ref{fig:game-class} and \ref{fig:design-flaw} into one theory. IDP finds all many-to-many associations, concludes that \textit{Loose} is a loose class and the upper bound on \textit{Alice} in \textit{Charlie}---\textit{Alice} is insufficiently precise.

\section{Translating sequence diagrams to FO($\cdot$)}\label{sec:seq}
We will use the framework of linear time calculus\cite{BogaertsBart2014Sdsu} (LTC). LTC can model dynamic systems that change over time. Across each time step, only the inertial predicates that need be changed to the diagram do so, while all other inertial predicates remain the same. If we model diagram variables, an instruction pointer, a stack level counter and a return point stack as inertial predicates, we have exactly the elements we need to translate mutually recursive sequence diagrams to FO($\cdot$) theories.

In this section, we first describe how to expand the vocabulary and theory generated according to the rules described in section \ref{sec:consistency} in order to implement the elements summed up in the previous paragraph.

We introduce the following logical types and logical symbols to the vocabulary:

\begin{itemize}
	\item \textit{SDPoint}: This serves as an instruction pointer. It is a constructed type that consists of labels for all instructions across all given sequence diagrams, landing points after each call to a sequence diagram, and additional hidden last instructions we add for sequence diagrams that model methods that return \textit{void}. \textit{SDPoint}s are named according to the pattern \textit{$diagramname\_\langle{}sequencenumber\rangle$}, except landing points after diagram calls, which are named according to \textit{$diagramname\_\langle{}sequencenumber\rangle$post}
	\item \textit{NextSD(SDPoint) : SDPoint}: A partial function that specifies which \textit{SDPoint} follows the given \textit{SDPoint}. All \textit{SDPoint}s except the last instruction for each diagram are members of the domain.
	\item \textit{SDPointAt(Time, SDPoint)}: The \textit{SDPoint} at any given time step.
	\item \textit{StackLevel}: A non-zero natural number that indicates the stack depth.
	\item \textit{CurrentStackLevel(Time) : StackLevel}: A total function which specifies what the stack depth is at any given time step.
	\item \textit{ReturnPoint(Time, StackLevel, SDPoint)}: This predicate contains the return instruction at the specified stack depth. If the execution reaches the final instruction of a diagram, the instruction counter should be set to the corresponding \textit{SDPoint} at the following time step.
	\item Predicates for every diagram variable named according to the pattern \textit{VariableNameT(Time, StackLevel, VariableType)}. These diagram variables include variables defined in an instruction, lifeline objects and arguments of calls.
\end{itemize}

In addition, we modify the predicates corresponding to class variables to take logical type \textit{Time} as an additional argument. At this time, we do not allow the interpretation for association predicates to vary over time.

We follow the method outlined in \cite{BogaertsBart2014Sdsu} to make the logical symbols \textit{SDPointAt/2}, \textit{CurrentStackLevel/1} and \textit{ReturnPoint/3} inertial, as well as all diagram variable predicates and class variable predicates. Only class variable predicates have a corresponding \textit{uncause} predicate, since we allow only class variables to have more than one value at any given time step. If a diagram specifies an upper bound equal to one for the multiplicity of a class variable, that class variable may still have only one value at any given time step.

With these logical symbols, predicates and functions at our disposal, we are now equipped to translate sequence diagrams to FO($\cdot$) theories. Consider figure \ref{fig:recursion}. Diagram \ref{fig:methodOne} calls diagram \ref{fig:methodTwo}, whereupon diagram \ref{fig:methodTwo} calls itself two times and then returns. In this case, the members of \textit{SDPoint} are \textit{methodOne\_1}, \textit{methodOne\_2}, \textit{methodOne\_3}, \textit{methodOne\_3post}, \textit{methodOne\_4}, the hidden final instruction \textit{methodOne\_5}, \textit{methodTwo\_1}, \textit{methodTwo\_2}, \textit{methodTwo\_3}, \textit{methodTwo\_3post}, \textit{methodTwo\_4} and hidden final instruction \textit{methodTwo\_5}. We introduce the inertial predicates \textit{ObjT/3}, \textit{Obj2T/3}, \textit{FinishedT/3} and \textit{MTwoArgT/3}. We add causation rules corresponding to each instruction that contains an assignment to a variable or an object call. In the latter case, we add causation rules that set the correct callee object, call arguments, ensure that the value for \textit{SDPointAt/2} is the first instruction of the called diagram and set the return instruction for the next stack level to the hidden landing point after the call instruction. We also add causation rules corresponding to the final instruction of each diagram where the return instruction for the current stack depth is retrieved and set to be the value of \textit{SDPointAt/2} at the next time step. If any diagram modelled a non-void method, then we would add a causation rule corresponding to the landing point after a call of that diagram where the returned value is retrieved and is set to be the value of the assigned variable.

\begin{figure*}[!t]
\centering
\subfloat[Example of calling another diagram]{\includegraphics[width=0.325\textwidth]{methodOne}%
\label{fig:methodOne}}
\hfil
\subfloat[Example of recursion]{\includegraphics[width=0.4\textwidth]{methodTwo}%
\label{fig:methodTwo}}
\caption{Example of a set of sequence diagrams}
\label{fig:recursion}
\end{figure*}

With regard to \textit{SDPointAt/2}, we add a causation rule specifying that every \textit{SDPoint} is always followed by the \textit{SDPoint} named in \textit{NextSD/1} for that \textit{SDPoint}, except if that \textit{SDPoint} corresponds to a diagram call, the final instruction of a diagram or if it precedes a combined fragment. We have already discussed call instructions and final instructions of a diagram. We have designed elaborate algorithms that output the correct causation rules related to combined fragments, but we will only outline our general approach in this article.

We consider only alt combined fragments and loop combined fragments, which respectively correspond to an \textit{if}-\textit{else}-structure and a loop structure.

If the execution is about to enter a combined fragment, then we wish to determine in a single step which instruction the execution should jump to. In order to facilitate this, we have need of three algorithms:

\begin{itemize}
	\item Given a combined fragment, return the instructions the execution might jump to when entering the fragment and which conditions should hold for the jump to each instruction. Make the distinction between the \textit{if} and \textit{else} parts of an alt combined fragment.
	\item Given a combined fragment, determine all possible final instructions of the fragment and return all instructions the execution might jump to from each of those final instructions and which conditions, if any, should hold in order to make that jump.
	\item Given a combined fragment, determine all possible final instructions and check for loop combined fragments that the execution might re-enter. Return all instructions the execution might jump to and which conditions should hold in order to make that jump.
\end{itemize} 

Each of these algorithms should account for the fact that combined fragments might be nested and that the execution skips over loop combined fragments if their associated condition does not hold.

The general approach with regard to parsing combined fragment then goes as follows. If the message immediately preceding the combined fragment currently under consideration is not part of a combined fragment, then determine which are the first possible messages of this fragment and note that the execution jumps from that preceding message to those first messages if their corresponding conditions hold. If the immediately preceding message is part of a combined fragment, then determine the possible exit instructions for that combined fragment and use that information to determine which final messages for the preceding fragment jump to which possible first messages for the current fragment under which conditions. Finally, check if the current fragment is a loop combined fragment and/or is nested in one or more loop combined fragments and determine what instructions the execution might jump to when re-entering the loops. The output of this procedure is a set of causation rules for \textit{SDPointAt/3}.

We now have a set of rules that allows us to automatically generate a theory that accurately captures the behavior modelled in a set of mutually recursive sequence diagrams.

\section{Evaluation of translation from sequence diagrams to FO($\cdot$) theory}\label{sec:evaluation}

In this section, we will design a class diagram and a set of sequence diagrams that model the game of Nim. We will evaluate the difficulty of the design process. We will also evaluate the performance of simulation using the output theory and of verification of certain requirements using model expansion with respect to execution time and memory usage in terms of virtual memory usage for the simulation and grounding size for the verifications.

\subsection{Evaluation of the design}
Figures \ref{fig:nim-cd-play}, \ref{fig:ahe-ie} and \ref{fig:tt-take} contain the class diagram and sequence diagrams for our model of Nim. Nim is a fairly simple game, which means that the only real difficulty was designing a workaround for changing an object's association relations not being allowed during execution. This is why the class \textit{Game} has variables \textit{gameFinished} and \textit{p1Win}. These are a substitute for having a class \textit{Player} and an association \textit{Game}---\textit{Player} that designates a winner. Other limitations that contributed to making the diagrams longer than they otherwise could have been were that every alt combined fragment must have both an \textit{if} branch and an \textit{else} branch, and also that object call instructions must not contain expressions that require evaluation more complex than variable retrieval.

\subsection{Evaluation of simulation}

We simulated a game with two heaps. One of the heaps contains two objects and the other contains three objects. The first player removes all three objects from the second heap. The second player then removes one object from the first heap. Finally, the first player has no choice but to remove the last object and loses.
Table \ref{tab:sim-mem} contains the virtual memory usage at the start of each turn. It shows how simulating a game will require several gigabytes of memory.

Figure \ref{fig:boxplot} shows a boxplot of the execution time per time step, and figure \ref{fig:boxplot-nooutliers} shows a boxplot of the same data with the outliers removed. Half of the data lies between approximately 6.65 seconds and 6.9 seconds. Figure \ref{fig:boxplot-nooutliers} also shows the vast majority of the data is spread between approximately 6.5 seconds and 7.15 seconds. This distribution does lead to the total duration of the game being slightly less than 19 minutes, since the simulation was 116 steps long.

\subsection{Evaluation of verification for requirements}

The requirements we tested for are whether \textit{isEmpty()} returns true if and only if the heap the method is called on contains zero objects and whether \textit{allHeapsEmpty()} returns true if and only if all heaps are empty. To test this, we wrote four theories for each diagram which we each merge with the output theory one at a time for a total of four runs per diagram, one for each of these sentences:

\begin{itemize}
	\item The result is true while the heap is not empty/not all heaps are empty.
	\item The result is false while the heap is empty/all heaps are empty.
	\item The result is true while the heap is empty/all heaps are empty.
	\item The result is false while the heap is not empty/not all heaps are empty.
\end{itemize}

The expectation for both diagrams is that the first two sentences lead to an unsatisfiable theory, while the third and fourth sentences lead IDP to answer with the expected models. The third sentence for \textit{isEmpty()} should lead IDP to answer with the model corresponding to an empty heap while the fourth sentence should lead to an answer with all models corresponding to non-empty heaps. Similarly, the third sentence for \textit{allHeapsEmpty()} should lead to an answer with the model corresponding to all heaps being empty and the fourth sentence should lead to an answer with all models corresponding to at least one non-empty heap.

The input structure for \textit{isEmpty()} specifies that there are seven time steps in total, while for \textit{allHeapsEmpty()}, there are 25 time steps in total. Each run did indeed lead to the expected answer. For each diagram, the runs were similar in terms of execution time and grounding size. For \textit{isEmpty()}, the execution time is approximately 40 seconds, the size of the grounding is 578,306 and memory usage is 2.14 GB. For \textit{allHeapsEmpty()} given two heaps, the execution time is approximately 43.29 seconds, the grounding size is 366,643 and memory usage is 1.63 GB.

\begin{figure*}[!t]
	\centering
	\subfloat[Class diagram for Nim]{\includegraphics[width=0.5\textwidth]{ClassDiagram1}%
		\label{fig:nim-cd}}
	\hfil
	\subfloat[Sequence diagram for \textit{play()}]{\includegraphics[width=0.5\textwidth]{play}}%
		\label{fig:play}
	\caption{Class diagram for Nim and sequence diagram for \textit{play()}}
	\label{fig:nim-cd-play}
\end{figure*}

\begin{figure*}[!t]
	\centering
	\subfloat[Sequence diagram from \textit{allHeapsEmpty()}]{\includegraphics[width=0.6\textwidth]{allHeapsEmpty}%
		\label{fig:ahe}}
	\hfil
	\subfloat[Sequence diagram for \textit{isEmpty()}]{\includegraphics[width=0.4\textwidth]{isEmpty}}%
	\label{fig:isEmpty}
	\caption{Sequence diagrams for \textit{allHeapsEmpty()} and \textit{isEmpty()}}
	\label{fig:ahe-ie}
\end{figure*}

\begin{figure*}[!t]
	\centering
	\subfloat[Sequence diagram from \textit{takeTurn()}]{\includegraphics[width=0.5\textwidth]{takeTurn}%
		\label{fig:takeTurn}}
	\hfil
	\subfloat[Sequence diagram for \textit{take(int)}]{\includegraphics[width=0.5\textwidth]{take}}%
	\label{fig:take}
	\caption{Sequence diagrams for \textit{takeTurn()} and \textit{take(int)}}
	\label{fig:tt-take}
\end{figure*}

\begin{table}[]
	\centering
	\begin{tabular}{|l|l|l|l|l|}
		\hline
		Turn 1 & 0,91 GB  \\ \hline
		Turn 2 & 2,09 GB  \\ \hline
		Turn 3 & 3,34 GB  \\ \hline
		End   & 4,84 GB  \\ \hline
	\end{tabular}
	\caption{Virtual memory usage during the simulation at the start of each turn}
	\label{tab:sim-mem}
\end{table}

\begin{figure*}[!t]
	\centering
	\subfloat[Boxplot of the execution time per simulation step]{\includegraphics[width=0.5\textwidth]{boxplot}%
		\label{fig:boxplot}}
	\hfil
	\subfloat[Boxplot of the execution time per simulation step with the outliers removed]{\includegraphics[width=0.5\textwidth]{boxplotnooutliers}}%
		\label{fig:boxplot-nooutliers}
	\caption{Boxplots for the execution time per simulation step}
	\label{fig:boxplots}
\end{figure*}

% An example of a floating figure using the graphicx package.
% Note that \label must occur AFTER (or within) \caption.
% For figures, \caption should occur after the \includegraphics.
% Note that IEEEtran v1.7 and later has special internal code that
% is designed to preserve the operation of \label within \caption
% even when the captionsoff option is in effect. However, because
% of issues like this, it may be the safest practice to put all your
% \label just after \caption rather than within \caption{}.
%
% Reminder: the "draftcls" or "draftclsnofoot", not "draft", class
% option should be used if it is desired that the figures are to be
% displayed while in draft mode.
%
%\begin{figure}[!t]
%\centering
%\includegraphics[width=2.5in]{myfigure}
% where an .eps filename suffix will be assumed under latex, 
% and a .pdf suffix will be assumed for pdflatex; or what has been declared
% via \DeclareGraphicsExtensions.
%\caption{Simulation results for the network.}
%\label{fig_sim}
%\end{figure}

% Note that the IEEE typically puts floats only at the top, even when this
% results in a large percentage of a column being occupied by floats.


% An example of a double column floating figure using two subfigures.
% (The subfig.sty package must be loaded for this to work.)
% The subfigure \label commands are set within each subfloat command,
% and the \label for the overall figure must come after \caption.
% \hfil is used as a separator to get equal spacing.
% Watch out that the combined width of all the subfigures on a 
% line do not exceed the text width or a line break will occur.
%
%\begin{figure*}[!t]
%\centering
%\subfloat[Case I]{\includegraphics[width=2.5in]{box}%
%\label{fig_first_case}}
%\hfil
%\subfloat[Case II]{\includegraphics[width=2.5in]{box}%
%\label{fig_second_case}}
%\caption{Simulation results for the network.}
%\label{fig_sim}
%\end{figure*}
%
% Note that often IEEE papers with subfigures do not employ subfigure
% captions (using the optional argument to \subfloat[]), but instead will
% reference/describe all of them (a), (b), etc., within the main caption.
% Be aware that for subfig.sty to generate the (a), (b), etc., subfigure
% labels, the optional argument to \subfloat must be present. If a
% subcaption is not desired, just leave its contents blank,
% e.g., \subfloat[].


% An example of a floating table. Note that, for IEEE style tables, the
% \caption command should come BEFORE the table and, given that table
% captions serve much like titles, are usually capitalized except for words
% such as a, an, and, as, at, but, by, for, in, nor, of, on, or, the, to
% and up, which are usually not capitalized unless they are the first or
% last word of the caption. Table text will default to \footnotesize as
% the IEEE normally uses this smaller font for tables.
% The \label must come after \caption as always.
%
%\begin{table}[!t]
%% increase table row spacing, adjust to taste
%\renewcommand{\arraystretch}{1.3}
% if using array.sty, it might be a good idea to tweak the value of
% \extrarowheight as needed to properly center the text within the cells
%\caption{An Example of a Table}
%\label{table_example}
%\centering
%% Some packages, such as MDW tools, offer better commands for making tables
%% than the plain LaTeX2e tabular which is used here.
%\begin{tabular}{|c||c|}
%\hline
%One & Two\\
%\hline
%Three & Four\\
%\hline
%\end{tabular}
%\end{table}


% Note that the IEEE does not put floats in the very first column
% - or typically anywhere on the first page for that matter. Also,
% in-text middle ("here") positioning is typically not used, but it
% is allowed and encouraged for Computer Society conferences (but
% not Computer Society journals). Most IEEE journals/conferences use
% top floats exclusively. 
% Note that, LaTeX2e, unlike IEEE journals/conferences, places
% footnotes above bottom floats. This can be corrected via the
% \fnbelowfloat command of the stfloats package.




\section{Conclusion}\label{sec:conclusion}
We have shown how to translate a class diagram and an accompanying set of sequence diagrams that model the behavior of the methods defined in the class diagram to an FO($\cdot$) theory. We have also shown how that theory may be used to verify consistency of the class diagram, to simulate the behavior of the system and to verify requirements on individual diagrams. There is also a translation of a class diagram to an FO($\cdot$) theory in an alternate form. This theory may be used to detect the presence of certain design flaws in a class diagram.

The results from section \ref{sec:evaluation} indicate that execution time and memory usage in terms of grounding size are already high even for diagrams that model simple games such as Nim. The way forward would be to find ways to make the output theory as small as possible. The most promising avenue would likely be to introduce declarative elements to the modelling language of sequence diagrams. Examples would be an instruction that extracts all elements that satisfy a query from a variable representing a collection and an instruction that allows the user to choose one or more elements that satisfy a query from a collection during simulation.


% conference papers do not normally have an appendix


% use section* for acknowledgment
%\section*{Acknowledgment}
%
%
%The authors would like to thank...





% trigger a \newpage just before the given reference
% number - used to balance the columns on the last page
% adjust value as needed - may need to be readjusted if
% the document is modified later
%\IEEEtriggeratref{8}
% The "triggered" command can be changed if desired:
%\IEEEtriggercmd{\enlargethispage{-5in}}

% references section

% can use a bibliography generated by BibTeX as a .bbl file
% BibTeX documentation can be easily obtained at:
% http://mirror.ctan.org/biblio/bibtex/contrib/doc/
% The IEEEtran BibTeX style support page is at:
% http://www.michaelshell.org/tex/ieeetran/bibtex/
\bibliographystyle{IEEEtran}
% argument is your BibTeX string definitions and bibliography database(s)
\bibliography{citations}{}
%
% <OR> manually copy in the resultant .bbl file
% set second argument of \begin to the number of references
% (used to reserve space for the reference number labels box)
%\begin{thebibliography}{1}
%
%\bibitem{IEEEhowto:kopka}
%H.~Kopka and P.~W. Daly, \emph{A Guide to \LaTeX}, 3rd~ed.\hskip 1em plus
%  0.5em minus 0.4em\relax Harlow, England: Addison-Wesley, 1999.
%
%\end{thebibliography}

% that's all folks
\end{document}




\chapter{Poster}

\includepdf[pages=1,fitpaper]{postera4.pdf}
	\printurls
	
	\bibliography{citations}{}
	\bibliographystyle{plain}
	
	\backmatter
	
\end{document}